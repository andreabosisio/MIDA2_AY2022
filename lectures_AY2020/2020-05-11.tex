\newlecture{Sergio Savaresi}{11/05/2020}

\begin{remark}[White noise]
    In the formulas of Kalman Filter there is a requirement that $v_1(t)$ and $v_2(t)$ must be white noises.
    In many practical applications this assumption can be too demanding.

    We need a workaround to deal with practical applications where this assumption is not valid.

    \[
        \begin{cases}
            x(t+1) = ax(t) + \eta(t) \qquad & \eta(t) \text{ is not a white noise}\\
            y(t) = bx(t) + v_2(t) \qquad & v_2(t) \sim WN(0,1)
        \end{cases}
    \]

    A model of $\eta(t)$ is given:
    \[
        \eta(t) = \frac{1}{1-cz^{-1}}e(t) \qquad e(t) \sim WN(0,1)
    \]
    $\eta(t)$ is not a white noise, but an AR(1) stochastic model.
    We cannot apply K.F. formula to this system

    We can proceed as follows
    \begin{align*}
        \eta(t) &= c\eta(t-1) + e(t) \\
        \eta(t+1) &= c\eta(t) + e(t+1) \\
        \eta(t+1) &= c\eta(t) + v(t) \qquad v(t) = e(t+1) \qquad v \sim WN(0,1) \qquad v \perp v_2
    \end{align*}

    \paragraph{Trick} Extension of the state vector.

    \begin{align*}
        x(t) \mapsto x_1(t) \\
        \eta(t) \mapsto x_2(t)
    \end{align*}
    \[
        \begin{cases}
            x_1(t+1) = ax_1(t) + x_2(t) \\
            x_2(t+1) = cx_2(t) + v(t) \\
            y(t) = bx_1(t) + v_2(t)
        \end{cases}
    \]

    \[
        n=2 \quad F\begin{bmatrix}
            a & 1 \\
            0 & c
        \end{bmatrix}
        \quad H = \begin{bmatrix}
            b & 0
        \end{bmatrix}
        \quad v_1 = \begin{bmatrix}
            0 \\ v(t)
        \end{bmatrix}
        \quad V_1 = \begin{bmatrix}
            0 & 0 \\
            0 & 1
        \end{bmatrix}
        \quad v_2 \sim WN(0,1)
        \quad V_{12} = 0
    \]

    We can apply K.F. to this system.
\end{remark}

\subsection{Extension to Non-Linear systems}

Consider a system with non-linear dynamics:
\[
    S: \begin{cases}
        x(t+1) = f(x(t), u(t)) + v_1(t) \\
        y(t) = h(x(t)) + v_2(t)
    \end{cases}
\]

Where $f$ and $h$ are non-linear functions of $x(t)$ and $u(t)$ (smoothness class $C^1$ or higher).

\begin{example}
    \[
        S: \begin{cases}
            x(t+1) = \frac{1}{2} x^5(t) + u^3(t) + v_1(t) \\
            y(t) = e^{x(t)} + v_2(t)
        \end{cases}
    \]
\end{example}

How can we design a Kalman Filter in this case?

\begin{figure}[H]
    \centering
    \begin{tikzpicture}[node distance=2.5cm,auto,>=latex']
        \node [int] (z1) at (1,4) {$z^{-1}$};
        \node [int, double, double distance=0.5mm] (F1) at (-1.5,3) {$f$};
        \node [int] (K) at (1,2) {$?$};
        \node [int] (z2) at (1,1) {$z^{-1}$};
        \node [int, double, double distance=0.5mm] (F2) at (-1.5,0) {$f$};

        \node [int, double, double distance=0.5mm] (H1) at (4,4) {$h$};
        \node [int, double, double distance=0.5mm] (H2) at (4,1) {$h$};

        \node [sum] (sum1) at (6,4) {};
        \node [sum] (sum2) at (7,2) {};
        \node [sum] (sum3) at (-1.5,4) {};
        \node [sum] (sum4) at (-1.5,1) {};

        \node (v1) at (-1.5,5) {$v_1(t)$};
        \node (v2) at (6,5) {$v_2(t)$};
        \node[right] (y) at (8,4) {$y(t)$};
        \node[right] (yhat) at (8,1) {$\hat{y}(t|t-1)$};
        \node[right] (xhat) at (8,0) {$\hat{x}(t|t-1)$};

        \draw[->] (v1) -- (sum3);
        \draw[->] (sum3) -- (z1) node[pos=0.5] {$x(t+1)$};
        \draw[->] (F1) -- (sum3);
        \draw[->] (z1) -- (H1) node[pos=0.5] {$x(t)$};
        \draw[->] (H1) -- (sum1);
        \draw[->] (v2) -- (sum1);
        \draw[->] (2.5,4) |- (F1);
        \draw[->] (K) -| (sum4);
        \draw[->] (sum4) -- (z2) node[pos=0.5] {$\hat{x}(t+1|t)$};
        \draw[->] (7,4) -- (sum2) node[pos=0.8] {$+$};
        \draw[->] (sum1) -- (y);
        \draw[<-] (K) -- (sum2) node[pos=0.5,black] {$e(t)$};
        \draw[->] (z2) -- (H2) node[pos=0.5] {$\hat{x}(t|t-1)$};
        \draw[->] (F2) -- (sum4);
        \draw[->] (2.5,1) |- (F2);
        \draw[->] (H2) |- (yhat);
        \draw[->] (7,1) -- (sum2) node[pos=0.8] {$-$};
        \draw[->] (F2) -- (xhat);
    \end{tikzpicture}
\end{figure}

For the gain block of K.F. we have 2 different types of solutions:
\begin{enumerate}
    \item (most natural and intuitive) The gain is a non-linear function of $e(t)$
    \item The gain is a linear time-varying function
\end{enumerate}

The second solution is less intuitive but is the most effective: we can reuse in full K.F. formulas with small adjustments.
K.F. can be applied to LTV systems.

In practice Extended Kalman Filter idea is to make a time-varying linear local approximation of a non-linear time-invariant system.

$K(t)$ in Extended Kalman Filter can be computed as:
\[
    K(t) = \left( F(t) P(t) H(t)^T + V_{12} \right) \left( H(t) P(t) H(T)^T + V_2 \right)^{-1}
\]
And $P(t)$ can be computed from D.R.E.
\[
    P(t) = \left( F(t)P(t)F(t)^T+V_1 \right) - \left( F(t)P(t)H(t)^T + V_{12}\right)\left( H(t)P(t)H(t)^T + V_2\right)^{-1}\left( F(t)P(t)H(t)^T + V_{12}\right)^T
\]

\begin{itemize}
    \item Equations of $K$ and DRE are the usual formulas of K.F.
    \item The only difference is $F(t)$ and $H(t)$ are computed as follows
\end{itemize}

\begin{align*}
    F(t) &=  \left.\frac{ \partial f(x(t), u(t))}{\partial x(t)} \right|_{x(t) = \hat{x}(t|t-1)} \\
    H(t) &= \left. \frac{\partial h(x(t))}{\partial x(t)} \right|_{x(t) = \hat{x}(t|t-1)}
\end{align*}

EKF is the time-varying solution of K.F. where $F(t)$ and $H(t)$ are local linearized matrices computed around the last available state prediction $\hat{x}(t|t-1)$.

Procedure to implement EKF at time $t$:
\begin{itemize}
    \item Take last available state prediction $\hat{x}(t|t-1)$
    \item Using $\hat{x}(t|t-1)$, compute $F(t)$ and $H(t)$
    \item Compute $K(t)$ and update D.R.E. equations
    \item Compute $\hat{x}(t+1|t)$
\end{itemize}

\begin{remark}
    \begin{itemize}
        \item EKF is very powerful since can be applied to non-linear systems
        \item Obviously EKF does not have a steady-state asymptotic solution
        \item Main problems with EKF (same of L.T.V. K.F.):
        \begin{itemize}
            \item Very difficult (almost impossible) to have a theoretical guarantee of E.K.F. stability (in practice extensive empirical testing).
            \item Computational load (at each time $F(t)$, $H(t)$, $K(t)$ and $P(t)$ must be computed run-time).
        \end{itemize}
    \end{itemize}

    EKF is largely used today with some limitations in:
    \begin{itemize}
        \item safety-critical applications
        \item mission-critical applications
    \end{itemize}
\end{remark}

\begin{exercise}[K.F. full procedure]
    \begin{figure}[H]
        \centering
        \begin{tikzpicture}[node distance=2.5cm,auto,>=latex']
            \draw[decoration={aspect=0.3, segment length=1.5mm, amplitude=3mm,coil},decorate] (-1,1) -- (-1,2);
            \draw (0,2) -- (0,2.5);
            \draw (-1,2) -- (1,2);
            \draw (-1,1) -- (1,1);

            \draw (1,2) -- (1,1.5);
            \draw (0.8, 1.7) -- (0.8, 1.3) -- (1.2, 1.3) -- (1.2, 1.7);
            \draw (1,1) -- (1,1.3);
            \draw (0,0.7) -- (0,1);

            \fill (-0.2,0.5) rectangle (0.2,0.7);
            \draw (-1.5,2.5) -- (1.5,2.5) -- (1.5,5.5) -- (1.1,5.5) -- (1.1,2.9) -- (-1.5,2.9) -- (-1.5,2.5);

            \draw[pattern=north east lines] (-3,0) rectangle (3,0.5);
            \draw (-3,0.5) arc[radius=6,start angle=180, end angle=90] -- (3,0.5);

            \draw[->] (0.4,0.55) -- (0.4,0.9) node[right,pos=0.5] {\footnotesize vertical acc.};
            \draw[->] (-1.4,0.55) -- (-1.4,2.45) node[left,pos=0.5] {\footnotesize elongation};
        \end{tikzpicture}
    \end{figure}

    Suspended seat in the cabin of an off-highway vehicle (agr. tractor, earth-moving machine, etc\dots).

    The physical sensors are:
    \begin{itemize}
        \item Vertical accelerometer placed at the basis of the cabin
        \item Elongation sensor between the basis of the cabin and the seat
    \end{itemize}

    \paragraph{Problem} Estimation of the seat vertical speed.

    Move from a pictorial representation to a schematic representation of it:
    \begin{figure}[H]
        \centering
        \begin{tikzpicture}[node distance=2.5cm,auto,>=latex']
            \draw (-1,2.5) rectangle (1,3.5);
            \draw[decoration={aspect=0.3, segment length=1.5mm, amplitude=3mm,coil},decorate] (-1,1) -- (-1,2);
            \draw (0,2) -- (0,2.5);
            \draw (-1,2) -- (1,2);
            \draw (-1,1) -- (1,1);

            \draw (1,2) -- (1,1.5);
            \draw (0.8, 1.7) -- (0.8, 1.3) -- (1.2, 1.3) -- (1.2, 1.7);
            \draw (1,1) -- (1,1.3);
            \draw (0,0.7) -- (0,1);

            \draw plot[smooth, tension=1] coordinates {(-2, 0) (0,0.7) (2,0)};

            \fill [pattern=north east lines] (2.1,0) rectangle (2.7,-0.1);

            \draw (2.1,0) -- (2.7,0);
            \draw[dotted] (0,1) -- (2.2,1);
            \draw[dotted] (0,2.5) -- (2.6,2.5);

            \draw [->] (2.2,0) -- (2.2,1);
            \draw [->] (2.6,0) -- (2.6,2.5);

            \node at (0,3) {$M$};
            \node[right] at (1.1,3.7) {\footnotesize full mass of seat+driver};

            \node at (-1.7,1.5) {$K$};
            \node[left] at (-1.4,1.8) {\footnotesize spring stiffness};

            \node at (1.5,1.5) {$C$};
            \node[right] at (1.3,1.8) {\footnotesize damping ratio};

            \node[left] at (2.2,0.5) {$z_d$};
            \node[right] at (2.6,1.25) {$z$};
        \end{tikzpicture}
    \end{figure}

    The sensor are:
    \begin{itemize}
        \item Acceleration $\ddot{z}_d$ (+noise)
        \item Elongation $z-z_d$ (+noise)
    \end{itemize}

    Model of the system dynamics (physical white-box model in continuous time domain).

    Core model equation (force balance in vertical direction):
    \begin{align*}
        M\ddot{z} &= \underbrace{-C\frac{d}{dt}(z-z_d)}_\text{damping} \underbrace{-K(z-z_d)}_{spring} \cancel{+ \text{gravity} - Mg} \\
                  &= -C(\dot{z} - \dot{z}_d) - K(z-z_d)
    \end{align*}

    Since we measure $\ddot{z}_d$ the overall dimension of the system is 4.
    The vector of state variables is:
    \[
        x(t) = \begin{bmatrix}
            z \\
            \dot{z} \\
            z_d \\
            \dot{z}_d
        \end{bmatrix} = \begin{bmatrix}
            x_1(t) \\
            x_2(t) \\
            x_3(t) \\
            x_4(t) \\
        \end{bmatrix}
        \qquad
        u(t) = \ddot{z}_d
        \qquad
        y(t) = z-z_d
    \]

    We can write the full model in state space form
    \[
        \begin{cases}
            \dot{x}_1 = x_2 \\
            \dot{x}_2 = -\frac{C}{M} (x_2-x_4) - \frac{K}{M}(x_1-x_3) \\
            \dot{x}_3 = x_4 \\
            \dot{x}_4 = u \\
            y = x_1-x_3
        \end{cases}
        \quad\Rightarrow\quad
        \begin{cases}
            \dot{x}_1 = x_2 \\
            \dot{x}_2 = -\frac{K}{M}x_1 - \frac{C}{M}x_2 + \frac{K}{M}x_3 + \frac{C}{M} x_4 \\
            \dot{x}_3 = x_4 \\
            \dot{x}_4 = u \\
            y = x_1-x_3
        \end{cases}
    \]
    \[
        A = \begin{bmatrix}
            0 & 1 & 0 & 0 \\
            -\frac{K}{M} & -\frac{C}{M} & \frac{K}{M} & \frac{C}{M} \\
            0 & 0 & 0 & 1 \\
            0 & 0 & 0 & 0
        \end{bmatrix}
        \quad
        B = \begin{bmatrix}
            0 \\ 0 \\ 0 \\ 1
        \end{bmatrix}
        \quad
        C = \begin{bmatrix}
            1 & 0 & -1 & 0
        \end{bmatrix}
    \]

    Next is discretization (we use digital systems), choice of sampling time $\Delta$ (for this application can be 5ms).

    Using Eulero-forward approximation of time derivative $\dot{x}(t) \approx \frac{x(t+1)-x(t)}{\Delta}$.

    Example for the first equation:
    \[
        \frac{x_1(t+1)-x_1(t)}{\Delta} = x_2(t) \quad\Rightarrow\quad x_1(t+1) = x_1(t) + \Delta x_2(t)
    \]

    \[
        \begin{cases}
            x_1(t+1) = x_1(t) + \Delta x_2(t) + v_{11}(t) \\
            x_2(t+1) = -\frac{\Delta K}{M} x_1(t) + \left( -\frac{\Delta C}{M} + 1 \right)x_2(t) + \frac{\Delta K}{M} x_3(t) + \frac{\Delta C}{M} x_4(t) + v_{12}(t) \\
            x_3(t+1) = x_3(t) + \Delta x_4(t) + v_{13}(t) \\
            x_4(t+1) = x_4(t) + \Delta u(t) + v_{14}(t) \\
            y(t) = x_1(t) - x_3(t) + v_2(t)
        \end{cases}
    \]

    This is normal state-space for in discrete time:
    \[
        \begin{cases}
            x(t+1) = Fx(t) + Gu(t) \\
            y(t) = Hx(t)
        \end{cases}
    \]
    \[
        F = \begin{bmatrix}
            1 & \Delta & 0 & 0 \\
            -\frac{\Delta K}{M} & -\frac{\Delta C}{M}+1 & \frac{\Delta K}{M} & \frac{\Delta C}{M} \\
            0 & 0 & 1 & \Delta \\
            0 & 0 & 0 & 1
        \end{bmatrix}
        \quad G = \begin{bmatrix}
            0 \\ 0 \\ 0 \\ \Delta
        \end{bmatrix}
        \quad H = \begin{bmatrix}
            1 & 0 & -1 & 0
        \end{bmatrix}
    \]

    Noise on the state equations:
    \[
        v_1(t) = \begin{bmatrix}
            v_{11}(t) \\
            v_{12}(t) \\
            v_{13}(t) \\
            v_{14}(t)
        \end{bmatrix}
        \sim WN(0, V_1)
        \qquad
        V_1 = \begin{bmatrix}
            \lambda_1^2 & 0 & 0 & 0\\
            0 & \lambda_1^2 & 0 & 0 \\
            0 & 0 & \lambda_1^2 & 0 \\
            0 & 0 & 0 & \lambda_1^2 \\
        \end{bmatrix}
        \qquad
        v_2(t) \sim WN(0, V_2)
    \]
    Assumptions:
    \begin{itemize}
        \item All white-noises
        \item All uncorrelated
    \end{itemize}

    $V_2$ can be estimated by datasheet of elongation sensor, $\lambda_4$ can be estimated by datasheet of accelerometer.
    We expect $\lambda_1$, $\lambda_2$ and $\lambda_3$ to be small, and we can use the simplifying assumption $\lambda_1^2=\lambda_2^2=\lambda_3^2 =\lambda^2$, estimated empirically.


    \begin{figure}[H]
        \centering
        \begin{tikzpicture}[node distance=2.5cm,auto,>=latex']
            \node[left] (u) at (0,0) {$u(t)$};
            \node[int] (d1) at (1,0) {$\Delta$};
            \node[sum] (s1) at (2,0) {};
            \node[above] (v14) at (2,0.5) {$v_{14}(t)$};
            \node[int] (z4) at (4,0) {$z^{-1}$};
            \node[int] (b4) at (6,1) {$\frac{\Delta C}{M}$};
            \node[int] (d2) at (4,1) {$\Delta$};

            \node[sum] (s2) at (2,2.5) {};
            \node[above] (v13) at (2,3) {$v_{13}(t)$};
            \node[int] (z3) at (4,2.5) {$z^{-1}$};
            \node[int] (b3) at (6,4.5) {$\frac{\Delta K}{M}$};

            \node[sum] (s3) at (7.5,4.5) {};
            \node[above] (v12) at (7.5,5) {$v_{12}(t)$};
            \node[int] (z2) at (9.5,4.5) {$z^{-1}$};
            \node[int] (b2) at (9.5,3.5) {$-\frac{\Delta C}{M}+1$};

            \node[sum] (s4) at (2,8) {};
            \node[above] (v11) at (2,8.5) {$v_{11}(t)$};
            \node[int] (z1) at (4,8) {$z^{-1}$};
            \node[int] (b1) at (6,5.5) {$-\frac{\Delta K}{M}$};
            \node[int] (d3) at (4,6.5) {$\Delta$};

            \node[sum] (s5) at (11.5,2.5) {};
            \node[below] (v2) at (11.5,2) {$v_2(t)$};
            \node[right] (y) at (12,2.5) {$y(t)$};

            \draw[->] (u) -- (d1);
            \draw[->] (d1) -- (s1);
            \draw[->] (v14) -- (s1);
            \draw[->] (s1) -- (z4) node[pos=0.5] {\footnotesize $x_4(t+1)$};
            \draw[->] (z4) -- (8,0) node[above] {\footnotesize $x_4(t)$};
            \draw[->] (4.7,0) -- (4.7,-0.5) -- (2,-0.5) -- (s1);
            \draw[->] (4.7,0) |- (d2);
            \draw[->] (6,0) -- (b4);

            \draw[->] (d2) -| (s2);
            \draw[->] (v13) -- (s2);
            \draw[->] (s2) -- (z3) node[pos=0.5] {\footnotesize $x_3(t+1)$};
            \draw[->] (z3) -- (s5) node[pos=0.1] {\footnotesize $x_3(t)$};
            \draw[->] (4.7,2.5) -- (4.7,2) -- (2.5,2) -- (s2);
            \draw[->] (6,2.5) -- (b3);

            \draw[->] (b3) -- (s3);
            \draw[->] (v12) -- (s3);
            \draw[->] (b4) -- (7,1) -- (7,4) -- (s3);
            \draw[->] (b1) -- (7,5.5) -- (7,5) -- (s3);
            \draw[->] (s3) -- (z2) node[pos=0.5] {\footnotesize $x_2(t+1)$};
            \draw[->] (z2) -- (11,4.5) node[above] {\footnotesize $x_2(t)$};
            \draw[->] (10.5,4.5) |- (b2);
            \draw[->] (b2) -| (s3);

            \draw[->] (v11) -- (s4);
            \draw[->] (s4) -- (z1) node[pos=0.5] {\footnotesize $x_1(t+1)$};
            \draw[->] (10.5,4.5) |- (d3);
            \draw[->] (d3) -| (s4);
            \draw[->] (4.7,8) -- (4.7,7.5) -- (2.5,7.5) -- (s4);
            \draw[->] (z1) -| (s5) node[pos=0.05] {\footnotesize $x_1(t)$};
            \draw[->] (6,8) -- (b1);

            \draw[->] (v2) -- (s5);
            \draw[->] (s5) -- (y);
        \end{tikzpicture}
    \end{figure}

    We can compute
    \[
        O = \begin{bmatrix}
            H \\
            HF \\
            HF^2 \\
            HF^3
        \end{bmatrix}
        \qquad
        R = \begin{bmatrix}
            G & FG & F^2G & F^3G
        \end{bmatrix}
    \]
    From visual inspection of the block schema we expect full observability from output and full controllability from input.

    K.F. at this point can be applied.
    Theorem 1 and 2 are valid, we can directly jump to A.R.E. solution.
\end{exercise}
