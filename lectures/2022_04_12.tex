%!TEX root = ../main.tex

\paragraph{Step 1} Build the Hankel matrix from data using ``one-shot'' all the available $N$ data points.

\[
    \tilde{H}_{qd} = \begin{bmatrix}
        \tilde{\omega}(1) & \tilde{\omega}(2) & \cdots & \tilde{\omega}(d) \\
        \tilde{\omega}(2) & \tilde{\omega}(3) & \cdots & \tilde{\omega}(d+1) \\
        \vdots            & \vdots            & \ddots & \vdots \\
        \tilde{\omega}(q) & \tilde{\omega}(q+1) & \cdots & \tilde{\omega}(q+d-1) \\
    \end{bmatrix}
\]

$\tilde{H}_{qd}$ is a $q\times d$ matrix.\\
\textbf{Note} that, as before, we start from $\tilde{\omega}(1)$ since we are assuming $\tilde{\omega}(0)=0$. \\
\textbf{Note} that $q+d-1$ must equal to $N$ so that we use all the dataset.


\begin{remark}[Choice of $q$ and $d$]
    Hypothesis: $q<d$ so that $q+d-1=N \Rightarrow q=N+1-d$.
    \begin{figure}[H]
        \centering
        \begin{tikzpicture}[node distance=2.5cm,auto,>=latex']
            \draw[->] (-0.5,0) -- (3,0) node[right] {$d$};
            \draw[->] (0,-0.5) -- (0,3) node[left] {$q$};
            \draw[domain=0.3:2.7,smooth,variable=\x,black] plot ({\x},{3-\x});
            \draw[line width=0.6mm,domain=1.5:1.8,smooth,variable=\x,green] plot ({\x},{3-\x});
            \draw[line width=0.6mm,domain=2.3:2.7,smooth,variable=\x,blue] plot ({\x},{3-\x});
            \draw[dotted] (0,0) -- (1.5,1.5);

            \node at (2.3,1.5) {$q \approx d$};
            \node at (3.1,0.7) {$q \ll d$};

            \node[below] at (0.3,0) {$1$};
            \node[below] at (2.7,0) {$N$};
            \node[left] at (0,0.3) {$1$};
            \node[left] at (0,2.7) {$N$};
        \end{tikzpicture}
    \end{figure}

    If $q \approx d$ the method has better accuracy but high computational effort.
    
    If $q \ll d$ it's computationally less intensive but has the worst accuracy.

    \textbf{Rule of thumb:} If $ q > \frac{d}{2}$ we get to a good enough result.
\end{remark}

\paragraph{Step 2} \emph{Singular Value Decomposition (SVD)} of $\tilde{H}_{qd}$

\[
    \underbrace{\tilde{H}_{qd}}_{q\times d} = \underbrace{\tilde{U}}_{q\times q} \underbrace{\tilde{S}}_{q\times d} \underbrace{\tilde{V}\transpose}_{d\times d}
\]

where $\tilde{U}$ and $\tilde{V}$ are \emph{square} and \emph{unitary} matrices and $\tilde{S}$ is a \emph{rectangular diagonal} matrix such that:

\[
    \tilde{S} = \begin{bmatrix}
        \sigma_1 & & & & &\\
        & \sigma_2 & & & &\\
        & & \ddots & & &\\
        & & & \sigma_d & &\\
    \end{bmatrix}
\]

where $\sigma_1$, $\sigma_2$, $\ldots$, $\sigma_q$ are the \emph{singular values} of $\tilde{H}_{qd}$.
Those are real, positive numbers, sorted in decreasing order (i.e. $\sigma_1 \ge \sigma_2 \ge \cdots \ge \sigma_q$).

\begin{definition}[Unitary matrix]
    A matrix $M$ is \emph{unitary} if:
    \begin{itemize}
        \item $\det M = 1$ (so it's invertible)
        \item $M^{-1} = M\transpose$
    \end{itemize}
\end{definition}

\begin{obs}
    The singular values of a rectangular matrix are a \emph{sort of eigenvalues} of a square matrix.\\
    SVD is a \emph{sort of diagonalization} of a rectangular matrix.
\end{obs}

\begin{recall}
    For a square matrix, $\text{eig}(A) = \text{roots}(\det(A-\lambda I))$. If $M$ is rectangular, $SV(M) = \sqrt{\text{eig}(MM\transpose)}$ (for non zero eigenvalues).
\end{recall}

\begin{remark}[How to compute SVD]
    The optimal numerical computation is not trivial. 

    Theoretical method for SVD computation is to make 2 diagonalization steps:
    \[
        \underbrace{\tilde{H}_{qd} \tilde{H}_{qd}\transpose}_{q\times q} = \tilde{U}\tilde{S}\tilde{S}\transpose\tilde{U}\transpose
    \]
    \[
        \underbrace{\tilde{H}_{qd}\transpose \tilde{H}_{qd}}_{d\times d} = \tilde{V}\tilde{S}\transpose\tilde{S}\tilde{V}\transpose
    \]
    
    Instead, use \texttt{svd(M)} in Matlab.
\end{remark}

\paragraph{Step 3} Plot the singular values and separate (cut-off) the system from noise.

\begin{figure}[H]
    \begin{minipage}[t]{0.5\textwidth}
        \centering
        \begin{tikzpicture}[node distance=2.5cm,auto,>=latex']
            \draw[->] (0,0) -- (5,0) node[right] {$i$};
            \draw[->] (0,0) -- (0,2) node[left] {$\sigma_i$};
            \node at (0.5,1.8) {\textbullet};
            \node at (1.0,1.3) {\textbullet};
            \node at (1.5,1.0) {\textbullet};
            \node at (2.0,0.9) {\textbullet};
            \node at (2.5,0.8) {\textbullet};
            \node at (3.0,0.2) {\textbullet};
            \node at (3.5,0.2) {\textbullet};
            \node at (4.0,0.2) {\textbullet};
            \node at (4.5,0.2) {\textbullet};
            \draw (2.5,-0.1) -- (2.5,0.1);
            \node at (2.5,-0.4) {$n$};

            \draw[decoration={brace}, decorate] (0.5,2) node {} -- (2.5,1);
            \node at (2.5,1.8) {signal SV};

            \draw[decoration={brace}, decorate] (3,0.4) node {} -- (4.5,0.4);
            \node at (3.75,0.8) {noise SV};
        \end{tikzpicture}
        \caption*{Ideal case}
    \end{minipage}
    \begin{minipage}[t]{0.5\textwidth}
        \centering
        \begin{tikzpicture}[node distance=2.5cm,auto,>=latex']
            \draw[->] (0,0) -- (5,0) node[right] {$i$};
            \draw[->] (0,0) -- (0,2) node[left] {$\sigma_i$};
            \node at (0.5,1.8) {\textbullet};
            \node at (1.0,1.3) {\textbullet};
            \node at (1.5,1.0) {\textbullet};
            \node at (2.0,0.8) {\textbullet};
            \node at (2.5,0.6) {\textbullet};
            \node at (3.0,0.4) {\textbullet};
            \node at (3.5,0.3) {\textbullet};
            \node at (4.0,0.2) {\textbullet};
            \node at (4.5,0.2) {\textbullet};
            \draw (2,-0.1) rectangle ++(1,0.2);
            \node at (2.5,-0.4) {$n$};

            \draw[decoration={brace}, decorate] (2,1) node {} -- (3,1);
            \node at (2.5,1.4) {transition};
        \end{tikzpicture}
        \caption*{Real case}
    \end{minipage}
\end{figure}

In the ideal case there is a perfect/clear separation between the signal and the noise S.V. (a jump).
The index of the jump is $n$, that is the order of the system.

In the real case there is no clear distinction between signal and noise singular values, since the order of the system can assume values in an interval.
With some empirical test we can select a good compromise between complexity, precision and overfitting (see \emph{cross-validation}).

After the decision on the value of $n$ we split $\tilde{H}_{qd}$ in $\tilde{U}$, $\tilde{S}$ and $\tilde{V}\transpose$:

\begin{figure}[H]
    \centering
    \begin{tikzpicture}
        \tikzset{BarreStyle/.style = {opacity=.3,line width=4 mm,color=blue}}
        \node (H) {$\tilde{H}_{qd} =$};
        \matrix (U) [right of=H, node distance=2cm,matrix of math nodes, nodes in empty cells,left delimiter={[},right delimiter={]},text depth=0ex,text height=1ex,text width=1ex]
        {
            & & & & \\
            & & & & \\
            \hat{U} & & \tilde{U} & & \\
            & & & & \\
            & & & & \\
        };
        \matrix (S) [right of=U,node distance=3cm,matrix of math nodes, nodes in empty cells,left delimiter={[},right delimiter={]},text depth=0ex,text height=1ex,text width=1ex]
        {
            \hat{S} & & & & \\
            & & & & \\
            & & \tilde{S} & & \\
            & & & & \\
            & & & & \\
        };
        \matrix (V) [right of=S,node distance=3cm,matrix of math nodes, nodes in empty cells,left delimiter={[},right delimiter={]},text depth=0ex,text height=1ex,text width=1ex]
        {
            & & \hat{V}\transpose & & \\
            & & & & \\
            & & \tilde{V}\transpose & & \\
            & & & & \\
            & & & & \\
        };

        \draw[decoration={brace}, decorate] (1,1.2) node {} -- (1.5,1.2);
        \node at (1.25,1.5) {$q\times n$};
        \draw[decoration={brace}, decorate] (4,1.2) node {} -- (4.5,1.2);
        \node at (4.25,1.5) {$n\times n$};
        \draw[decoration={brace}, decorate] (7,1.2) node {} -- (9,1.2);
        \node at (8,1.5) {$n\times d$};

        \draw [BarreStyle] (U-1-1.north) to (U-5-1.south);
        \draw [BarreStyle] (S-1-1.west) to (S-1-1.east);
        \draw [BarreStyle] (V-1-1.west) to (V-1-5.east);
    \end{tikzpicture}
\end{figure}


\[
    \tilde{H}_{qd} = \underbrace{\hat{U} \hat{S} \hat{V}\transpose}_{\hat{H}_{qd}} + H_{res,qd} \qquad \rank (\tilde{H}_{qd}) = q \quad \rank (\hat{H}_{qd}) = n \quad \rank (H_{res,qd}) = q
\]
where $\hat{H}_{qd}$ is the \emph{"signal part"} and $H_{res,qd}$ is the \emph{"noise (or residual) part"} of $\tilde{H}_{qd}$.\\

\textbf{Note:} $\hat{S}$ is now a square diagonal matrix of $\sigma_1, \dots, \sigma_q$.\\
\textbf{Note:} from $\tilde{H}_{qd}$ to $\hat{H}_{qd}$ the rank is hugely reduced.\\
\textbf{Note:} $\hat{H}_{qd}$ is the "cleaned" Hankel matrix.

\paragraph{Step 4} Estimation of $\hat{F}$, $\hat{G}$ and $\hat{H}$ using the cleaned matrix $\hat{H}_{qd}$

\[
    \hat{H}_{qd} = \hat{U} \hat{S} \hat{V}\transpose = \hat{U} \hat{S}^{\frac{1}{2}} \hat{S}^{\frac{1}{2}} \hat{V}\transpose
\]

where $\hat{S}^{\frac{1}{2}}$ is the square diagonal matrix with elements the square roots of the elements of $\hat{S}$, that is $\hat{S}^{\frac{1}{2}}=diag(\sqrt{\sigma_1}, \dots, \sqrt{\sigma_q})$. \\

Defining $\hat{O} = \hat{U}\hat{S}^{\frac{1}{2}}$ and $\hat{R} = \hat{S}^{\frac{1}{2}} \hat{V}\transpose$, $\hat{H}_{qd}$ become 
\[\hat{H}_{qd} = \underbrace{\hat{O}}_{q\times n} \underbrace{\hat{R}}_{n\times d}\]


We can view $\hat{O}$ as the \emph{extended observability matrix} and the $\hat{R}$ the \emph{extended reachability matrix} of the system. \\

Now it is possible to estimate $\hat{H}$ with the first row of $\hat{O}$ and $\hat{G}$ with the first column of $\hat{R}$, similarly to the \nameref{step3} of the noise-free case.

\begin{align*}
    \hat{G} = \hat{R}(\texttt{:;1}) & \quad\text{(first column of $\hat{R}$)} \\
    \hat{H} = \hat{O}(\texttt{1;:}) & \quad\text{(first row of $\hat{O}$)} \\
\end{align*}



What about the estimation of $\hat{F}$? Let's try to proceed again similarly to what we have done at the \nameref{step3} of the noise-free case.
Consider for example $\hat{O}$ and define $\hat{O}_1$ as $\hat{O}$ without the last row, and $\hat{O}_2$ as $\hat{O}$ without the first row.

Using the \emph{shift-invariance} property we have that $\hat{O}_1 \hat{F} = \hat{O}_2$, but $\hat{O}_1$ is not a square matrix so it's NOT invertible.
To avoid this problem we can use the approximate \emph{least-squares} solution.


\begin{recall}[Solution of Linear Systems]
    Consider a generic system $Ax = B$ with dimension  $(h\times n) \cdot (n \times 1) = (h \times 1)$. We have 3 different cases:
    \begin{enumerate}
        \item $h < n$. We have less equations than variables: the system is \emph{under determined} and we have infinite solutions.
        \item $h = n$. We have one and only one solution if $A$ is invertible.
        \item $h > n$. We have more equations than variables: the system is \emph{over determined} and it's impossible (no solutions).
    \end{enumerate}
\end{recall}

\begin{remark}[Least-Squares Method]
    In case we have more equations than variable (i.e. $h>n$) we can use an approximate solution using the least-squares method, which for a generic system is as follows:
    \begin{align*}
        Ax &= B \\
           &\Downarrow\\
        A\transpose A x &= A\transpose B \implies \hat{X} = \underbrace{(A\transpose A)^{-1}A\transpose}_{A^+} B
    \end{align*}
    
    $A^+$ is called \emph{pseudo-inverse}, which is ``surrogate inverse'' when $A$ is rectangular.
\end{remark}


Applying the least-square method to $\hat{O}_1 \hat{F} = \hat{O}_2$ we have
\begin{align*}
    \hat{O}_1\hat{F} = \hat{O}_2 \quad \Rightarrow \quad
    \hat{O}_1\transpose\hat{O}_1\hat{F} = \hat{O}_1\transpose\hat{O}_2 \quad \Rightarrow \quad 
    \hat{F} = \left(\hat{O}_1\transpose\hat{O}_1\right)^{-1} \hat{O}_1\transpose\hat{O}_2
\end{align*}


\paragraph{Conclusion} Starting from a noisy \gls{ir} $\{\widetilde{\omega}(1), \widetilde{\omega}(2), \ldots, \widetilde{\omega}(N)\}$ we have estimated a model $\{\hat{F}, \hat{G}, \hat{H}\}$ in a non-parametric and constructive way.
\[ \left\{\hat{H}, \hat{G}, \hat{F} \right\} = \left \{\hat{O}(\texttt{1;:}), \hat{R}(\texttt{:;1}), \left(\hat{O}_1\transpose\hat{O}_1\right)^{-1} \hat{O}_1\transpose\hat{O}_2\right \}\]

\begin{obs}
    This method can be extended also to the case where the input signal is generic (i.e. not an impulse).
\end{obs}

\begin{remark}[Optimality of 4SID]
    The method is \emph{optimal} in the sense that it makes the best possible rank reduction of $\tilde{H}_{qd}$, that is from $q$ to $n$.
    
    However, notice that in general there are infinite ways to make a rank reduction.
\end{remark}

\begin{remark}[Rank Reduction]
Our goal is to obtain the desired rank reduction by discarding the minimum amount of information contained in the original matrix.
SVD makes exactly this: $\tilde{H}_{res,qd}$ is the minimum possible (in the sense of the \emph{Frobenius norm}).
\[
    \left|\tilde{H}_{res,qd}\right|_F = \sqrt{\sum_{ij} \left(\tilde{H}_{res,qd}^{(ij)} \right)^2}
\]
    
\end{remark}

\begin{example}[Rank Reduction]
    \[
        \underbrace{\begin{bmatrix}
            2 & 5 & 3 & 6 & 5 \\
            5 & 3 & 6 & 5 & 7 \\
            3 & 6 & 5 & 7 & 1
        \end{bmatrix}}_{\rank = 3}
        =
        \underbrace{\begin{bmatrix}
            1 & 0 & 0 & 0 & 0 \\
            0 & 1 & 0 & 0 & 0 \\
            0 & 0 & 0 & 0 & 0
        \end{bmatrix}}_{\rank = 2}
        +
        \underbrace{
        \begin{bmatrix}
            1 & 5 & 3 & 6 & 5 \\
            5 & 2 & 6 & 5 & 7 \\
            3 & 6 & 5 & 7 & 1
        \end{bmatrix}}_{\rank = 3}
    \]
    It's not the optimal rank reduction matrix, because it factors out a matrix with lower rank but a lot of information of the original matrix is lost.
\end{example}


\begin{remark}
    4SID is a constructive method that can be implemented in a fully-automatic way, except for these steps:
    \begin{itemize}
        \item $q$ and $d$ selection (not critical)
        \item choice of $n$ (typically supervised by the designer). It can be made automatic using a cross-validation method.
    \end{itemize}
\end{remark}

\begin{remark}
    SVD was an historical turning point in machine learning algorithms because it allows:
    \begin{itemize}
        \item very efficient compression of information.
        \item very efficient separation of \emph{important} information from noise.
        \item order reduction of a model.
    \end{itemize}
    Notice that those are 3 different prospective of the same general problem.
\end{remark}
