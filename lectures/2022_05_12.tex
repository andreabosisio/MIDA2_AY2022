%!TEX root = ../main.tex

\externaldocument{2022_05_09}

\begin{remark}
    The closed-loop behavior is very simple. Visually:

    \begin{figure}[H]
        \centering
        \begin{tikzpicture}[node distance=2cm,auto,>=latex']
            \node[block] at (0,1.5) (c) {$C(z)$};
            \node[sum] at (1,1.5) (s1) {};
            \node[block, right=0.3cm of s1] (b1) {$\frac{1}{B(z) E(z)}$};
            \node[block, right=0.7cm of b1] (b2) {$\frac{z^{-k} B(z)}{A(z)}$};
            \node[sum, right=0.5cm of b2] (s2) {};
            \node[block] at (2,0) (b3) {$\tilde{R}(z)$};
            \node[block, above=0.5cm of s2] (b4) {$\frac{C(z)}{A(z)}$};

            \draw[<-] (c) -- ++(-1,0) node[left] {$y^0(t)$};
            \draw[->] (c) -- (s1) node[pos=0.8] {+};
            \draw[->] (s1) -- (b1);
            \draw[->] (b1) -- (b2) node[above, midway] {$\scriptstyle u(t)$};
            \draw[->] (b2) -- (s2);
            \draw[->] (b3) -| (s1) node[pos=0.9] {-};
            \draw[->] (b4) -- (s2);
            \draw[->] (s2) -- ++(1,0) node[right] {$y(t)$};
            \draw[<-] (b4) -- ++(0,1) node[above] (in_noise_1) {$e(t)$};
            \draw[->] ($(s2) + (0.5,0)$) |- (b3);

            \node[block] at (11,1.5) (z) {$z^{-k}$};
            \node[block] at (12,3) (e) {$E(z)$};
            \node[sum] at (12,1.5) (s3) {};

            \draw[<-] (z) -- ++(-1,0) node[left] (y0_2) {$y^0(t)$};
            \draw[->] (s3) -- ++(1,0) node[right] (out_2) {$y(t)$};
            \draw[->] (z) -- (s3);
            \draw[->] (e) -- (s3);
            \draw[<-] (e) -- ++(0,1) node[above] {$e(t)$};

            \draw[dashed] (-0.7,-0.7) rectangle ($(in_noise_1) + (0.7, 0.5)$);
            \draw[dashed] ($(y0_2) + (0.6, -0.5)$) rectangle ($(out_2) + (-0.6, 3)$);

            \node[right=1.5cm of b4] (equiv) {$\equiv$};
            \node[below=0.001cm of equiv] {(closed-loop)};
        \end{tikzpicture}
    \end{figure}

    In principle the \emph{ideal} desired behavior would be $y(t) = y^0(t) + \cancelto{0}{E(z)}e(t)$, but, since the system has internally $k$-step delay and the reference $y^0(t)$ is not predictable, the \emph{optimal} result is the one found in \ref{subsec:MVC_performance}, that is:

    \[
    	y(t) = y^0(t-k) + E(z) e(t)
    \]

    \textbf{Note} Where are all the poles of the original system?

    MVC \emph{pushes} all the system poles into the non-observable and/or non-controllable parts of the system (it makes internal cancellations). However,
    this is something wanted and therefore it is not a problem since we verified it's internally asymptotically stable.
\end{remark}

\section{Main limitations of MVC}

\begin{itemize}
    \item Can be applied only to minimum-phase systems.
    \item We cannot moderate the control/actuation effort.
    \item We cannot design a specific behavior from $y^0(t)$ to $y(t)$.
\end{itemize}


\subsection{Generalized Minimum Variance Control (GMVC)}\label{subsec:GMVC}

To overcome these limits there is an extension called \emph{Generalized Minimum Variance Control (GMVC)}.
The main difference w.r.t. MVC is the extension of the performance index:

\begin{align*}
    \text{MVC: } J &= E\left[ \left(y(t) - y^0(t)\right)^2 \right] \\
    \text{GMVC: } J &= E\left[ \left(P(z)y(t) - y^0(t) + Q(z)u(t)\right)^2 \right]
\end{align*}
where $P(z)$ and $Q(z)$ are \gls{tf}s designed by engineers; in particular:
\begin{itemize}
    \item $P(z)$ is the \emph{reference model}
    \item $Q(z) u(z)$ is used to penalize the usage of control action, which has to be moderated since it's expensive in terms of energy, power, aging, ...
\end{itemize}

\begin{obs}
	The performance index $J$ of the MVC is just a special case of the GMVC's one, where $P(z) = 1$ and $Q(z) = 0$.
\end{obs}

\begin{remark}[Another important difference w.r.t. MVC]
	GMVC can be applied also to non-minimum phase systems.
\end{remark}

\begin{remark}[Reference model $P(t)$]
	In a generic feedback control system
    \begin{figure}[H]
        \centering
        \begin{tikzpicture}[node distance=2cm,auto,>=latex']
            \node[block] at (0,1.5) (c) {C};
            \node[block] at (2,1.5) (s) {S};

            \draw[->] (c) -- (s) node[pos=0.5] {$u(t)$};
            \draw[->] (c) -- (s);
            \draw[<-] (c) -- ++(-1,0) node[left] {$y^0(t)$};
            \draw[->] (s) -- ++(2,0) node[right] {$y(t)$};
            \draw[->] (3,1.5) -- (3,0.5) -| (c);
        \end{tikzpicture}
    \end{figure}

    the typical goal is to obtain the best possible tracking $y(t) = y^0(t)$. However, perfect tracking may not be the best solution: sometimes the best solution is to track a \emph{reference model} or \emph{reference behavior} from $y^0(t)$ to $y(t)$:
    \[
    	y(t) = P(z)y^0(t)
    \]
\end{remark}

\begin{example}[Cruise control in a car]
	Cruise control is a typical example of \emph{reference model}'s usage.
    \begin{figure}[H]
        \centering
        \begin{tikzpicture}[node distance=2cm,auto,>=latex']
            \node[block] at (0,1.5) (c) {Cruise Control};
            \node[block] at (4,1.5) (s) {Car};

            \draw[->] (c) -- (s) node[pos=0.5] {};
            \draw[->] (c) -- (s) node[above, midway, align=center] {engine torque\\$u(t)$};
            \draw[<-] (c) -- ++(-2,0) node[left, align=center] {desired\\speed\\$v^0(t)$};
            \draw[->] (s) -- ++(2,0) node[right, align=center] {speed\\$v(t)$};
            \draw[->] (3,1.5) -- (3,0.5) -| (c);
        \end{tikzpicture}
    \end{figure}

    \begin{figure}[H]
        \centering
        \begin{tikzpicture}[
                node distance=2cm,auto,>=latex',
                declare function={
                    f(\x) =  (\x < 0.5) * 1 +
                            (\x >= 0.5) * (\x < 2) * 2 +
                            (\x >= 2) * (\x < 4) * 3 +
                            (\x >= 4) * 1.5;
                    f2(\x) = f(\x) - (
                                (f(\x) - f(\x-0*0.1)) * 0.8 +
                                (f(\x) - f(\x-1*0.1)) * 0.4 +
                                (f(\x) - f(\x-2*0.1)) * 0.2 +
                                (f(\x) - f(\x-3*0.1)) * 0.1
                            ) / 1.5;
                    f3(\x) = f(\x) - (
                                (f(\x) - f(\x-0*0.1)) * 0.5 +
                                (f(\x) - f(\x-1*0.1)) * 0.4 +
                                (f(\x) - f(\x-2*0.1)) * 0.4 +
                                (f(\x) - f(\x-3*0.1)) * 0.3 +
                                (f(\x) - f(\x-4*0.1)) * 0.3 +
                                (f(\x) - f(\x-5*0.1)) * 0.2 +
                                (f(\x) - f(\x-6*0.1)) * 0.2 +
                                (f(\x) - f(\x-7*0.1)) * 0.2 +
                                (f(\x) - f(\x-8*0.1)) * 0.1 +
                                (f(\x) - f(\x-9*0.1)) * 0.1 +
                                (f(\x) - f(\x-10*0.1))* 0.05
                            ) / 2.75;
                }
            ]
            \draw[->] (0,0) -- (0,3.5) node[above] {$v(t)$};
            \draw[->] (0,0) -- (5,0) node[below] {$t$};
            \draw[domain=0:5,variable=\x,blue,samples=100] plot ({\x},{f(\x)}) node[right, blue] {$v^0(t)$};
            \draw[domain=0:5,variable=\x,red,smooth,samples=50] plot ({\x},{f2(\x)});
            \draw[domain=0:5,variable=\x,green,smooth,samples=50] plot ({\x},{f3(\x)});
            \node[red, right]   at (3,1) {$\scriptstyle P(z) \approx 1$};
            \node[green, right] at (3,0.7) {$\scriptstyle P(z)$ \footnotesize to smooth the behavior};
        \end{tikzpicture}
    \end{figure}

    In this scenario it is intuitive that a very aggressive tracking (i.e. $P(z) \approx 1$) is unwanted since it is uncomfortable and dangerous. A smoother reference behavior is preferred (i.e. $P(z)$ is a low-pass filter).
\end{example}

\begin{appendices}

\chapter{Discretization of Analog Dynamical systems}\label{appendix:discr}
Discretization allows us to do modeling, SW-sensing and control in a digital context. 

\begin{figure}[H]
    \centering
    \begin{tikzpicture}[node distance=2cm,auto,>=latex']
        \node[block,ellipse,align=center] at (0,0) (controller) {Control\\algorithm\\(discrete time)};
        \node[block,align=center] at (6,0) (system) {Physical\\system\\(analog)};
        \node[block] at (3.5,0) (DAC) {D/A};
        \node[block] at (3.5,-2) (ADC) {A/D};

        \draw[->] (controller) -- (DAC);
        \draw[->] (DAC) -- (system) node[pos=0.5] {$\scriptstyle u(t)$};
        \draw[->] (system) -- ++(2,0) node[above] {$y(t)$};
        \draw[->] (7.3,0) |- (ADC);
        \draw[->] (ADC) -| (controller);
    \end{tikzpicture}
\end{figure}

\subsection*{Analog to Digital Converter (A/D)}

\begin{figure}[H]
    \centering
    \begin{tikzpicture}[node distance=2cm,auto,>=latex',declare function={
            f(\x) = sin(180*\x/3.14)+1.2;
        }]
        \draw[->] (0,0) -- (5,0) node[below] {$t$};
        \draw[->] (0,0) -- (0,3) node[left] {$y(t)$};

        \draw[domain=0:5,variable=\x,smooth] plot ({\x},{f(\x)});
        \foreach \x in {0,...,9} {
            \draw[dotted] (\x/2,0) -- (\x/2,3);
            \draw[fill, blue] (\x/2,{round(2*f(\x/2))/2}) circle (0.03);
        }
        \foreach \y in {0,...,5} {
            \draw[dotted] (0,\y/2) -- (5,\y/2);
        }

        \draw[<->] (0.5,-0.1) -- (1,-0.1) node[pos=0.5,below] {$\scriptstyle\Delta T$};
        \node[right] at (1.05,-0.33) {\footnotesize (sampling time)};
        \draw[<->] (-0.1,1.5) -- (-0.1,2);
        \node[left,align=right] at (-0.1,1.75) {\footnotesize amplitude\\\footnotesize discretization \\\footnotesize step};
    \end{tikzpicture}
\end{figure}

\begin{description}
    \item[Time discretization] $\Delta T$ is the sampling time
    \item[Amplitude discretization step] Number of levels used for discretization, e.g. \emph{10-bit discretization} uses $2^{10}$ levels of amplitude
\end{description}

An high quality A/D converter:
\begin{itemize}
    \item small $\Delta T$
    \item high number of levels (e.g. 16-bits)
\end{itemize}


\subsection*{Digital to Analog Converter (D/A)}

\begin{figure}[H]
    \centering
    \begin{tikzpicture}[node distance=2cm,auto,>=latex',declare function={
            f(\x) = sin(180*\x/3.14)+1.2;
        }]
        \draw[->] (0,0) -- (5,0) node[below] {$t$};
        \draw[->] (0,0) -- (0,3) node[left] {$u(t)$};

        \foreach \x in {0,...,9} {
            \draw[dotted] (\x/2,0) -- (\x/2,3);
            \draw[fill] (\x/2,{round(2*f(\x/2))/2}) circle (0.03);
        }
        \foreach \x in {0,...,8} {
            \draw[blue] (\x/2,{round(2*f(\x/2))/2}) -- (\x/2+0.5,{round(2*f(\x/2))/2});
            \draw[blue] (\x/2+0.5,{round(2*f(\x/2))/2}) -- (\x/2+0.5,{round(2*f(\x/2+0.5))/2});
        }
        \foreach \y in {0,...,5} {
            \draw[dotted] (0,\y/2) -- (5,\y/2);
        }
    \end{tikzpicture}
\end{figure}

If $\Delta T$ is sufficiently small, the step-wise analog signal is very similar to a smooth analog signal (the steps become "invisible").

\begin{figure}[H]
    \centering
    \begin{tikzpicture}[node distance=2cm,auto,>=latex',declare function={
            f(\x) = sin(180*\x/3.14)+1.2;
        }]
        \draw[->] (0,0) -- (5,0) node[below] {$t$};
        \draw[->] (0,0) -- (0,3) node[left] {$u(t)$};

        \foreach \x in {0,...,49} {
            \draw[fill] (\x/10,{round(20*f(\x/10))/20}) circle (0.01);
        }
        \foreach \x in {0,...,49} {
            \draw[blue] (\x/10,{round(20*f(\x/10))/20}) -- (\x/10+0.1,{round(20*f(\x/10))/20});
            \draw[blue] (\x/10+0.1,{round(20*f(\x/10))/20}) -- (\x/10+0.1,{round(20*f(\x/10+0.1))/20});
        }
    \end{tikzpicture}
\end{figure}

\subsection*{Model of the Digital Perspective}

\begin{figure}[H]
    \centering
    \begin{tikzpicture}[node distance=2cm,auto,>=latex']
        \node[block,align=center,minimum height=2cm] at (0,0) (cont) {Digital\\controller};
        \node[block,align=center,minimum height=2cm] at (6,0) (sys) {Analog\\system};
        \node[block] at (3,0.5) (DAC) {D/A};
        \node[block] at (3,-0.5) (ADC) {A/D};

        \draw[->] (cont.east|-DAC) -- (DAC) node[pos=0.2] (u_t) {$\scriptstyle u(t)$};
        \draw[<-] (cont.east|-ADC) -- (ADC) node[pos=0.2] {$\scriptstyle y(t)$};
        \draw[->] (DAC) -- (DAC-|sys.west);
        \draw[<-] (ADC) -- (ADC-|sys.west);

        \draw[dashed,red] (1.7,-1.5) rectangle (7.5,1.5) 
        	node[above=0.3cm of u_t] {\color{red} $\Sys$:};
        \node[below] at (4.6,-1.5) {\footnotesize System from a digital perspective};
    \end{tikzpicture}
\end{figure}

\textbf{Note} Both $u(t)$ and $y(t)$ are digital signal.

We can obtain a discrete-time model from $u(t)$ to $y(t)$ for the system \textcolor{red}{$\Sys$}:

\begin{figure}[H]
	\centering 
	\begin{tikzpicture}[node distance=2cm,auto,>=latex']
		\node[block, red] (s) {$\Sys$};

		\draw[<-] (s) --++ (-1,0) node[left] {$u(t)$}; 
		\draw[->] (s) --++ (1,0) node[right] {$y(t)$}; 
	\end{tikzpicture}
\end{figure}

and we can either:

\begin{itemize}
    \item make \gls{bb} system identification starting from measured data: directly estimate a discrete-time model
    \item starting from a physical \gls{wb} model (continuous time) that we need to discretize 
\end{itemize}

Let's analyze the latter method.

\subsubsection*{Discretization of a physical \gls{wb} model}
We introduce two methods for discretize a physical \gls{wb} model.

\paragraph{1) \acrlong{ss} Transformation}

The most used approach is the \emph{\acrlong{ss} Transformation}.

\[
    \Sys: \begin{cases}
        \dot{x} = Ax + Bu \\
        y = Cx + (Du) \\
        \qquad \text{($A$, $B$, $C$, $D$ cont. time)}
    \end{cases}
    \quad
    \xRightarrow[\Delta T]{\text{discretization}}
    \quad
    \Sys: \begin{cases}
        x(t+1) = Fx(t) + Gu(t) \\
        y(t) = Hu(t) + (Du(t)) \\
        \qquad \text{($F$, $G$, $H$, $D$ discrete time)}
    \end{cases}
\]

Discretization is done via those transformation formulas:
\begin{align*}
    F &= e^{A\Delta T} \\
    G &= \int_0^{\Delta T} e^{A\delta}B\, d\delta \\
    H &= C \\
    D &= D
\end{align*}

\begin{recall}[$s$-domain] Let's recall the transformation from the $s$-domain to the $z$-domain.
	
	\paragraph{Poles}
    How the poles of the continuous time system are transformed?

    It can be proved that the eigenvalues (poles) follow the \emph{sampling transformation rule}.
    \[
        z = e^{s\Delta T} \qquad \lambda_F = e^{\lambda_A \Delta T}
    \]

    \begin{figure}[H]
        \centering
        \begin{tikzpicture}[node distance=2cm,auto,>=latex']
            \draw[->] (-2,0) -- (2,0) node[above] {$\operatorname{Re}$};
            \draw[->] (0,-2) -- (0,2) node[left] (im1) {$\operatorname{Im}$};

            \draw[->] (4,0) -- (8,0) node[above] {$\operatorname{Re}$};
            \draw[->] (6,-2) -- (6,2) node[left] (im2) {$\operatorname{Im}$};

            \draw[dashed, red, line width=0.4mm] (0,-2) -- (0,1.9);
            \fill[green, opacity=0.1] (-2,-2) rectangle (0,2);

            \draw[dashed, red, line width=0.4mm] (6,0) circle (1cm);
            \fill[green, opacity=0.1] (6,0) circle (1cm);

            \node[align=center] at (0,2.8) {\textbf{$s$-domain}\\(cont. time)};
            \node[align=center] at (6,2.8) {\textbf{$z$-domain}\\(discrete time)};

            \draw[fill,blue] (0,0) circle (0.05);
            \draw[fill,blue] (7,0) circle (0.05);

            \draw[->] (2,1) -- (4,1) node[pos=0.5] {$z=e^{s\Delta T}$};

            \fill[green, opacity=0.1] ($(im2) + (3.2,0)$) rectangle ++(2,1) 
            	node[pos=0.5, above=0.005cm, align=center, black, opacity=1] {as. stable\\part};
        \end{tikzpicture}
    \end{figure}

    \paragraph{Zeroes}
    How the zeroes of the continuous time system are transformed?

    Unfortunately there is no simple rule like the poles. We can only say:
    \[
        G(s) = \frac{\text{polynomial in $s$ with $h$ zeros}}{\text{polynomial in $s$ with $k$ poles}} \qquad \text{if $G(s)$ is strictly proper:  (i.e. $k > h$) }
    \]
    \[
        G(z) = \frac{\text{polynomial in $z$ with $k-1$ zeros}}{\text{polynomial in $z$ with $k$ poles}} \qquad \text{$G(z)$ with relative degree 1}
    \]

    We have new $k-h-1$ zeros that are generated by the discretization.
    They are called \emph{hidden zeros}.

    Unfortunately these hidden zeros are frequently outside the unit circle, which means that $G(z)$ is not minimum phase even if $G(s)$ is minimum phase.

    Therefore, we need for instance GMVC to design the control system.
\end{recall}

\paragraph{2) Time-derivative discretization}

Another simple discretization technique frequently used is the discretization of time-derivative $\dot{x}$.

\begin{align*}
    \text{\textbf{Eulero backward}} &\qquad \dot{x} \approx \frac{x(t)-x(t-1)}{\Delta T} = \frac{x(t)-z^{-1}x(t)}{\Delta T} = \frac{z-1}{z\Delta T} x(t) \\
    \text{\textbf{Eulero forward}} &\qquad \dot{x} \approx \frac{x(t+1)-x(t)}{\Delta T} = \frac{zx(t)-x(t)}{\Delta T} = \frac{z-1}{\Delta T} x(t)
\end{align*}

General formula
\[
    \dot{x}(t) \approx \left[ \frac{z-1}{\Delta T} \frac{1}{\alpha z + (1-\alpha)} \right]x(t) \qquad \text{with } 0 \le \alpha \le 1
\]
We consider three special cases:
\begin{itemize}
    \item if $\alpha = 0$ it's Eulero Forward
    \item if $\alpha = 1$ it's Eulero Backward
    \item if $\alpha = \frac{1}{2}$ it's Tustin method (mostly used)
\end{itemize}

\subsection*{Choice of the sampling time}

The critical choice is $\Delta T$ (sampling time).

\paragraph{Simple idea}
The general intuitive rule is: the smaller $\Delta T$, the better.

\begin{recall}[Sampling and Nyquist frequency] If $\Delta T$ is sampling time, then
    \paragraph{Notation} 
    \[
        f_S = \frac{1}{\Delta T} [\text{Hz}] \qquad \omega_S = \frac{2\pi}{\Delta T} [\text{rad/s}] \qquad f_N = \frac{1}{2} f_S [\text{Hz}] \qquad \omega_N = \frac{1}{2} \omega_S [\text{rad/s}]
    \]
    where $f_S$ is the \emph{sampling frequency} and $f_N$ is the \emph{Nyquist frequency}.

    \paragraph{Spectrum of a discretized signal}
    For a discretized signal its spectrum is limited and it ranges over $[0, \omega_N]$; as a result, the fitting between the spectrum of the discretized signal and the one of the original signal is very close at low frequencies and becomes more inaccurate as we approach $\omega_N$.
\end{recall}

If $\Delta T$ is large, $f_S$ is small (and accordingly,  $f_N$ too). 

\begin{figure}[H]
    \centering
    \begin{tikzpicture}[node distance=2cm,auto,>=latex',declare function={
        a(\x) = (\x-0.5)^2+2;
        b(\x) = -(\x-1)^2+2.125;
        c(\x) = 2*(\x-2.5)^2+0.625;
        d(\x) = -(\x-3)^2+0.794;
        f(\x) = e^(-\x+2.9517);
        z(\x) = (\x < 0.5) * 2 +
                (0.5 <= \x) * (\x < 0.75) * a(\x) +
                (0.75 <= \x) * (\x < 2) * b(\x) +
                (2 <= \x) * (\x < 2.64) * c(\x) +
                (2.64 <= \x) * (\x < 3.35) * d(\x) +
                (3.35 <= \x) * f(\x);
        z2(\x) = z(\x) * (\x < 1) + z(\x) * (\x >= 1) * (\x < 2.5) * (1-(\x-1)^2/1.5/4);
    }]
        \draw[->] (0,0) -- (5.5,0) node[below] {$\omega$};
        \draw[->] (0,0) -- (0,3) node[left] {$|G|$};
        \draw[samples=50,domain=0:5,variable=\x,smooth] plot ({\x},{z(\x)});
        \draw[samples=50,red,domain=0:2.5,variable=\x,smooth] plot ({\x},{z2(\x)});
        \draw[dotted] (2.5,0.625) -- (2.5,-0.1) node[below] {$\scriptstyle \omega_N$};
        \draw (3.5,0.1) -- (3.5,-0.1) node[below] {$\scriptstyle \omega_S$};
        \draw (3.5,0) edge[bend right=30,->, dashed] (2.5,0);
        \node at (3,0.32) {$\scriptstyle \times \frac{1}{2}$};

        \node[right] at (3,2) {\footnotesize Original signal (cont. time)};
        \node[right,red] at (3,1.7) {\footnotesize Discretized signal (discr. time)};
    \end{tikzpicture}
\end{figure}

Otherwise, if $\Delta T$ is smaller, $\omega_S$ (and $\omega_N$ accordingly) larger.

\begin{figure}[H]
    \centering
    \begin{tikzpicture}[node distance=2cm,auto,>=latex',declare function={
        a(\x) = (\x-0.5)^2+2;
        b(\x) = -(\x-1)^2+2.125;
        c(\x) = 2*(\x-2.5)^2+0.625;
        d(\x) = -(\x-3)^2+0.794;
        f(\x) = e^(-\x+2.9517);
        z(\x) = (\x < 0.5) * 2 +
                (0.5 <= \x) * (\x < 0.75) * a(\x) +
                (0.75 <= \x) * (\x < 2) * b(\x) +
                (2 <= \x) * (\x < 2.64) * c(\x) +
                (2.64 <= \x) * (\x < 3.35) * d(\x) +
                (3.35 <= \x) * f(\x);
        z2(\x) = z(\x) * (\x < 2.5) + z(\x) * (\x >= 2.5) * (\x < 4) * (1-(\x-2.5)^2/1.5/4);
    }]
        \draw[->] (0,0) -- (5.5,0) node[below] {$\omega$};
        \draw[->] (0,0) -- (0,3) node[left] {$|G|$};
        \draw[samples=50,domain=0:5,variable=\x,smooth] plot ({\x},{z(\x)});
        \draw[samples=50,red,domain=0:4,variable=\x,smooth] plot ({\x},{z2(\x)});

        \draw[dotted] (4,0.35) -- (4,-0.1) node[below] {$\scriptstyle \omega_N$};
        \draw (4.8,0.1) -- (4.8,-0.1) node[below] {$\scriptstyle \omega_S$};
        \draw (4.8,0) edge[bend right=30,->, dashed] (4,0);

        \node[right] at (3,2) {\footnotesize Original signal (cont. time)};
        \node[right,red] at (3,1.7) {\footnotesize Discretized signal (discr. time)};
    \end{tikzpicture}
\end{figure}

As we can see from the two graphs, in both case we have approximations of the original spectrum but in the latter case this approximation is valid over a larger \emph{bandwidth}.

\paragraph{Drawbacks} Hidden problems of a too-small $\Delta T$:
\begin{itemize}
    \item sampling devices (A/D and D/A) cost
    \item computational cost (e.g. update an algorithm every $1 \mu s$ is much heavier than every $1 ms$)
    \item cost of memory (if data storing is needed)
    \item numerical precision cost 
\end{itemize}

The latter one is the most critical one and it is really an hidden problem.

\subparagraph{Numerical precision cost}
Let's make a numerical example to understand it.

Consider a continous time \gls{tf} with two asymptotically stable poles (eigenvalues) in $s=-3$ and $s=-2$. These can be mapped in the $z$-domain using the formula $\lambda_F=e^{\lambda_A\Delta T}$. 
If $\Delta T$ is very small (tends to zero), we squeeze all the poles very closed to $(1,0)$.

Graphically:

\begin{figure}[H]
    \centering
    \begin{tikzpicture}[node distance=2cm,auto,>=latex']
        \draw[->] (-2,0) -- (2,0) node[above] {$\operatorname{Re}$};
        \draw[->] (0,-2) -- (0,2) node[left] {$\operatorname{Im}$};

        \draw[->] (4,0) -- (8,0) node[above] {$\operatorname{Re}$};
        \draw[->] (6,-2) -- (6,2) node[left] {$\operatorname{Im}$};

        \draw (6,0) circle (1cm);

        \node[cross] at (-0.5,0) {};
        \node[cross] at (-1,0) {};

        \draw (-0.5,0) edge[dotted, bend left=30,->] (6.99,0);
        \draw (-1,0) edge[dotted, bend left=30,->] (6.98,0);
        \node[right] at (7,-0.2) {$\scriptstyle 0.99995$};
        \node[right] at (7,-0.4) {$\scriptstyle 0.99996$};

        \node at (0,2.5) {\textbf{$s$-domain}};
        \node at (6,2.5) {\textbf{$z$-domain}};

        \draw[->] (2,2) -- (4,2) node[pos=0.5] {$\lambda_F=e^{\lambda_A\Delta T}$};
    \end{tikzpicture}
\end{figure}

Therefore, we need very high numerical precision (use a lot of digits) to avoid instability.

Rule of thumb of control engineers: $f_S$ is between 10 and 20 times the system bandwidth we are interested in.

\begin{figure}[H]
    \centering
    \begin{tikzpicture}[node distance=2cm,auto,>=latex',declare function={
        a(\x) = (\x-0.5)^2+2;
        b(\x) = -(\x-1)^2+2.125;
        c(\x) = 2*(\x-2.5)^2+0.625;
        d(\x) = -(\x-3)^2+0.794;
        f(\x) = e^(-\x+2.9517);
        z(\x) = (\x < 0.5) * 2 +
                (0.5 <= \x) * (\x < 0.75) * a(\x) +
                (0.75 <= \x) * (\x < 2) * b(\x) +
                (2 <= \x) * (\x < 2.64) * c(\x) +
                (2.64 <= \x) * (\x < 3.35) * d(\x) +
                (3.35 <= \x) * f(\x);
    }]
        \draw[->] (0,0) -- (5.5,0) node[below] {$f$};
        \draw[->] (0,0) -- (0,3) node[left] {$|G|$};
        \draw[samples=50,domain=0:5,variable=\x,smooth] plot ({\x},{z(\x)});

        \draw[dotted] (1.5,1.875) -- (1.5,0) node[below] {};
        \node[below, align=center] at (0.75,0) {\footnotesize bandwidth\\ \footnotesize of interest};
        \draw (4,0.1) -- (4,-0.1) node[below] {$\scriptstyle f_N$};
        \draw (1.5,0) edge[bend left=20,->, dashed] (4,0);
        \node at (2.75,0.45) {$\scriptstyle \times 10$};
    \end{tikzpicture}
\end{figure}

Usually, the bandwidth of interest is the one of the closed-loop control system.


\paragraph{Aliasing problem}
Another problem is managing the \emph{aliasing} problem, which is a big and critical problem in the A/D step. 

\begin{theorem}[Nyquist–Shannon sampling theorem]\label{th:shannon}
    The maximum frequency content of a signal $f_{MAX}$ to be sampled should be such that $f_{MAX} \le f_N$.
\end{theorem}


When we want measure a signal $x$ we capture also the measurement noise; indeed, what we really obtain is $\tilde{x}(t) = x(t) + e(t)$, where $e$ is the noise. 
Hence, the spectrum will also be composed by the spectrum of the original signal $x$ and the spectrum of the noise.

\begin{figure}[H]
    \centering
    \begin{tikzpicture}[node distance=2cm,auto,>=latex',declare function={
        f(\x) = 5/(\x^2+2) - 0.6 ;
        g(\x) = 10/(5*\x+3) - 1.6 ;
        h(\x) = 0.2 * (1 - (\x - 1.1)^3/2.5) ;
    }]
        \draw[->] (0,0) -- (5.5,0) node[below] {$f$};
        \draw[->] (0,0) -- (0,2.5) node[left] {$\Gamma_{\tilde{x}}$};
        \draw[blue, domain=0:3,samples=40,variable=\x,smooth] plot ({\x},{f(\x)}) 
            node[above, blue] {$\tilde{x}$};
        \draw[green,domain=0:1,samples=40,variable=\x,smooth] plot ({\x},{g(\x)}) 
            node[above, green] {$x$};
        \draw[domain=0:3,samples=40,variable=\x,smooth] plot ({\x},{h(\x)})
            node[right] {$e$};

        \draw (2.5,0.1) -- ++(0,-0.1) node[below] {$f_{MAX}$};
    \end{tikzpicture}
\end{figure}

Therefore, if we want to sample the measured signal, and for example, $f_{MAX} = 2 \text{KHz}$, then we need $f_N \ge 2 \text{KHz} \implies f_S \ge 4 \text{KHz}$. On the other hand, we know that the bandwidth of the original signal $x(t)$ will be much smaller due to the presence of the noise. 
For example, suppose that the frequency content of $x(t)$ is contained in the range $[0, 0.5]$ KHz: we can therefore sample with an A/D that samples at $f_S = 1$ KHz.

\subparagraph{Analog solution} The classical way to deal with aliasing is to use anti-alias analog filters. 

\begin{recall}[Anti-Aliasing Filter (AAF)]
    An \emph{anti-aliasing filter (AAF)} is a filter used before a signal sampler to restrict the bandwidth of a signal to satisfy the Nyquist–Shannon sampling theorem (\ref{th:shannon}) over the band of interest.
    
    In practice, it is an analog low-pass filter with a cut frequency $f_{cut} \le \frac{1}{2} f_S = f_N$. 
\end{recall}

\begin{figure}[H]
    \centering
    \begin{tikzpicture}[node distance=2cm,auto,>=latex']
        \node[block,align=center] at (0,0) (sys) {\textbf{analog}\\anti-alias\\low-pass filter\\$f_{cut} = 0.5$KHz};
        \node[block] at (3,0) (ADC) {A/D};

        \draw[->] (-2.5,0) -- (sys) node[pos=0.5] {$\tilde{x}(t)$};
        \draw[->] (sys) -- (ADC) node[pos=0.5] {$\tilde{x}'(t)$};
        \draw[->] (ADC) -- ++(1.5,0) node[pos=0.5] {$\tilde{x}''(t)$};
    \end{tikzpicture}
\end{figure}

In this case, since $f_S = 1$ KHz we need a low-pass filter that cut everything above $f_N = \frac{1}{2} f_S = 0.5$ KHz.

\begin{figure}[H]
    \centering
    \begin{tikzpicture}[node distance=2cm,auto,>=latex',declare function={
        f(\x) = 5/(\x^2+2) - 0.6 ;
        g(\x) = 10/(5*\x+3) - 1.6 ;
    }]
        \draw[->] (0,0) -- (4,0) node[below] {$f$};
        \draw[->] (0,0) -- (0,3) node[left] {$\Gamma_{\tilde{x}}$};
        \draw[blue, domain=0:3,samples=40,variable=\x,smooth] plot ({\x},{f(\x)}) 
            node[above, blue] {$\tilde{x}$};
        \draw[green,domain=0:1,samples=40,variable=\x,smooth] plot ({\x},{g(\x)}) 
            node[above, green] {$x$};
        \draw (2.5,0.05) -- (2.5,-0.05) node[below] {$\scriptstyle 2KHz$};

        \draw[->] (5,0) -- (9,0) node[below] {$f$};
        \draw[->] (5,0) -- (5,3) node[left] {$\Gamma_{\tilde{x}'}$};
        \draw[blue, domain=0:1,samples=13,variable=\x,smooth] plot ({\x + 5},{f(\x)}) 
            node[above, blue] {$\tilde{x}$};
        \draw[green,domain=0:1,samples=13,variable=\x,smooth] plot ({\x + 5},{g(\x)}) 
            node[below, green] {$x$};
        \draw (6,0.05) -- (6,-0.05) node[below] {$\scriptstyle 0.5KHz$};
        \draw[dotted] (6,0) -- (6,2.1);

        % \node[align=center] at (9,2) {Nyquist generates no\\alias at 1KHz};
    \end{tikzpicture}
\end{figure}

In this way the signal $\tilde{x}$ is ready to be sampled: aliasing is avoided ($f'_{MAX} = f_{cut} = f_N$); furthermore, all the frequency components in the bandwidth of the original signal $x(t)$ are preserved and the ones introduced by the noise are cut-off.

\subparagraph{Full digital solution}  Today the most used solution is the full digital one in which there is no more the need of an analog anti-alias filter.

\begin{figure}[H]
    \centering
    \begin{tikzpicture}[node distance=2cm,auto,>=latex',declare function={
        f(\x) = 2.7 * (1 - \x/3) + rand^2/3;
    }]
        \node[block, align=center] at (0,5) (ADC) {A/D \\ $f_S=4$ KHz};
        \node[block, align=center] at (3,5) (filter) {\textbf{digital}\\low-pass\\filter \\$f_{cut} = 0.5$ KHz};
        \node[block, align=center] at (6.3,5) (sampl) {under-sampling\\$(1:4)$\\{$f'_S=1$ KHz}};

        \draw[->] (-2,5) -- (ADC) node[pos=0.5] {$\tilde{x}(t)$};
        \draw[->] (ADC) -- (filter);
        \draw[->] (filter) -- (sampl);
        \draw[->] (sampl) -- ++(2,0);

        % \draw [decorate,decoration={brace,amplitude=10pt},yshift=-0.1cm] (7,4.5) -- (-0.5,4.5) node [black,midway,yshift=-0.4cm] {\footnotesize fully digital};
    \end{tikzpicture}
\end{figure}

With this approach we consider the bandwidth of the noisy signal $\tilde x(t)$ ($f_{MAX} = 2$ KHz) and we sample it at $f_S = 2 f_{MAX} = 4$ KHz. Then we cut-off all the frequency components introduced by the noise with a digital low-pass filter ($f_{MAX} \rightarrow f'_{MAX} = 0.5$ KHz). Last, since now the bandwidth of the signal is $f'_{MAX} = 0.5$ KHz, we can re-sample the signal at $f'_S = 2 f'_{MAX} = 1$ KHz: this is \emph{under-sampling (1:4)} (because $f'_S = \frac{1}{4} f_S$) which in practice means to take one sample every out of four.


\end{appendices}