%!TEX root = ../main.tex

\chapter{Chapter 4: BB SW-Sensing}
Content of chapter 4 (short)

\chapter{Chapter 5: Gray-Box System Identification}
Content of chapter 5 (short)

\chapter{Minimum Variance Control}
Minimum Variance Control (MVC) is about design and analysis of feedback systems, it is not about system identification nor software sensing.

Why we dedicate a chapter on \emph{control}?
\begin{itemize}
    \item Control design is the main motivation to system identification and software sensing.
    \item MVC is based on \emph{mathematics} of system identification and software sensing (prediction theory).
    \item MVC can be considered as a general tool of stochastic optimization of feedback systems.
\end{itemize}

\section{MVC System}

\paragraph{Setup of the problem} Consider a generic \gls{armax} model

\begin{recall}[\gls{armax} system]
	\[
	    y(t) = \frac{B(z)}{A(z)}u(t-k) + \frac{C(z)}{A(z)}e(t) \qquad e(t) \sim WN(0, \lambda^2)
	\]
	\begin{align*}
	    B(z) &= b_0 + b_1z^{-1} + \dots + b_pz^{-p} \\
	    A(z) &= 1   + a_1z^{-1} + \dots + a_mz^{-m} \\
	    C(z) &= 1   + c_1z^{-1} + \dots + c_nz^{-n}
	\end{align*}

	\textbf{Note}: The input is $u(t)$ and the output is $y(t)$. The noise $e(t)$ is a non-measurable input that models the uncertainty of the system.  
\end{recall}

Assumptions:
\begin{itemize}
    \item $\frac{C(z)}{A(z)}$ is in \emph{canonical form}
    \item $b_0\ne 0$ (this implies that $k \ge 0$ is the actual delay)
    \item $\frac{B(z)}{A(z)}$ is \emph{minimum phase}
\end{itemize}

\begin{remark}
    $\frac{B(z)}{A(z)}$ is said to be \emph{minimum phase} if all the roots of $B(z)$ are strictly inside the unit circle.

    What does it means in practice? 

    \begin{figure}[H]
        \centering
        \begin{tikzpicture}[node distance=2cm,auto,>=latex']
            \draw[->] (0,4) -- (0,5) node[left] {$u(t)$};
            \draw[->] (0,4) -- (5,4) node[below] {$t$};
            \draw[->] (0,0) -- (0,3) node[left] {$y(t)$};
            \draw[->] (0,0) -- (5,0) node[below] {$t$};

            \draw[line width=0.4mm] (0,4) -- (1,4) -- (1,4.5) -- (5,4.5);
            \draw[dotted] (1,0) -- (1,4);
            \draw[dotted] (0,2.5) -- (5,2.5) node[right] {\footnotesize steady state};

            \draw[domain=1:4.5,smooth,variable=\x,green,samples=70] plot ({\x},{2.5-2.5*(1-(\x-1)/3.5)^5});
            \draw[domain=1:4.5,smooth,variable=\x,green,samples=70] plot ({\x},{2.5-2.5*(1-(\x-1)/3.5)^5+sin(\x*180)/\x});
            \draw[domain=1:4.5,smooth,variable=\x,red,samples=70] plot ({\x},{2.5-2.5*(1-(\x-1)/3.5)^5+3*sin(\x*180)/(\x^2)});

            \node[red,right] at (1.5,-0.5) {\footnotesize non- minimum phase};
            \node[green,right] at (2,1.5) {\footnotesize minimum phase};
        \end{tikzpicture}
    \end{figure}

    At the very beginning the response of a \textcolor{red}{non-minimum phase} system goes to the opposite direction w.r.t. the final value. Intuitively it's very difficult to control non-minimum phase systems: you can take the wrong decision if you react immediately.

    Also for human it's difficult, for example \emph{steer to roll} dynamics in a bicycle: if you want to steer left, you must first steer a little to the right and then turn left.

    Design of controller for non-minimum phase is difficult and requires special design techniques (no MVC but \emph{generalized MVC (GMVC)}).
\end{remark}

Some additional (small) technical assumptions:
\begin{itemize}
    \item $y^0(t)$ and $e(t)$ are not correlated (usually fulfilled).
    \item We assume (worst case) that $y^0(t)$ is known only up to time $t$ (present time): we have no preview of the future desired $y^0(t)$ (i.e. $y^0(t)$ is totally unpredictable or $\hat{y^0}(t+k|t) = y^0(t)$).
\end{itemize}

\paragraph{Goal of the problem} 
The problem we wish to solve is optimal tracking of the desired behavior of output (which is the classical goal of control systems):
\begin{figure}[H]
    \centering
    \begin{tikzpicture}[node distance=2cm,auto,>=latex']
        \node[block,ellipse,align=center] at (-1,0) (cont) {controller\\algorithm};
        \node[block] at (2.5,0) (ba) {$\frac{B(z)}{A(z)}z^{-k}$};
        \node[block] at (4,1.5) (ca) {$\frac{C(z)}{A(z)}$};
        \node[sum] at (4,0) (sum) {};

        \draw[dotted] (0.8,-1) rectangle (5,3.2) node[right] {system to be controlled};
        \draw[->] (cont) -- (ba) node[pos=0.7] {$u(t)$};
        \draw[->] (ba) -- (sum);
        \draw[->] (ca) -- (sum);
        \draw[<-] (cont) -- ++(-2,0) node[left] {$y^0(t)$};
        \draw[<-] (ca) -- ++(0,1) node[above] {$e(t)$};
        \draw[->] (sum) -- ++(2,0) node[right] {$y(t)$};
        \draw[->] (5.5,0) -- (5.5,-1.5) -- (-1,-1.5) -- (cont);
    \end{tikzpicture}
\end{figure}
where $y^0(t)$ os the desired output value, called \emph{reference}. 

In a more formal way MVC is an optimization control problem that tries to find $u(t)$ that minimize this performance index:
\[
    J = E\left[ (y(t) - y^0(t))^2 \right]
\]

where $J$ is the variance of the tracking error: that's why it's called Minimum Variance Control.

\paragraph{Solution of the problem}
The main trick is to split $y(t)$ into $\hat{y}(t|t-k)$, the \emph{predictor}, and $\epsilon(t)$, the \emph{prediction error}.

Now we can write $J$ as 
\begin{align*}
	J =& \E \left[ \left( \hat{y}(t|t-k) + \epsilon(t) - y^0(t) \right)^2 \right] \\
	  =& \E \left[ \left( (\hat{y}(t|t-k) - y^0(t)) + \epsilon(t) \right)^2 \right] \\
	  =& \E \left[ \left( (\hat{y}(t|t-k) - y^0(t) \right)^2 \right] + \E [\epsilon(t)^2 ] + 2 \E \left[ \epsilon(t) \left( \hat{y}(t|t-k) + y^0(t) \right) \right] 
\end{align*}

where we note that

\begin{itemize}
	\item $\E [\epsilon(t)^2 ]$ doesn't depend on $u(t)$: it's not subject to minimization, since it's just a constant number.
	\item $\E [ \epsilon(t) y^0(t) ] = 0$, since $e(t) = f(e(t), e(t-1), \dots) $ and, by assumption, $y^0 \perp e$.
	\item $\E [ \epsilon(t) \hat{y}(t|t-k) ] = 0$, since by construction $\epsilon(t) \perp \hat{y}(t|t-k)$ (otherwise $\hat{y}(t|t-k)$ wouldn't be the optimal predictor).
\end{itemize}

Therefore, minimizing $J$ is equivalent to minimizing  $\E \left[ \left( (\hat{y}(t|t-k) - y^0(t) \right)^2 \right]$, which is minimized if

\begin{align}\label{eq:min_J_condition}
	\hat{y}(t|t-k) = y^0(t)
\end{align}

Now we have to compute $\hat{y}(t|t-k)$ and impose $\hat{y}(t+k|t) = y^0(t+k)$ (obtained by shifting the equation \ref{eq:min_J_condition} using the $z$ operator).

However, by assumption, at time $t$ we don't know $y^0(t+k)$: the best we can do is to replace it with the last available value of $y^0$, that is $y^0(t)$.

Therefore, the condition \ref{eq:min_J_condition} becomes 

\begin{align}\label{eq:new_min_J_condition}
	\hat{y}(t+k|t) = y^0(t)
\end{align}

\begin{recall}[$k$-step predictor of an \gls{armax} system]

	\[
		\hat{y}(t+k|t) = \frac{B(z) E(z)}{C(z)} u(t) + \frac{\tilde{R}(z)}{C(z)} y(t)
	\] 

	where $E(z)$ is the $result$ and $R(z) = \tilde{R}(z) z^{-k}$ is the $residual$ of the $k$-step long division between $C(z)$ and $A(z)$.

\end{recall}

By plugging in the $k$-step predictor of an \gls{armax} system formula in the equation \ref{eq:new_min_J_condition} we obtain

\[
	\frac{B(z) E(z)}{C(z)} u(t) + \frac{\tilde{R}(z)}{C(z)} y(t) = y^0(t)
\]

and by making $u(t)$ explicit we obtain the \emph{General Formula of MVC}:

\begin{align}\label{eq:MVC_general-formula}
	u(t) = \frac{1}{B(z)E(z)} \left( C(z) y^0(t) - \tilde{R}(z) y(t) \right)
\end{align}

The block scheme representation of the MVC system is

\begin{figure}[H]
    \centering
    \begin{tikzpicture}[node distance=2cm,auto,>=latex']
        \node[sum] at (0,0) (sum) {};
        \node[block, left=1cm of sum] (c1) {$C$}; 
        \node[block] at (1.5,0) (b1) {$\frac{1}{B(z) E(z)}$};
        \node[block] at (5,0) (b2) {$\frac{B(z)}{A(z)} z^{-k}$};
        \node[block] at (3,-1.5) (b3) {$\tilde{R}(z)$};
        \node[block] at (7,1.5) (b4) {$\frac{C(z)}{A(z)}$};
        \node[sum] at (7,0) (sum2) {};

        \draw[<-] (c1) --++ (-1,0) {} node[left] (in) {$y^0(t)$};
        \draw[->] (sum) -- (b1);
        \draw[->] (b1) -- (b2) node[pos=0.5] {$u(t)$};
        \draw[->] (b3) -| (sum) node[pos=0.9] {$-$};
        \draw[->] (c1) -- (sum)  node[above, near end] {$+$};
        \draw[->] (sum2) -- ++(1.5,0) node[right] {$y(t)$};
        \draw[->] (b2) -- (sum2);
        \draw[->] (b4) -- (sum2);
        \draw[<-] (b4) -- ++(0,1) node[above] (noise_in) {$e(t)$};
        \draw[->] (8,0) |- (b3);

        \draw[dashed, blue] ($(in) + (0.7, -2.5)$) rectangle ($(b2) + (-1.2, 1)$) node[pos=0, below] {\text{controller}};
        \draw[dashed] ($(b2) + (-1, -0.6)$) rectangle ($(noise_in) + (0.7, 0.5)$) node[above] {\text{system}};
    \end{tikzpicture}

    \caption{MVC System}
    \label{fig:MVC_sys}
\end{figure}

\section{Analysis of the MVC System}

\subsection{Stability of the MVC System}

\begin{recall}
    For stability let's recall a result of \emph{negative feedback system}:
    \begin{figure}[H]
        \centering
        \begin{tikzpicture}[node distance=2cm,auto,>=latex']
            \node[block] at (0,1.5) (f1) {$F_1(t)$};
            \node[block] at (0,0.5) (f2) {$F_2(t)$};
            \node[sum] at (-1.5,1.5) (sum) {};

            \draw[<-] (sum) -- ++(-1,0) node[pos=0.2, above] {$+$};
            \draw[->] (f2) -| (sum) node[pos=0.9] {$-$};
            \draw[->] (sum) -- (f1);
            \draw[->] (f1) -- ++(2,0);
            \draw[->] (1.5,1.5) |- (f2);
        \end{tikzpicture}
    \end{figure}

    To check the closed-loop stability:
    \begin{itemize}
        \item compute the \emph{loop-function} $L(z) = F_1(z) F_2(z)$ (\textbf{remember}: do not simplify!)
        \item build the \emph{characteristic polynomial} $\chi(z) = L_N(z) + L_D(z)$ (sum of the numerator and the denominator of $L(z)$)
        \item find the roots of $\chi(z)$: closed loop system is asymptotically stable if and only if all the roots of $\chi(z)$ are strictly inside the unit circle
    \end{itemize}
\end{recall}

Check the stability of the MVC system represented in figure \ref{fig:MVC_sys}

\[
    L(z) = \frac{1}{B(z)E(z)}\cdot \frac{z^{-k}B(z)}{A(z)}\cdot\tilde{R}(z)
\]
\begin{align*}
    \chi(z) &= z^{-k}B(z)\tilde{R}(z) + B(z)E(z)A(z) \\
    &= B(z) \underbrace{\left( z^{-k}\tilde{R}(z)+E(z)A(z) \right)}_{C(z)}  \\
    &= B(z)C(z)
\end{align*}

where we used the following result of the long division between $C(z)$ and $A(z)$:

\[ \frac{C(z)}{A(z)} = E(z) + \frac{z^{-k} \tilde{R}(z)}{A(z)} \]

The MVC system is always guaranteed asymptotically stable, since the roots of $\chi(z)$ are the roots of $B(z)$ and $C(z)$ and:
\begin{itemize}
    \item all roots of $B(z)$ are strictly inside the unit circle (thanks to minimum phase assumption on $\frac{B(z)}{A(z)}$)
    \item all roots of $C(z)$ are ?? inside the unit circle (thanks to the assumption of canonical representation of $\frac{C(z)}{A(z)}$)
\end{itemize}

{\large roots of $C(z)$ are (NOT strictly) inside the unit circle!}

\subsection{Performance of the MVC System}

The system can be rewritten as follows considering the two inputs: $y^0(t)$ and $e(t)$ (non-measurable input)
\[ 
	y(t) = F_{y^0y}(z) y^0(t) + F_{ey}(z) e(t)
\]

\begin{recall}
    The transfer function from the input to the output of a \emph{negative feedback system} can be computed with:
    \begin{figure}[H]
        \centering
        \begin{tikzpicture}[node distance=2cm,auto,>=latex']
            \node[block] at (0,1.5) (f1) {$F_1(t)$};
            \node[block] at (0,0.5) (f2) {$F_2(t)$};
            \node[sum] at (-1.5,1.5) (sum) {};

            \draw[<-] (sum) -- ++(-1,0) node[left] {$u(t)$} node[pos=0.2, above] {$+$} ;
            \draw[->] (f2) -| (sum) node[pos=0.9] {$-$};
            \draw[->] (sum) -- (f1);
            \draw[->] (f1) -- ++(2,0) node[right] {$y(t)$};
            \draw[->] (1.5,1.5) |- (f2);
        \end{tikzpicture}
    \end{figure}

    \[ 
    	F(z) = \frac{F_1(z)}{1 + F_1(z) F_2(z)} \qquad y(t) = F(z) u(t)
    \]
    where we recall that $F_1(z) F_2(z)$ is the loop-function $L(z)$
 	and $F_1(z)$ is the direct line from the input to the output.
\end{recall}

Therefore, the \gls{tf} from $y^0(t)$ to $y(t)$ is
\begin{align*}
	F_{y^0y}(z) =& \frac{C(z) \cdot \frac{1}{B(z) E(z)} \cdot \frac{z^{-k} B(z)}{A(z)}}{1 + \underbrace{\frac{1}{B(z)E(z)}\cdot \frac{z^{-k}B(z)}{A(z)}\cdot\tilde{R}(z)}_{L(z)}} \\
	=& \dots \\
	=& z^{-k}
\end{align*}
where we remind that $L(z)$ has been already computed for the stability check. 

Similarly, the \gls{tf} from $e(t)$ to $y(t)$ is 
\begin{align*}
	F_{ey}(z) =& \frac{\frac{C(z)}{A(z)}}{1 + \underbrace{\frac{1}{B(z)E(z)}\cdot \frac{z^{-k}B(z)}{A(z)}\cdot\tilde{R}(z)}_{L(z)}} \\
	=& \dots \\
	=& E(z)
\end{align*}

Thus we can say that

\begin{align*}
	y(t) =& F_{y^0y}(z) y^0(t) + F_{ey}(z) e(t) \\
		 =& z^{-k} y^0(t) + E(z) e(t)
\end{align*}
 
which is the very simple closed-loop relationship between input and output in a MVC system.

% \begin{remark}
%     There are 2 sub-classes of control problems:
%     \begin{itemize}
%         \item When $y^0(t)$ is constant or step-wise (regulation problem)
%         \item When $y^0(t)$ is varying (tracking problem)
%     \end{itemize}

%     \begin{figure}[H]
%         \centering
%         \begin{minipage}[t]{0.48\textwidth}
%             \centering
%             \begin{tikzpicture}[
%                 node distance=2cm,auto,>=latex',
%                 declare function={
%                     f(\x) =  (\x < 0.5) * 1 +
%                              (\x >= 0.5) * (\x < 2) * 2 +
%                              (\x >= 2) * (\x < 3) * 3 +
%                              (\x >= 3) * (\x < 4) * 1.5 +
%                              (\x >= 4) * 1;
%                     f2(\x) = (f(\x-0.5) + (f(\x) - f(\x-0.5)) / 720 +
%                              f(\x-0.4) + (f(\x) - f(\x-0.4)) / 120 +
%                              f(\x-0.3) + (f(\x) - f(\x-0.3)) / 24 +
%                              f(\x-0.2) + (f(\x) - f(\x-0.2)) / 6 +
%                              f(\x-0.1) + (f(\x) - f(\x-0.1)) / 2) / 5 +
%                              rand/8;
%                 }
%             ]
%                 \draw[->] (0,0) -- (0,3) node[above] {$y(t)$};
%                 \draw[->] (0,0) -- (5,0) node[below] {$t$};
%                 \draw[domain=0:5,variable=\x,blue,samples=100] plot ({\x},{f(\x)});
%                 \draw[domain=0:5,variable=\x,red,smooth,samples=100] plot ({\x},{f2(\x)});
%             \end{tikzpicture}
%             \caption*{Regulation problem}
%         \end{minipage}
%         \begin{minipage}[t]{0.48\textwidth}
%             \centering
%             \begin{tikzpicture}[
%                 node distance=2cm,auto,>=latex',
%                 declare function={
%                     f(\x) =  (sin(\x*180)/2+sin(\x*270)/2)+1.5;
%                     f2(\x) = (f(\x-0.3) + (f(\x) - f(\x-0.3)) / 24 +
%                              f(\x-0.2) + (f(\x) - f(\x-0.2)) / 6 +
%                              f(\x-0.1) + (f(\x) - f(\x-0.1)) / 2) / 3 +
%                              rand/8;
%                 }
%             ]
%                 \draw[->] (0,0) -- (0,3) node[above] {$y(t)$};
%                 \draw[->] (0,0) -- (5,0) node[below] {$t$};
%                 \draw[domain=0:5,variable=\x,blue,samples=70] plot ({\x},{f(\x)});
%                 \draw[domain=0:5,variable=\x,red,smooth,samples=50] plot ({\x},{f2(\x)});
%             \end{tikzpicture}
%             \caption*{Tracking problem}
%         \end{minipage}
%     \end{figure}
% \end{remark}

% Bottom-up way of presenting M.V.C.

% \paragraph{Simplified problem \#1}
% \[
%     S: y(t) = ay(t-1) + b_0u(t-1) + b_1u(t-2) \qquad y(t) = \frac{b_0+b_1z^{-1}}{1-az^{-1}}u(t-1)
% \]

% We assume that $y^0(t)=\overline{y}^0$ (regulation problem) and the system is noise-free.
% \begin{itemize}
%     \item $b_0\ne 0$
%     \item Root of numerator must be inside the unit circle
% \end{itemize}

% To design the minimum variance controller we must minimize the performance index:
% \[
%     J = E\left[ (y(t) - y^0(t))^2 \right]
% \]
% There is no noise so we can remove the expected value
% \begin{align*}
%     J &= \left( y(t) - y^0(t) \right)^2 = \left( y(t) - \overline{y}^0 \right)^2 = \left( ay(t-1)+b_0u(t-1)+b_1u(t-2) - \overline{y}^0 \right)^2 = \\
%     &= \left( ay(t) + b_0u(t) + b_1u(t-1)-\overline{y}^0 \right)^2 \\
%     \frac{\partial J}{\partial u(t)} &= 2\left( ay(t)+b_0u(t)+b_1u(t-1)-\overline{y}^0 \right)\left(b_0\right)
% \end{align*}

% Why the derivative is just $b_0$? We are at present time $t$ and at time $t$ the control algorithm must take a decision on the value of $u(t)$.
% At time $t$, $y(t)$, $y(t-1)$, \dots, $u(t-1)$, $u(t-2)$, \dots{} are no longer variables but numbers.

% \[
%     \frac{\partial J}{\partial u(t)} = 0 \qquad ay(t)+b_0u(t)+b_1u(t-1)-\overline{y}^0 = 0 \\
% \]
% \[
%     u(t) = \left( \overline{y}^0  - ay(t)\right)\frac{1}{b_0+b_1z^{-1}}
% \]
% \begin{figure}[H]
%     \centering
%     \begin{tikzpicture}[node distance=2cm,auto,>=latex']
%         \node[sum] at (0,0) (sum) {};
%         \node[block] at (1.5,0) (b1) {$\frac{1}{b_0+b_1z^{-1}}$};
%         \node[block] at (5,0) (b2) {$z^{-1}\frac{b_0+b_1z^{-1}}{1-az^{-1}}$};
%         \node[block] at (3,-1.5) (b3) {$a$};

%         \draw[->] (sum) -- (b1);
%         \draw[->] (b1) -- (b2) node[pos=0.5] {$u(t)$};
%         \draw[->] (b3) -| (sum) node[pos=0.9] {-};
%         \draw[<-] (sum) -- ++(-1.5,0) node[left] {$\overline{y}^0$} node[pos=0.2] {+};
%         \draw[->] (b2) -- ++(2.5,0) node[right] {$y(t)$};
%         \draw[->] (6.5,0) |- (b3);
%     \end{tikzpicture}
% \end{figure}

% \paragraph{Simplified problem \#2}
% \[
%     S: y(t) = ay(t-1) + b_0u(t-1) + b_1u(t-2) + e(t) \qquad e(t) \sim WN(0, \lambda^2)
% \]

% The reference variable is $y^0(t)$ (tracking problem).

% The performance index is
% \[
%     J = E\left[ (y(t) - y^0(t))^2 \right]
% \]

% The fundamental trick to solve this problem is to re-write $y(t)$ as:
% \[
%     y(t) = \hat{y}(t|t-1) + \epsilon(t)
% \]

% Since $k=1$ we know that $\epsilon(t) = e(t)$, so $y(t) = \hat{y}(t|t-1)+e(t)$.
% \begin{align*}
%     J &= E\left[ \left(\hat{y}(t|t-1)+e(t) - y^0(t)\right)^2 \right] \\
%     &= E\left[   \left((\hat{y}(t|t-1)-y^0(t)) +e(t)\right)^2 \right] \\
%     &= E\left[ \left(\hat{y}(t|t-1)-y^0(t)\right)^2 \right] + E\left[e(t)^2\right] + \cancel{2E\left[e(t)\left( \hat{y}(t|t-1)-y^0(t) \right)\right]} \\
% \end{align*}

% Notice that
% \[
%     \argmin_{u(t)} \left\{ E\left[ \left(\hat{y}(t|t-1)-y^0(t)\right)^2 \right] + \lambda^2 \right\} = \argmin_{u(t)} \left\{ E\left[ \left(\hat{y}(t|t-1)-y^0(t)\right)^2 \right] \right\}
% \]
% The best result is when $\hat{y}(t|t-1)=y^0(t)$, we can force this relationship.

% Now we must compute the 1-step predictor of the system:
% \[
%     S: y(t) = \frac{b_0+b_1z^{-1}}{1-az^{-1}}u(t-1) + \frac{1}{1-az^{-1}}e(t)
% \]
% Note that this is an $ARMAX(1,0,1+1)=ARX(1,2)$.
% \[
%     k=1 \qquad B(z) = b_0+b_1z^{-1} \qquad A(z)=1-az^{-1} \qquad C(z) = 1
% \]

% General solution for 1-step prediction of ARMAX:
% \[
%     \hat{y}(t|t-1) = \frac{B(z)}{C(z)}u(t-1) + \frac{C(z)-A(z)}{C(z)}y(t)
% \]
% If we apply this formula we obtain:
% \[
%     \hat{y}(t|t-1) = \frac{b_0+b_1z^{-1}}{1}u(t-1) + \frac{1-1+az^{-1}}{1}y(t) = (b_0+b_1z^{-1})u(t-1)+ay(t-1)
% \]

% Now we can impose that $\hat{y}(t|t-1)=y^0(t)$
% \[
%     b_0u(t) + b_1u(t-1) + ay(t) = y^0(t+1) \qquad u(t) = \left( y^0(t+1) - ay(t) \right)\frac{1}{b_0+b_1z^{-1}}
% \]
% But we don't have $y^0(t+1)$, so we use $y^0(t)$.
% \[
%     u(t) = \left( y^0(t) - ay(t) \right)\frac{1}{b_0+b_1z^{-1}}
% \]
% \begin{figure}[H]
%     \centering
%     \begin{tikzpicture}[node distance=2cm,auto,>=latex']
%         \node[sum] at (0,0) (sum) {};
%         \node[block] at (1.5,0) (b1) {$\frac{1}{b_0+b_1z^{-1}}$};
%         \node[block] at (5,0) (b2) {$z^{-1}\frac{b_0+b_1z^{-1}}{1-az^{-1}}$};
%         \node[block] at (3,-1.5) (b3) {$a$};
%         \node[block] at (7,1.5) (b4) {$\frac{1}{1-az^{-1}}$};
%         \node[sum] at (7,0) (sum2) {};

%         \draw[->] (sum) -- (b1);
%         \draw[->] (b1) -- (b2) node[pos=0.5] {$u(t)$};
%         \draw[->] (b3) -| (sum) node[pos=0.9] {-};
%         \draw[<-] (sum) -- ++(-1.5,0) node[left] {$y^0(t)$} node[pos=0.2] {+};
%         \draw[->] (sum2) -- ++(1.5,0) node[right] {$y(t)$};
%         \draw[->] (b2) -- (sum2);
%         \draw[->] (b4) -- (sum2);
%         \draw[<-] (b4) -- ++(0,1) node[above] {$e(t)$};
%         \draw[->] (8,0) |- (b3);
%     \end{tikzpicture}
% \end{figure}

