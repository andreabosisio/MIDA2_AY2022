%!TEX root = ../main.tex

% --- Posticipated chapters --- 
\chapter{Software-sensing with Black-Box Methods}
%!TEX root = ../main.tex

\externaldocument{2022_05_02}

In chapter 3 we have seen classical technology of software-sensing based on \acrlong{kf}:
\begin{figure}[H]
    \centering
    \begin{tikzpicture}[node distance=2.5cm,auto,>=latex']
        \node[left] at (0,4) (u) {$u(t)$};
        \node[block] at (2,4) (sys) {$\Sc$};
        \node[block] at (2,2.3) (k) {$\bar{K}$};
        \node[sum] at (4,2.3) (sum) {};
        \node[right] at (6,4) (y) {$y(t)$};
        \node[block, align=center] at (2,1) (model) {model of \\ $\Sc$};
        \node[above] at (2,5) (dist) {disturbances};

        \draw[->] (dist) -- (sys);
        \draw[->] (u) -- (sys);
        \draw[->] (sys) -- (y);
        \draw[<-,red,line width=0.4mm] (sum) -- (4,4) node[pos=0.2] {$+$};
        \draw[->] (sum) -- (k) node[midway, above] {$e(t)$};
        \draw[->,red,line width=0.4mm] (0.5,4) |- (model);
        \draw[->] (k) -- (model);
        \draw[->] (model) -| (sum) 
        	node[pos=0.85] {$-$}
        	node[right, pos=0.7] {$\hat{y}(t|t-1)$};
        \draw[->,red,line width=0.4mm,transform canvas={yshift=-0.2cm}] (model) -- (6,1) node[right] {$\hat{x}(t|t)$};
        \draw[dashed, blue] (0,0) rectangle (5,3) node[right] {$\mathcal{KF}$};
    \end{tikzpicture}
\end{figure}

\textbf{Note} If the \gls{kf} is the asymptotic one (i.e. $K(t) = \bar{K}$), it is a MIMO LTI system. 

Main features of this approach:
\begin{itemize}
    \item A (\acrlong{wb}/physical) model is needed.
    \item No need (in principle) of a training dataset including measurements of the state to be estimated.
    \item It is a feedback estimation algorithm (feedback correction of the model using estimated output error).
    \item Constructive method (non-parametric, no optimization involved).
    \item Can be used (in principle) to estimate states which are impossible to be measured (also at prototyping/training/design stage).
\end{itemize}

Are there other classes of software-sensing techniques? Yes, black-box approaches with \emph{learning}/\emph{training} from data (system identification).
 
In this chapter we see them focusing on the architecture (we do not need new algorithms, just use something we have already studied). We will re-cast the SW-sensing problem into a system identification problem. 

\section{Linear Time Invariant Systems}\label{sec:BB-SW-LTI}

To find the relationship between $u(t) \rightarrow \hat{x}(t|t)$ and $y(t) \rightarrow \hat{x}(t|t)$ we can use a dataset.
Indeed, if we have a dataset
\begin{align*}
    \left\{ u(1), u(2), \ldots, u(N) \right\} \\
    \left\{ y(1), y(2), \ldots, y(N) \right\} \\
    \left\{ x(1), x(2), \ldots, x(N) \right\}
\end{align*}

we can estimate the relationship (i.e. \gls{tf}s) between the inputs ($u(t)$ and $y(t)$) and the output ($x(t)$) without modelling the system and adopting a \gls{bb} approach instead.

\textbf{Note} This is a supervised training approach, thus, only for the training phase, we need measurements of the state to be estimated (using physical sensor that, in \emph{production phase}, will be replaced by the trained SW-sensor).


\paragraph{Model selection} We focus on this family of models:
\[
	\hat{x}(t|t) = S_{ux}(z, \theta) u(t-1) + S_{yx}(z,\theta)y(t)
\]
where we have at least 1-step delay on $u(t)$. The block scheme of this model is: 
\begin{figure}[H]
    \centering
    \begin{tikzpicture}[node distance=2cm,auto,>=latex']
    	\node[block] (z) {$z^{-1}$};
        \node[block, right of=z] (ux) {$S_{ux}(z, \theta)$};
        \node[block, below of=ux] (yx) {$S_{yx}(z, \theta)$};
        \node[left,left of=z] (u) {$u(t)$};
        \node[left, below of=u] (y) {$y(t)$};
        \node[sum,right of=ux,xshift=1cm,yshift=-1cm] (sum) {};
        \node[right, right of=sum] (x) {$\hat{x}(t|t)$};

        \draw[->] (u) -- (z);
        \draw[->] (z) -- (ux);
        \draw[->] (y) -- (yx);
        \draw[->] (ux) -| (sum);
        \draw[->] (yx) -| (sum);
        \draw[->] (sum) -- (x);
    \end{tikzpicture}
\end{figure}

\paragraph{Performance index}
We define the usual performance index as the \emph{sample variance of the estimation error}: 
\[
    J_N(\theta) = \frac{1}{N}\sum_{t=1}^N \left( x(t) - (S_{ux}(z, \theta) u(t-1) + S_{yx}(z,\theta)y(t)) \right)^2
\]

\paragraph{Optimization}
\[
    \hat{\theta}_N = \argmin_\theta J_N(\theta)
\]

We obtain the \acrlong{bb} software sensor $\hat{x}(t|t)$ as: 
\[
	\hat{x}(t|t) = S_{ux}(z, \hat{\theta}_N) u(t-1) + S_{yx}(z,\hat{\theta}_N)y(t)
\]


\textbf{Note} Once the SW-sensor has been designed (trained), we no longer need samples of $x(t)$.

\textbf{Note} The above method is a classic \gls{bb} parametric approach (using \gls{tf}s) but the same can also be done using 4-SID algorithm.


\section{Non-linear Systems}

In case the system is non-linear, we can use the same idea used in the LTI case (see \ref{sec:BB-SW-LTI}), where we replace the asymptotic \gls{kf} with a non-linear SW-sensor (like EKF, described in \ref{subsec:KF_non-lin_ext})

\begin{figure}[H]
    \centering
    \begin{tikzpicture}[node distance=2cm,auto,>=latex']
        \node[block,dashed border,align=center] at (2,4) (n1) {$\Sc_{NL}$};
        \node[block,double border,align=center] at (2,2) (n2) {$f_{NL}$};
        \node[block,dashed border,align=center] at (2,0) (n3) {model of\\ $\Sc_{NL}$};
        \node[sum] at (4,2) (sum) {};
        \node[left] at (0,4) (u) {$u(t)$};
        \node[left] at (6,4) (y) {$y(t)$};

        \draw[dashed, blue] (0,-1) rectangle (5,3) 
        	node[right, align=center] {non-linear\\SW-sensor};

        \draw[->] (u) -- (n1);
        \draw[->] (0.5,4) |- (n3);
        \draw[->] (n2) -- (n3);
        \draw[->] (sum) -- (n2);
        \draw[->] (n1) -| (sum) node[pos=0.9] {+};
        \draw[<-] (sum) |- (n3) node[pos=0.1] {-};
        \draw[->] (n1) -- (y);
        \draw[->,transform canvas={yshift=-0.3cm}] (n3) -- ++(4,0) node[right] {$\hat{x}(t|t)$};
    \end{tikzpicture}
\end{figure}

\begin{rem}[Block scheme notation for non-linear system] 
	Notation used from now on to represent non-linear systems:
	\begin{figure}[H]
	    \centering
	    \begin{tikzpicture}[node distance=2cm,auto,>=latex']
	        \node[block,double border,align=center] (n1) {non-lin.\\\textbf{static}\\system};
	        \node[block,dashed border,align=center, right, right=3cm of n1](n2) {non-lin.\\\textbf{dynamic}\\system};
	    \end{tikzpicture}
	\end{figure}
\end{rem}

% \begin{rem}
%     In \gls{kf} the E.K.F. extension uses the trick of a time-varying linear gain $K(t)$ but the obvious choice is a non-linear gain (static nonlinear function).
% \end{rem}

The content of the box (non-linear SW-sensor) is:

\begin{figure}[H]
    \centering
    \begin{tikzpicture}[node distance=2cm,auto,>=latex']
        \node[block,dashed border,align=center] at (0,0) (n) {non-linear\\dynamic\\time invariant\\system};
        \draw[<-,transform canvas={yshift=0.3cm}] (n) -- ++(-2,0) node[left] {$u(t)$};
        \draw[<-,transform canvas={yshift=-0.3cm}] (n) -- ++(-2,0) node[left] {$y(t)$};
        \draw[->] (n) -- ++(2,0) node[right] {$\hat{x}(t|t)$};
    \end{tikzpicture}
\end{figure}

The problem is again the \gls{bb} identification of a non-linear dynamic system, starting from a measured training dataset.

There are 4 (3+1) different architectures to design the non-linear SW-sensor.

\paragraph{Architecture \#1} Use a \emph{Dynamical Recurrent Neural Network (Dynamical RNN)} in which we update \emph{static neurons} into \emph{dynamic neurons}.

	\begin{figure}[H]
	    \centering
	    \begin{tikzpicture}[node distance=2cm,auto,>=latex']
	        \node[block,dashed border,align=center] at (0,0) (n) {recurrent\\neural network};
	        \draw[<-,transform canvas={yshift=0.3cm}] (n) -- ++(-2,0) node[left] {$u(t)$};
	        \draw[<-,transform canvas={yshift=-0.3cm}] (n) -- ++(-2,0) node[left] {$y(t)$};
	        \draw[->] (n) -- ++(2,0) node[right] {$\hat{x}(t|t)$};
	    \end{tikzpicture}
	\end{figure}

If we zoom into a single neuron: 

\begin{figure}[H]
    \centering
    \begin{minipage}[t]{0.48\textwidth}
        \centering
        \begin{tikzpicture}[node distance=2.5cm,auto,>=latex']
            \node[block] at (0,4) (a1) {$a_1$};
            \node[block] at (0,3) (a2) {$a_2$};
            \node at (0,2) {$\vdots$};
            \node[block] at (0,1) (ah) {$a_h$};
            \node[sum] at (2,2) (sum) {};
            \node[block,ellipse,align=center] at (3.5,2) (nlf) {$f_{NL}$};

            \draw[<-] (a1) -- ++(-1,0);
            \draw[<-] (a2) -- ++(-1,0);
            \draw[<-] (ah) -- ++(-1,0);

            \draw[->] (a1) -- (sum);
            \draw[->] (a2) -- (sum);
            \draw[->] (ah) -- (sum);
            \draw[<-] (sum) -- ++(0,0.5) node[above] {$b$};
            \draw[->] (sum) -- (nlf);
            \draw[->] (nlf) -- ++(1.5,0);
        \end{tikzpicture}
        \caption*{Static neuron (non-linear static system)}
    \end{minipage}
    \begin{minipage}[t]{0.48\textwidth}
        \centering
        \begin{tikzpicture}[node distance=1.5cm,auto,>=latex']
            \node[block] at (0,4) (a1) {$a_1$};
            \node[block] at (0,3) (a2) {$a_2$};
            \node at (0,2) {$\vdots$};
            \node[block] at (0,1) (ah) {$a_h$};
            \node[sum] at (2,2) (sum) {};
            \node[block,ellipse,align=center] at (3.5,2) (nlf) {$f_{NL}$};
            \node[block, below of=nlf] (z) {$z^{-1}$};
            \node[block, below of=sum] (c) {$c$};

            \draw[<-] (a1) -- ++(-1,0);
            \draw[<-] (a2) -- ++(-1,0);
            \draw[<-] (ah) -- ++(-1,0);

            \draw[->] (a1) -- (sum);
            \draw[->] (a2) -- (sum);
            \draw[->] (ah) -- (sum);
            \draw[<-] (sum) -- ++(0,0.5) node[above] {$b$};
            \draw[->] (sum) -- (nlf);
            \draw[->] (nlf) -- ++(1.5,0);

            \draw[->] (4.5,2) |- (z);
            \draw[->] (z) -- (c);
            \draw[->] (c) -- (sum);
        \end{tikzpicture}
        \caption*{Dynamic neuron (non-linear dynamic system)}
    \end{minipage}
\end{figure}
where $f_{NL}$ is a non-linear function (e.g. sigmoid function).

Using an RNN with dynamic neurons is the most general approach but it is also practically seldom used due to stability issues and convergence of training.

\paragraph{Architecture \#2} Split the SW-sensor into a static non-linear system and a dynamic linear system (namely, a non-recursive FIR scheme)

\begin{figure}[H]
    \centering
    \begin{tikzpicture}[node distance=2cm,auto,>=latex']
        \node[block] at (1.5,0) (yn) {$z^{-1}$};
        \node[block] at (1.5,1.5) (y2) {$z^{-1}$};
        \node[block] at (1.5,2.5) (y1) {$z^{-1}$};

        \node[block] at (1.5,4) (un) {$z^{-1}$};
        \node[block] at (1.5,5.5) (u2) {$z^{-1}$};
        \node[block] at (1.5,6.5) (u1) {$z^{-1}$};

        \node[left] at (0,6.5) (u) {$u(t)$};
        \node[left] at (0,3.2) (y) {$y(t)$};

        \node[block,minimum height=7cm,double border,align=center] at (5.25,3.25) (sys) {non-linear\\static\\parametric\\function\\$f_{NL}(\cdot; \theta)$ \vspace{10pt} \\ (e.g. static\\neural\\network)};

        \draw[->] (u) -- (u1);
        \draw[->] (u1) -- (u2);
        \draw[->,dotted] (u2) -- (un);
        \draw[->] (u1) -- (u1-|sys.west) node[pos=0.5] {$u(t-1)$};
        \draw[->] (u2) -- (u2-|sys.west) node[pos=0.5] {$u(t-2)$};
        \draw[->] (un) -- (un-|sys.west) node[pos=0.5] {$u(t-n_u)$};

        \draw[->] (y) -- (y-|sys.west);
        \draw[->] (1.5,3.2) -- (y1);
        \draw[->] (y1) -- (y2);
        \draw[->,dotted] (y2) -- (yn);
        \draw[->] (y1) -- (y1-|sys.west) node[pos=0.5] {$y(t-1)$};
        \draw[->] (y2) -- (y2-|sys.west) node[pos=0.5] {$y(t-2)$};
        \draw[->] (yn) -- (yn-|sys.west) node[pos=0.5] {$y(t-n_y)$};

        \draw[->] (sys) -- ++(2,0) node[right] {$\hat{x}(t|t)$};

        \draw[decoration={brace}, decorate] (3.5,-0.7) node {} -- (0,-0.7);
        \node[align=center,below] at (1.75,-0.9) {linear dynamic\\system};

        \draw[decoration={brace}, decorate] (7,-0.7) node {} -- (3.6,-0.7);
        \node[align=center,below] at (5.25,-0.9) {non-linear static\\system to be\\estimated};
    \end{tikzpicture}
\end{figure}

\begin{rem}[Pros]
	Some pros about this architecture:
	\begin{itemize}
		\item Training (supervised) done only to the non-linear static part of the SW-sensor (much simpler than the estimation of an RNN).
		\item Stability is guaranteed by construction since it is a static FIR architecture.
	\end{itemize}	
\end{rem}

\begin{rem}[Cons]
    In case of a MIMO system with
    \begin{align*}
        m \text{ inputs: } u(t) = \begin{bmatrix}
            u_1(t) \\
            \vdots \\
            u_m(t)
        \end{bmatrix} \quad p \text{ outputs: } y(t) = \begin{bmatrix}
            y_1(t) \\
            \vdots \\
            y_p(t)
        \end{bmatrix} \quad n \text{ states: } x(t) = \begin{bmatrix}
            x_1(t) \\
            \vdots \\
            x_n(t)
        \end{bmatrix}
    \end{align*}

    the estimation problem is the search of the optimal parameter vector $\theta$ for the function
    \[
        f(\cdot; \theta): \RR^{m\times n_u + p \times (n_y + 1)} \rightarrow \RR^n
    \]    
    Therefore, the I/O size of this non-linear parametric function can be very large.
\end{rem}

\paragraph{Architecture \#3} Like architecture \#2, we split the SW-sensor into a static non-linear system and a linear dynamic system but, this time, with a recursive IIR scheme.

\begin{figure}[H]
    \centering
    \begin{tikzpicture}[node distance=2cm,auto,>=latex']
        \node[block] at (1.5,0) (yn) {$z^{-1}$};
        \node[block] at (1.5,1.5) (y2) {$z^{-1}$};
        \node[block] at (1.5,2.5) (y1) {$z^{-1}$};

        \node[block] at (1.5,4) (un) {$z^{-1}$};
        \node[block] at (1.5,5.5) (u2) {$z^{-1}$};
        \node[block] at (1.5,6.5) (u1) {$z^{-1}$};

        \node[block] at (1.5,8) (xn) {$z^{-1}$};
        \node[block] at (1.5,9.5) (x2) {$z^{-1}$};
        \node[block] at (1.5,10.5) (x1) {$z^{-1}$};

        \node[left] at (0,10.5) (x) {$x(t)$};
        \node[left] at (0,6.5) (u) {$u(t)$};
        \node[left] at (0,3.2) (y) {$y(t)$};

        \node[block,minimum height=11cm,minimum width=1.5cm,double border,align=center] at (5.25,5.25) (sys) {non-linear\\static\\parametric\\function\\$f_{NL}(\cdot; \theta)$};

        \draw[->] (u) -- (u1);
        \draw[->] (u1) -- (u2);
        \draw[->,dotted] (u2) -- (un);
        \draw[->] (u1) -- (u1-|sys.west) node[pos=0.5] {$u(t-1)$};
        \draw[->] (u2) -- (u2-|sys.west) node[pos=0.5] {$u(t-2)$};
        \draw[->] (un) -- (un-|sys.west) node[pos=0.5] {$u(t-n_u)$};

        \draw[->] (y) -- (y-|sys.west);
        \draw[->] (1.5,3.2) -- (y1);
        \draw[->] (y1) -- (y2);
        \draw[->,dotted] (y2) -- (yn);
        \draw[->] (y1) -- (y1-|sys.west) node[pos=0.5] {$y(t-1)$};
        \draw[->] (y2) -- (y2-|sys.west) node[pos=0.5] {$y(t-2)$};
        \draw[->] (yn) -- (yn-|sys.west) node[pos=0.5] {$y(t-n_y)$};

        \draw[->] (x) -- (x1);
        \draw[->] (x1) -- (x2);
        \draw[->,dotted] (x2) -- (xn);
        \draw[->] (x1) -- (x1-|sys.west) node[pos=0.5] {$x(t-1)$};
        \draw[->] (x2) -- (x2-|sys.west) node[pos=0.5] {$x(t-2)$};
        \draw[->] (xn) -- (xn-|sys.west) node[pos=0.5] {$x(t-n_x)$};

        \draw[->] (sys) -- ++(2,0) node[right] {$\hat{x}(t|t)$};

    \end{tikzpicture}
    \caption*{Architecture \#3 during training phase}
\end{figure}

\begin{rem}[Pros \& Cons]
	The advantage is that usually in IIR architecture $n_u$ and $n_y$ are much smaller (thanks to recursion): lower computation 
	effort. 

	For the disadvantages we have to notice that only for the training part we use $x(t)$ data from a physical sensor; then, in production, the recursion comes into play since we need to plug-in the output $\hat{x}(t|t)$ in the recursive part: this feedback can provide instability. 

	\begin{figure}[H]
    \centering
    \begin{tikzpicture}[node distance=2cm,auto,>=latex']
        \node[block] at (1.5,0) (yn) {$z^{-1}$};
        \node[block] at (1.5,1.5) (y2) {$z^{-1}$};
        \node[block] at (1.5,2.5) (y1) {$z^{-1}$};

        \node[block] at (1.5,4) (un) {$z^{-1}$};
        \node[block] at (1.5,5.5) (u2) {$z^{-1}$};
        \node[block] at (1.5,6.5) (u1) {$z^{-1}$};

        \node[block] at (1.5,8) (xn) {$z^{-1}$};
        \node[block] at (1.5,9.5) (x2) {$z^{-1}$};
        \node[block] at (1.5,10.5) (x1) {$z^{-1}$};

        \node[left] at (0,10.5) (x) {};
        \node[left] at (0,6.5) (u) {$u(t)$};
        \node[left] at (0,3.2) (y) {$y(t)$};

        \node[block,minimum height=11cm,minimum width=1.5cm,double border,align=center] at (5.7,5.25) (sys) {non-linear\\static\\parametric\\function\\$f_{NL}(\cdot; \theta)$};

        \draw[->] (u) -- (u1);
        \draw[->] (u1) -- (u2);
        \draw[->,dotted] (u2) -- (un);
        \draw[->] (u1) -- (u1-|sys.west) node[pos=0.5] {$u(t-1)$};
        \draw[->] (u2) -- (u2-|sys.west) node[pos=0.5] {$u(t-2)$};
        \draw[->] (un) -- (un-|sys.west) node[pos=0.5] {$u(t-n_u)$};

        \draw[->] (y) -- (y-|sys.west);
        \draw[->] (1.5,3.2) -- (y1);
        \draw[->] (y1) -- (y2);
        \draw[->,dotted] (y2) -- (yn);
        \draw[->] (y1) -- (y1-|sys.west) node[pos=0.5] {$y(t-1)$};
        \draw[->] (y2) -- (y2-|sys.west) node[pos=0.5] {$y(t-2)$};
        \draw[->] (yn) -- (yn-|sys.west) node[pos=0.5] {$y(t-n_y)$};

        \draw[->] (x) -- (x1);
        \draw[->] (x1) -- (x2);
        \draw[->,dotted] (x2) -- (xn);
        \draw[->] (x1) -- (x1-|sys.west) node[pos=0.5] {$\hat{x}(t-1|t-1)$};
        \draw[->] (x2) -- (x2-|sys.west) node[pos=0.5] {$\hat{x}(t-2|t-2)$};
        \draw[->] (xn) -- (xn-|sys.west) node[pos=0.5] {$\hat{x}(t-n_x|t-n_x)$};

        \draw (sys.east) -| ($(x) + (7.5,1.5)$) node[midway, right] {$\hat{x}(t|t)$};

        \draw ($(x) + (7.5,1.5)$) -- ($(x) + (0,1.5)$);

        \draw[->] ($(x) + (0,1.5)$) |- (x1.west) 
        	node[midway, left] {$\hat{x}(t|t)$};

        \draw[dashed, blue] (0.7,7.2) rectangle (4.6,11.3) node[left,above] {recursive part};
    \end{tikzpicture}
    \caption*{Architecture \#3 during production phase}
\end{figure}
\end{rem}

\paragraph{Architecture \#4} Modification with a-priori signal processing of architectures \#1, \#2 and \#3. The idea is to split the SW-sensor in two stages:
\begin{itemize}
	\item first stage: from $u(t)$ and $y(t)$, $h$ \emph{regressors} $r_i(t)$ are produced starting from physical knowledge of the system (where $h$ is much smaller than the number of $u(t)$ and $y(t)$ signals)
	\item second stage: SW-sensor (could be both linear or non-linear and both static or dynamic system) to be firstly trained and then used in production.
\end{itemize}


\begin{figure}[H]
    \centering
    \begin{tikzpicture}[node distance=2cm,auto,>=latex']
        \node[block, dashed border, minimum width=1.5cm, minimum height=3cm, align=center] at (0,0) (sys) {pre-processing\\filter};
        \node[block, dashed border, minimum height=3cm] at (4,0) (f) {$f(\cdot;\theta)$};

        \draw[<-,transform canvas={yshift=0.5cm}] (sys) -- ++(-2cm,0) node[left] {$u(t)$};
        \draw[<-,transform canvas={yshift=-0.5cm}] (sys) -- ++(-2cm,0) node[left] {$y(t)$};

        \draw[->,transform canvas={yshift=1.2cm}] (sys) -- (f) node[pos=0.5] {$r_1(t)$};
        \draw[->,transform canvas={yshift=0.6cm}] (sys) -- (f) node[pos=0.5] {$r_2(t)$};
        \draw[->,transform canvas={yshift=-1.2cm}] (sys) -- (f) node[pos=0.5] {$r_h(t)$};
        \node at (2,0) {$\vdots$};
        \draw[->] (f) -- ++(2cm,0) node[right] {$\hat{x}(t|t)$};
    \end{tikzpicture}
\end{figure}


The idea is to facilitate the estimation of $f(\cdot; \theta)$ by presenting at its input a smaller and more meaningful set of signals (regressors). In this way the \gls{bb} model identification is much simpler.

\paragraph{Conclusions} In case of \gls{bb} SW-sensing with non-linear systems the problem can be quite complex.
Using \emph{brute-force} approach (1 dynamic neural network) is usually doomed to failure.
The best is to gain some insight into the system and build some \emph{smart} regressors before black-box map.

\section{Comparison between \gls{kf} and \gls{bb} software sensing}

\begin{table}[htpb]
    \centering
    \bgroup
    \def\arraystretch{1.5}
    \begin{tabular}{l|c|c}
        & \textbf{\gls{kf}} & \textbf{\gls{bb}} \\
        \hline\hline 
        Need of (\gls{wb}) physical model of the system & \color{red} Yes & \color{green} No \\ \hline 
        Need of a training dataset & \color{green} No {\color{black} \footnote{In practice some tuning through data is needed.}} & \color{red} Yes \\ \hline 
        Interpretability of the obtained SW-sensor & \color{green} Yes & \color{red} No \\ \hline 
        Easy retuning for a similar (different) system & \color{green} Yes & \color{red} No \\ \hline 
        Accuracy of the obtained SW-sensor & \color{green} Good & \color{green} Very Good \\ \hline 
        Can be used also in case of un-measurable states & \color{green} Yes & \color{red} No \\ \hline\hline 
    \end{tabular}
    \egroup
\end{table}
\FloatBarrier

\begin{exa}[Example \ref{ex:KF_full-proc} continued]
    Model (key equation) of the system:
    \[
        M\ddot{z} = -c(t)(\dot{z}-\dot{z}_d) - K(z-z_d)
    \]

    Measurable input $\ddot{z}$ with an accelerometer, $z-z_d$ measurable output with elongation sensor.
    We want to estimate $\dot{z}$.

    The change is $c(t)$ is a semi-active suspension, can be electronically changed (control variable).

    We can solve the problem with \gls{kf} or we can make an experiment and collect training data:
    \begin{align*}
        c(t)        : & \left\{ c(1), c(2), \cdots, c(N) \right\} \\
        z(t)-z_d(t) : & \left\{ z(1)-z_d(1), z(2)-z_d(2), \cdots, z(N)-z_d(N) \right\} \\
        \ddot{z}(t) : & \left\{ \ddot{z}(1), \ddot{z}(2), \cdots, \ddot{z}(N) \right\} \\
        \dot{z}(t)  : & \left\{ \dot{z}(1), \dot{z}(2), \cdots, \dot{z}(N) \right\} \text{ (just for training)} \\
    \end{align*}

    % this part has been done the 18/05/2020

    Back to the main equation:
    \[
        M\ddot{z} = -K(z-z_d)-C(t)(\dot{z}-\dot{z}_d)
    \]
    \[
        \underbrace{\dot{z}}_{\text{to be estim.}} =
        -\frac{K}{M} \underbrace{\int (z-z_d)dt}_{r_1(t)}
        -\frac{1}{M} \underbrace{\int C(t)(\dot{z}-\dot{z}_d)dt}_{r_2(t)}
    \]

    We also consider this equation
    \[
        \dot{z}_d = \underbrace{\int \ddot{z}_d dt}_{r_3(t)}
    \]

    $r_1(t)$ and $r_2(t)$ are the primary regressors, directly linked to $\dot{z}(t)$. $r_3(t)$ is a secondary regressor, it can help $r_1(t)$.

    Since these regressors are obtained by integration to avoid drifting (by DC components of noise integration) we have to high-pass the inputs with high-pass filters $\left(\frac{z-1}{z-a}\right)$.

    \paragraph{Full filtering scheme} \phantom{lol}
    \begin{figure}[H]
        \centering
        \begin{tikzpicture}[node distance=2cm,auto,>=latex']
            \draw[block, dashed border] (0.5,-0.5) rectangle ++(5.5,6);
            \node[block, double border, minimum width=1.5cm, minimum height=6cm] at (8,2.5) (f) {$f(\cdot, \theta)$};
            \node[left] at (0,0) (c) {$c(t)$};
            \node[left] at (0,3) (d) {$z-z_d$};
            \node[left] at (0,5) (z) {$\ddot{z}$};
            \node[sum] at (2,0) (mult) {$\times$};
            \node[block] at (2,1.5) (d1) {$\frac{z-1}{z}$};
            \node[block] at (3.5,0) (d2) {$\frac{z-1}{z-a}$};
            \node[block] at (5,0) (d3) {$\frac{1}{z-1}$};
            \node[block] at (3.5,3) (d4) {$\frac{z-1}{z-a}$};
            \node[block] at (5,3) (d5) {$\frac{1}{z-1}$};
            \node[block] at (3.5,5) (d6) {$\frac{z-1}{z-a}$};
            \node[block] at (5,5) (d7) {$\frac{1}{z-1}$};

            \node[below] at (3,-0.7) {regressors building block};

            \draw[->] (c) -- (mult);
            \draw[->] (d1) -- (mult);
            \draw[->] (mult) -- (d2);
            \draw[->] (d2) -- (d3);
            \draw[->] (z) -- (d6);
            \draw[->] (d6) -- (d7);
            \draw[->] (d) -- (d4);
            \draw[->] (d4) -- (d5);
            \draw[->] (d) -| (d1);

            \draw[->] (d7) -- (d7-|f.west) node[pos=0.7] {$r_3(t)$};
            \draw[->] (d5) -- (d5-|f.west) node[pos=0.7] {$r_1(t)$};
            \draw[->] (d3) -- (d3-|f.west) node[pos=0.7] {$r_2(t)$};
            \draw[->] (f) -- (9.5,2.5) node[right] {$\hat{\dot{z}}$};
        \end{tikzpicture}
    \end{figure}
\end{exa}


\chapter{Gray-Box System Identification}\label{ch5}
%!TEX root = ../main.tex

%TODO

We'll see two approaches
\begin{itemize}
    \item Using Kalman Filter
    \item Using Simulation error method
\end{itemize}

\section{Using Kalman Filter}
Kalman Filter is not a system identification method, it is a variable estimation approach (software-sensor, observer).
However we can use it also for gray-box system identification (\emph{side benefit} of K.F.).

\paragraph{Problem definition} We have a model, typically built as a white-box model using first principles:
\[
    S: \begin{cases}
        x(t+1) = f(x(t), u(t), \theta) + v_1(t) \\
        y(t) = h(x(t), \theta) + v_2(t)
    \end{cases}
\]

$f$ and $h$ are linear or non-linear functions, depending on some unknown parameter $\theta$ (with a physical meaning, e.g. mass, resistance, friction, \dots).

The problem is to estimate $\hat{\theta}$.

K.F. solves this problem by transforming the unknown parameters in extended states: K.F. makes the simultaneous estimation of $\hat{x}(t|t)$ (classic Kalman Filter problem) and $\hat{\theta}(t)$ (parameter identification problem).

\paragraph{Trick} State extension

\[
    S: \begin{cases}
        x(t+1) = f(x(t), u(t), \theta(t)) + v_1(t) \\
        \theta(t+1) = \theta(t) + v_\theta(t) \\
        y(t) = h(x(t), \theta(t)) + v_2(t)
    \end{cases}
\]

The new extended state vector is $x_E = \begin{bmatrix} x(t) \\ \theta(t) \end{bmatrix}$.
The unknown parameters are transformed in unknown variables.

The new equation we created
\[
    \theta(t+1) = \theta(t) + v_\theta(t)
\]
It is a \emph{fictitious} equation (not a physical equation).

The core dynamics is $\theta(t+1)=\theta(t)$, it's the equations of something which is constant.
This is exactly the nature of $\theta(t)$ which is indeed a constant vector of parameters.

We need a \emph{fictitious} noise in order to force Kalman Filter to find the right value of $\theta$ (if no noise in this equation K.F. probably would stay fixed on the initial condition).
We tell K.F. to do not rely on initial conditions.

Notice that this equation is not of an asymptotic stable system but a simply-stable system.
It's not a problem because K.F. can deal with non-asymptotically stable systems.

\paragraph{Design choice} The choice of the covariance matrix of $v_\theta(t) \sim WN(0, V_\theta)$.

We make the empirical assumption that $v_1 \perp v_\theta$ and $v_2 \perp v_\theta$ (there is no reason for $v_\theta$ to be correlated with $v_1$ and $v_2$).
\[
    V_\theta = \begin{bmatrix}
        \lambda_{1\theta}^2 & & & \\
        & \lambda_{2\theta}^2 & & \\
        & & \ddots & \\
        & & & \lambda_{n_\theta\theta}^2\\
    \end{bmatrix}
\]

It is a $n_\theta\times n_\theta$ and usually it is assumed that $\lambda_{1\theta}^2=\lambda_{2\theta}^2=\dots=\lambda_{n_\theta\theta}^2$.
We assume that $v_\theta(t)$ is a set of independent W.N. all with the same variance $\lambda_\theta^2$ (tuned empirically).

\begin{figure}[H]
    \centering
    \begin{tikzpicture}[
            node distance=2cm,auto,>=latex',
            declare function={
                f1(\x) = (\x < 2) * (\x/2*(3-1)) +
                         (\x >= 2) * (3-1) +
                         (\x > 0.2) * rand/2.5 +
                         1;
                f2(\x) = (\x < 4.5) * (\x/4.5*(3-1)) +
                         (\x >= 4.5) * (3-1) +
                         (\x > 0.2) * rand/15 +
                         1;
            }
        ]
        \draw[->] (0,0) -- (6,0) node[below] {$t$};
        \draw[->] (0,0) -- (0,4) node[left] {$\theta$};

        \node[green] at (2.3,1.5) {\footnotesize small $\lambda_\theta^2$};
        \node[blue] at (2.3,3.7) {\footnotesize big $\lambda_\theta^2$};

        \draw[dotted] (6,3) -- (0,3) node[left] {$\overline{\theta}$};
        \draw[domain=0:5.5,smooth,variable=\x,blue,samples=70] plot ({\x},{f1(\x)});
        \draw[domain=0:5.5,smooth,variable=\x,green,samples=70] plot ({\x},{f2(\x)});

        \draw[mark=*] plot coordinates {(0,1)} node[left, align=right] {Initial\\condition};
    \end{tikzpicture}
    \caption*{Influence of choice of $\lambda_\theta^2$}
\end{figure}

With a small value of $\lambda_\theta^2$ there is a slow convergence with small oscillations (noise) at steady-state (big trust to initial conditions).
With large values of $\lambda_\theta^2$ there's fast convergence but noisy at steady-state.

$\lambda_\theta^2$ is selected according to the best compromise for your specific application.

\paragraph{Notice} This trick can work in principle with any number of unknown parameters (e.g. 3 sensors, 10 states and 20 parameters).
In practice it works well only on a limited number of parameters (e.g. 3 sensors, 5 states and 2 parameters).

\begin{example}
    \begin{figure}[H]
        \centering
        \begin{tikzpicture}[node distance=2cm,auto,>=latex']
            \draw (0,0) -- (6,0);
            \draw[pattern=north east lines] (0,0) rectangle (-0.5,2);
            \draw[] (2,0) rectangle (4,1.5);
            \node at (3,0.75) {$M$};
            \draw[decoration={aspect=0.3, segment length=1mm, amplitude=2mm,coil},decorate] (0.5,0.75) -- (1.5,0.75);
            \draw (0,0.75) -- (0.5,0.75);
            \draw (1.5,0.75) -- (2,0.75);
            \draw[->] (4,0.75) -- (5,0.75) node[right] {$F$};
            \fill[pattern=north east lines] (1.8,0) rectangle ++(2.4,-0.1) node[above right] {\footnotesize friction $c$};
            \draw[->] (3,-0.5) -- (4,-0.5) node[right] {$x$};
            \draw (3,-0.55) -- (3,-0.45);
        \end{tikzpicture}
    \end{figure}

    \begin{description}
        \item[Input] $F(t)$
        \item[Output] Position $x(t)$ (measured)
        \item[Parameters] $K$ and $M$ are known (measured), $c$ is unknown
    \end{description}

    \begin{figure}[H]
        \centering
        \begin{tikzpicture}[node distance=2cm,auto,>=latex']
            \node[block,align=center] at (0,0) (s) {system\\$c = ?$};
            \draw[<-] (s) -- ++(-1.5,0) node[left] {$F(t)$};
            \draw[->] (s) -- ++(1.5,0) node[right] {$x(t)$};
        \end{tikzpicture}
    \end{figure}

    \paragraph{Problem} Estimate $c$ with a K.F.

    Using K.F. we do not need a training dataset.

    \paragraph{Step 1} Model the system
    \[
        \ddot{x}M = -Kx - c\dot{x} + F
    \]
    It's a differential, continuous time linear equation.
    It's second order so we need 2 state variables: $x_1(t) = x(t)$ and $x_2(t) = \dot{x}(t)$.
    \[
        \begin{cases}
            \dot{x}(t) = x_2(t) \\
            M\dot{x}_2(t) = -Kx_1(t) -cx_2(t) + F(t) \\
            y(t) = x_1(t)
        \end{cases}
        \begin{cases}
            \dot{x}_1(t) = x_2(t) \\
            \dot{x}_2(t) = -\frac{K}{M} x_1(t) - \frac{c}{M} x_2(t) + \frac{1}{M}F(t) \\
            y(t) = x_1(t)
        \end{cases}
    \]

    \paragraph{Step 2} Discretization

    Eulero forward: $\dot{x}(t) \approx \frac{x(t+1)-x(t)}{\Delta}$.

    \[
        \begin{cases}
            \frac{x_1(t+1)-x_1(t)}{\Delta} &= x_2(t) \\
            \frac{x_2(t+1)-x_2(t)}{\Delta} &= -\frac{K}{M} x_1(t) - \frac{c}{M} x_2(t) + \frac{1}{M}F(t) \\
            y(t) &= x_1(t)
        \end{cases}
    \]
    \[
        \begin{cases}
            x_1(t+1) = x_1(t) + \Delta x_2(t) \\
            x_2(t+1) = -\frac{K\Delta}{M}x_1(t) + \left(1-\frac{c\Delta}{M}\right)x_2(t) + \frac{\Delta}{M}F(t) \\
            y(t) = x_1(t)
        \end{cases}
    \]

    \paragraph{Step 3} State extension
    \[
        x_3(t+1) = x_3(t)
    \]
    \[
        \begin{cases}
            x_1(t+1) = x_1(t) + \Delta x_2(t) + v_{11}(t) \\
            x_2(t+1) = -\frac{K\Delta}{M}x_1(t) + \left(1-\frac{\Delta {\color{red}x_3(t)}}{M}\right)x_2(t) + \frac{\Delta}{M}F(t) + v_{12}(t) \\
            {\color{red}x_3(t+1) = x_3(t) + v_{13}(t)} \\
            y(t) = x_1(t) + v_2(t)
        \end{cases}
    \]

    The system is ready for K.F. application: we get at the same time $\hat{x}(t)$ and $\hat{c}(t)$.

    Notice that we need Extended Kalman Filter: even if the original system was linear, state extension moved to a non-linear system.
\end{example}


\section{Using Simulation Error Method}

Are there alternative ways to solve gray-box system identification problems?
A commonly (and intuitive) used method is parametric identification approach based on Simulation Error Method (SEM).

\begin{figure}[H]
    \centering
    \begin{tikzpicture}[node distance=2cm,auto,>=latex']
        \node[block, align=center] at (0,0) (sys) {Model with\\some unknown\\parameters};
        \draw[<-] (sys) -- ++(-2,0) node[left] {$u(t)$};
        \draw[->] (sys) -- ++(2,0) node[right] {$y(t)$};
    \end{tikzpicture}
\end{figure}

\paragraph{Step 1} Collect data from an experiment

\begin{align*}
    \{ \tilde{u}(1), \tilde{u}(2), \dots, \tilde{u}(N) \} \\
    \{ \tilde{y}(1), \tilde{y}(2), \dots, \tilde{y}(N) \}
\end{align*}

\paragraph{Step 2} Define model structure
\[
    y(t) = \mathcal{M}(u(t), \overline{\theta}, \theta)
\]
Mathematical model (linear or non-linear) usually written from first principle equations. $\overline{\theta}$ is the set of known parameters (mass, resistance, \dots), $\theta$ is the set of unknown parameters (possibly with bounds).

\paragraph{Step 3} Performance index definition
\[
    J_N(\theta) = \frac{1}{N} \sum_{t=1}^N \left( \tilde{y}(t) - \mathcal{M}(\tilde{u}(t), \overline{\theta}, \theta) \right)^2
\]

\paragraph{Step 4} Optimization

\[
    \hat{\theta}_N = \argmin_\theta J_N(\theta)
\]

\begin{itemize}
    \item Usually no analytic expression of $J_N(\theta)$ is available.
    \item Each computation of $J_N(\theta)$ requires an entire simulation of the model from $t=1$ to $t=N$.
    \item Usually $J_N(\theta)$ is a non-quadratic and non-convex function. Iterative and randomized optimization methods must be used.
    \item It's intuitive but very computationally demanding.
\end{itemize}

\begin{figure}[H]
    \centering
    \begin{tikzpicture}[node distance=2cm,auto,>=latex']
        \draw (0,2) rectangle ++(2,2);
        \node[align=center] at (1,3) {Model of\\the system};
        \node[draw, ellipse, align=center] at (1,0) (m) {Model\\$\mathcal{M}(\theta)$};
        \node[sum] at (4,0) (sum) {};
        \node[block] at (4,-1.5) (J) {$J(\theta)$};

        \draw[<-] (m) -- (-1,0) node[left] {$\tilde{F}(t)$};
        \draw[->] (m) -- (sum) node[pos=0.8] {+} node[pos=0.5] {$\hat{y}(t)$};
        \draw[->] (4,3) -- (sum) node[pos=0.9] {-};
        \draw[->] (sum) -- (J) node[pos=0.5, right] {\footnotesize simulation error};
        \draw[->] (J) -| (m);

        \draw[<-] (0,3) -- (-1,3) node[left] {$\tilde{F}(t)$};
        \draw[->] (2,3) -- (5,3) node[right] {$\tilde{y}(t)$};
    \end{tikzpicture}
\end{figure}

Can S.E.M. be applied also to B.B. methods?

\begin{example}
    We collect data $\{ \tilde{u}(1), \tilde{u}(2), \dots, \tilde{u}(N) \}$ and $\{ \tilde{y}(1), \tilde{y}(2), \dots, \tilde{y}(N) \}$, we want to estimate from data the I/O model.

    \[
        y(t) = \frac{b_0 + b_1z^{-1}}{1+a_1z^{-1} + a_2z^{-2}}u(t-1) \qquad \theta = \begin{bmatrix}
            a_1 \\ a_2 \\ b_0 \\ b_1
        \end{bmatrix}
    \]

    In time domain $y(t) = -a_1y(t-1)-a_2y(t-2)+b_0u(t-1)+b_1u(t-2)$.

    Using P.E.M.
    \[
        \hat{y}(t|t-1) = -a_1\hat{y}(t-1)-a_2\hat{y}(t-2)+b_0\hat{u}(t-1)+b_1\hat{u}(t-2)
    \]
    \begin{align*}
        J_N(\theta) &= \frac{1}{N}\sum_{t=1}^N \left( \tilde{y}(t) - \hat{y}(t|t-1, \theta) \right)^2 \\
        &= \frac{1}{N}\sum_{t=1}^N \left( \tilde{y}(t) +a_1\tilde{y}(t-1)+a_2\tilde{y}(t-2)-b_0\tilde{u}(t-1)-b_1\tilde{u}(t-2) \right)^2 \\
    \end{align*}

    Notice that it's a quadratic formula.

    \begin{figure}[H]
        \begin{minipage}[t]{0.5\textwidth}
            \centering
            \begin{tikzpicture}[node distance=2.5cm,auto,>=latex']
                \node[block] at (1,1) (zu) {$z^{-1}$};
                \node[block] at (1,3.5) (zy) {$z^{-1}$};

                \node at (0,2) (u) {$\tilde{u}(t)$};
                \node at (0,4.5) (y) {$\tilde{y}(t)$};

                \node[block,minimum height=5cm,minimum width=1.5cm,align=center] at (3.5,2.5) (sys) {Linear\\function\\of $\theta$};

                \draw[->] (zu) -- (zu-|sys.west) node[pos=0.5] {$\scriptstyle\tilde{u}(t-1)$};
                \draw[->] (u) -- (u-|sys.west);
                \draw[->] (1,2) -- (zu);
                \draw[->] (zy) -- (zy-|sys.west) node[pos=0.5] {$\scriptstyle\tilde{y}(t-1)$};
                \draw[->] (y) -- (y-|sys.west);
                \draw[->] (1,4.5) -- (zy);
                \draw[->] (sys) -- ++(2,0) node[above] {$\scriptstyle\hat{y}(t|t-1)$};
            \end{tikzpicture}
            \caption*{P.E.M.}
        \end{minipage}
        \begin{minipage}[t]{0.4\textwidth}
            \centering
            \begin{tikzpicture}[node distance=2.5cm,auto,>=latex']
                \node[block] at (1,1) (zu) {$z^{-1}$};
                \node[block] at (1,3) (zy) {$z^{-1}$};
                \node[block] at (1,4.5) (zy2) {$z^{-1}$};

                \node at (0,2) (u) {$\tilde{u}(t)$};

                \node[block,minimum height=5cm,minimum width=1.5cm,align=center] at (3.5,2.5) (sys) {Linear\\function\\of $\theta$};

                \draw[->] (zu) -- (zu-|sys.west) node[pos=0.5] {$\scriptstyle\tilde{u}(t-1)$};
                \draw[->] (u) -- (u-|sys.west);
                \draw[->] (1,2) -- (zu);

                \draw[->] (zy) -- (zy-|sys.west) node[pos=0.5] {$\scriptstyle\hat{y}(t-2)$};
                \draw[->] (zy2) -- (zy2-|sys.west) node[pos=0.5] {$\scriptstyle\hat{y}(t-1)$};
                \draw[->] (zy2) -- (zy);
                \draw[->] (4.8,2.5) -- (4.8,5.5) -- (1,5.5) -- (zy2);

                \draw[->] (sys) -- ++(2,0) node[above] {$\scriptstyle\hat{y}(t|t-1)$};
            \end{tikzpicture}
            \caption*{S.E.M.}
        \end{minipage}
    \end{figure}

    Using S.E.M.
    \[
        \hat{y}(t|t-1) = -a_1\hat{y}(t-1)-a_2\hat{y}(t-2)+b_0\tilde{u}(t-1)+b_1\tilde{u}(t-2)
    \]
    \begin{align*}
        J_N(\theta) &= \frac{1}{N}\sum_{t=1}^N \left( \tilde{y}(t) - \hat{y}(t|t-1, \theta) \right)^2 \\
        &= \frac{1}{N}\sum_{t=1}^N \left( \tilde{y}(t) +a_1\hat{y}(t-1)+a_2\hat{y}(t-2)-b_0\tilde{u}(t-1)-b_1\tilde{u}(t-2) \right)^2 \\
    \end{align*}

    Notice that it's non-linear with respect to $\theta$.
\end{example}

P.E.M. approach looks much better, but do not forget the noise! P.E.M. is much less robust w.r.t. noise, we must include a model of the noise in the estimated model.
We use ARMAX models.

If we use ARX models:
\[
    y(t) = \frac{b_0+b_1z^{-1}}{1+a_1z^{-1}+a_2z^{-2}}u(t-1) + \frac{1}{1+a_1z^{-1}+a_2z^{-2}}e(t)
\]
\[
    \hat{y}(t|t-1) = b_0u(t-1)+b_1u(t-2) - a_1y(t-1)-a_2y(t-2)
\]

If we use ARMAX models the numerator of the T.F. for $e(t)$ is $1+c_1z^{-1}+\ldots+c_mz^{-m}$, in this case $J_N(\theta)$ is non-linear.
This leads to the same complexity of S.E.M.

The second problem of P.E.M. is high sensitivity to sampling time choice.
Remember that when we write at discrete time $y(t)$ we mean $y(t\cdot \Delta)$.

\[
    \hat{y}(t|t-1) = -a_1\tilde{y}(t-1)-a_2\tilde{y}(t-2) + b_0\tilde{u}(t-1)+b_1\tilde{u}(t-2)
\]

If $\Delta$ is very small the difference between $\tilde{y}(t)$ and $\tilde{y}(t-1)$ becomes very small.
The effect is that the P.E.M. optimization ends to provide this \emph{trivial} solution:
\[
    a_1 = -1 \qquad a_2 \rightarrow 0 \qquad b_0 \rightarrow 0 \qquad b_1 \rightarrow 0 \qquad \Rightarrow \qquad \tilde{y}(t) \approx \tilde{y}(t-1)
\]

This is a wrong model due to the fact that the recursive part of the model is using past measures of the output instead of past values of the model outputs.

\section{Conclusion}

Summary of system identification methods for I/O systems
\begin{figure}[H]
    \centering
    \begin{tikzpicture}[node distance=2cm,auto,>=latex']
        \node[block, dashed border, minimum width=1.5cm, minimum height=1.5cm] at (0,0) (sys) {};
        \draw[<-] (sys) -- ++(-1.5,0) node[left] {$u(t)$};
        \draw[->] (sys) -- ++(1.5,0) node[right] {$y(t)$};
    \end{tikzpicture}
\end{figure}

\begin{itemize}
    \item Collect a dataset for training (if needed)
    \item Choose a model domain (linear static/non-linear static/linear dynamic/non-linear dynamic), using gray-box or black-box
    \item Estimation method: constructive (4SID), parametric (P.E.M. or S.E.M.) or filtering (state extension of K.F.)
\end{itemize}

Better black-box for system identification and software-sensing or white box?

It depends on the goals and type of applications.

\begin{itemize}
    \item Black box is very general and very flexible, make maximum use of data and no or little need of domain knowhow
    \item White box is very useful when you are the system-designer (not only the control algorithm designer), can provide more insight in the system.
    \item Gray box can sometimes be obtained by hybrid systems (part is black-box and part is white-box).
\end{itemize}
% -----------------------------

\chapter{Minimum Variance Control}
Minimum Variance Control (MVC) is about design and analysis of feedback systems, it is not about system identification nor software sensing.

Why we dedicate a chapter on \emph{control}?
\begin{itemize}
    \item Control design is the main motivation to system identification and software sensing.
    \item MVC is based on \emph{mathematics} of system identification and software sensing (prediction theory).
    \item MVC can be considered as a general tool of stochastic optimization of feedback systems.
\end{itemize}

\section{MVC System}

\paragraph{Setup of the problem} Consider a generic \gls{armax} model

\begin{recall}[\gls{armax} system]
	\[
	    y(t) = \frac{B(z)}{A(z)}u(t-k) + \frac{C(z)}{A(z)}e(t) \qquad e(t) \sim WN(0, \lambda^2)
	\]
	\begin{align*}
	    B(z) &= b_0 + b_1z^{-1} + \dots + b_pz^{-p} \\
	    A(z) &= 1   + a_1z^{-1} + \dots + a_mz^{-m} \\
	    C(z) &= 1   + c_1z^{-1} + \dots + c_nz^{-n}
	\end{align*}

	\textbf{Note} The input is $u(t)$ and the output is $y(t)$. The noise $e(t)$ is a non-measurable input that models the uncertainty of the system.  
\end{recall}

\subparagraph{Assumptions}
\begin{itemize}
    \item $\frac{C(z)}{A(z)}$ is in \emph{canonical form}
    \item $b_0\ne 0$ (this implies that $k \ge 1$ is the \emph{pure delay}, which is not greater than $k$)
    \item $\frac{B(z)}{A(z)}$ is \emph{minimum phase}
\end{itemize}

\begin{recall}[Transfer function in canonical form]
    A transfer function $W(z) = \frac{C(z)}{A(z)}$ is in \emph{canonical form} if 
    \begin{enumerate}
        \item $C(z)$ and $A(z)$ are monic (i.e. the coefficients of the maximum degree terms of $C(z)$ and $A(z)$ are equal to 1);
        \item $C(z)$ and $A(z)$ have null relative degree (i.e. they share the same degree);
        \item $C(z)$ and $A(z)$ are coprime (i.e. they have no common factors);
        \item[4a.]the poles of $W(z)$ are such that $|z| < 1$;
        \item[4b.]the zeros of $W(z)$ are such that $|z| < 1$. (more stringent condition than the one we saw in MIDA1)
    \end{enumerate}
\end{recall}

\begin{recall}[Minimum Phase filter]
    A filter described by a \gls{tf} $\frac{B(z)}{A(z)}$ is said to be \emph{minimum phase} if all the roots of $B(z)$ are strictly inside the unit circle.
\end{recall}

\begin{remark}[Minimum Phase filter in practice]
    What does it means in practice? 

    \begin{figure}[H]
        \centering
        \begin{tikzpicture}[node distance=2cm,auto,>=latex']
            \draw[->] (0,4) -- (0,7) node[left] {$u(t)$};
            \draw[->] (0,4) -- (5,4) node[below] {$t$};
            \draw[->] (0,0) -- (0,3) node[left] {$y(t)$};
            \draw[->] (0,0) -- (5,0) node[below] {$t$};

            \draw[line width=0.3mm] (0,4) -- (1,4) -- (1,6) -- (5,6);
            \draw[dotted] (1,0) -- (1,4);
            \draw[dotted] (0,2.5) -- (5,2.5) node[right] {\footnotesize steady state};

            \draw[domain=1:4.5,smooth,variable=\x,green,samples=70] plot ({\x},{2.5-2.5*(1-(\x-1)/3.5)^5});
            \draw[domain=1:4.5,smooth,variable=\x,green,samples=70] plot ({\x},{2.5-2.5*(1-(\x-1)/3.5)^5+sin(\x*180)/\x});
            \draw[domain=1:4.5,smooth,variable=\x,red,samples=70] plot ({\x},{2.5-2.5*(1-(\x-1)/3.5)^5+3*sin(\x*180)/(\x^2)});

            \node[red,right] at (1.5,-0.5) {\footnotesize non- minimum phase};
            \node[green,right] at (2,1.5) {\footnotesize minimum phase};
        \end{tikzpicture}
    \end{figure}

    At the very beginning the response of a \textcolor{red}{non-minimum phase} system goes to the opposite direction w.r.t. the final value. Intuitively it's very difficult to control non-minimum phase systems: you can take the wrong decision if you react immediately.

    Also for human it's difficult, for example \emph{steer to roll} dynamics in a bicycle: if you want to steer left, you must first steer a little to the right and then turn left.

    Design of controller for non-minimum phase is difficult and requires special design techniques (no MVC but \emph{generalized MVC (GMVC)}, described in \ref{subsec:GMVC}).
\end{remark}

\paragraph{Goal of the problem} 
The problem we wish to solve is the optimal tracking of the desired behavior of the output (which is the classical goal of control systems):
\begin{figure}[H]
    \centering
    \begin{tikzpicture}[node distance=2cm,auto,>=latex']
        \node[block,ellipse,align=center] at (-1,0) (cont) {controller\\algorithm};
        \node[block] at (2.5,0) (ba) {$\frac{B(z)}{A(z)}z^{-k}$};
        \node[block] at (4,1.5) (ca) {$\frac{C(z)}{A(z)}$};
        \node[sum] at (4,0) (sum) {};

        \draw[dotted] (0.8,-1) rectangle (5,3.2) node[right] {system to be controlled};
        \draw[->] (cont) -- (ba) node[pos=0.7] {$u(t)$};
        \draw[->] (ba) -- (sum);
        \draw[->] (ca) -- (sum);
        \draw[<-] (cont) -- ++(-2,0) node[left] {$y^0(t)$};
        \draw[<-] (ca) -- ++(0,1) node[above] {$e(t)$};
        \draw[->] (sum) -- ++(2,0) node[right] {$y(t)$};
        \draw[->] (5.5,0) -- (5.5,-1.5) -- (-1,-1.5) -- (cont);
    \end{tikzpicture}
\end{figure}
where $y^0(t)$ os the desired output value, called \emph{reference}.

Some additional (small) technical \textbf{assumptions}:
\begin{itemize}
    \item $y^0(t)$ and $e(t)$ are not correlated (usually fulfilled).
    \item We assume (worst case) that $y^0(t)$ is known only up to time $t$ (present time): we have no preview of the future desired $y^0(t)$ (i.e. $y^0(t)$ is totally unpredictable or $\hat{y^0}(t+k|t) = y^0(t)$).
\end{itemize} 

In a more formal way MVC is an optimization control problem that tries to find $u(t)$ that minimize this performance index:
\[
    J = E\left[ (y(t) - y^0(t))^2 \right]
\]

where $J$ is the variance of the tracking error: that's why it's called Minimum Variance Control.

\paragraph{Solution of the problem}
The main trick is to split $y(t)$ into $\hat{y}(t|t-k)$, the \emph{predictor}, and $\epsilon(t)$, the \emph{prediction error}.

Now we can write $J$ as 
\begin{align*}
	J =& \E \left[ \left( \hat{y}(t|t-k) + \epsilon(t) - y^0(t) \right)^2 \right] \\
	  =& \E \left[ \left( (\hat{y}(t|t-k) - y^0(t)) + \epsilon(t) \right)^2 \right] \\
	  =& \E \left[ \left( (\hat{y}(t|t-k) - y^0(t) \right)^2 \right] + \E [\epsilon(t)^2 ] + 2 \E \left[ \epsilon(t) \left( \hat{y}(t|t-k) - y^0(t) \right) \right] 
\end{align*}

where we note that

\begin{itemize}
	\item $\E [\epsilon(t)^2 ]$ doesn't depend on $u(t)$: it's not subject to minimization, since it's just a constant number.
	\item $\E [ \epsilon(t) y^0(t) ] = 0$, since $\epsilon(t) = f(e(t), e(t-1), \dots) $ and, by assumption, $y^0 \perp e$.
	\item $\E [ \epsilon(t) \hat{y}(t|t-k) ] = 0$, since by construction $\epsilon(t) \perp \hat{y}(t|t-k)$ (otherwise $\hat{y}(t|t-k)$ wouldn't be the optimal predictor).
\end{itemize}

Therefore, minimizing $J$ is equivalent to minimizing  $\E \left[ \left( (\hat{y}(t|t-k) - y^0(t) \right)^2 \right]$, which is minimized if

\begin{align}\label{eq:min_J_condition}
	\hat{y}(t|t-k) = y^0(t)
\end{align}

Now we have to compute $\hat{y}(t|t-k)$ and impose $\hat{y}(t+k|t) = y^0(t+k)$ (obtained by shifting the equation \ref{eq:min_J_condition} using the $z$ operator).

However, by assumption, at time $t$ we don't know $y^0(t+k)$: the best we can do is to replace it with the last available value of $y^0$, that is $y^0(t)$.

Therefore, the condition \ref{eq:min_J_condition} becomes 

\begin{align}\label{eq:new_min_J_condition}
	\hat{y}(t+k|t) = y^0(t)
\end{align}

\begin{recall}[$k$-step predictor of an \gls{armax} system]

	\[
		\hat{y}(t+k|t) = \frac{B(z) E(z)}{C(z)} u(t) + \frac{\tilde{R}(z)}{C(z)} y(t)
	\] 

	where $E(z)$ is the $result$ and $R(z) = \tilde{R}(z) z^{-k}$ is the $residual$ of the $k$-step long division between $C(z)$ and $A(z)$.

\end{recall}

By plugging in the $k$-step predictor of an \gls{armax} system formula in the equation \ref{eq:new_min_J_condition} we obtain

\[
	\frac{B(z) E(z)}{C(z)} u(t) + \frac{\tilde{R}(z)}{C(z)} y(t) = y^0(t)
\]

and by making $u(t)$ explicit we obtain the \emph{General Formula of MVC}:

\begin{align}\label{eq:MVC_general-formula}
	u(t) = \frac{1}{B(z)E(z)} \left( C(z) y^0(t) - \tilde{R}(z) y(t) \right)
\end{align}

The block scheme representation of the MVC system is

\begin{figure}[H]
    \centering
    \begin{tikzpicture}[node distance=2cm,auto,>=latex']
        \node[sum] at (0,0) (sum) {};
        \node[block, left=1cm of sum] (c1) {$C(z)$}; 
        \node[block] at (1.5,0) (b1) {$\frac{1}{B(z) E(z)}$};
        \node[block] at (5,0) (b2) {$\frac{B(z)}{A(z)} z^{-k}$};
        \node[block] at (3,-1.5) (b3) {$\tilde{R}(z)$};
        \node[block] at (7,1.5) (b4) {$\frac{C(z)}{A(z)}$};
        \node[sum] at (7,0) (sum2) {};

        \draw[<-] (c1) --++ (-1.5,0) {} node[left] (in) {$y^0(t)$};
        \draw[->] (sum) -- (b1);
        \draw[->] (b1) -- (b2) node[pos=0.5] {$u(t)$};
        \draw[->] (b3) -| (sum) node[pos=0.9] {$-$};
        \draw[->] (c1) -- (sum)  node[above, near end] {$+$};
        \draw[->] (sum2) -- ++(1.5,0) node[right] {$y(t)$};
        \draw[->] (b2) -- (sum2);
        \draw[->] (b4) -- (sum2);
        \draw[<-] (b4) -- ++(0,1) node[above] (noise_in) {$e(t)$};
        \draw[->] (8,0) |- (b3);

        \draw[dashed, blue] ($(in) + (0.9, -2.5)$) rectangle ($(b2) + (-1.2, 1)$) node[above=0.6cm of c1] {\text{MVC (controller)}};
        \draw[dashed] ($(b2) + (-1, -0.6)$) rectangle ($(noise_in) + (0.7, 0.3)$) node[right] {\text{system}};
    \end{tikzpicture}

    \caption{MVC System}
    \label{fig:MVC_sys}
\end{figure}

\clearpage

\section{Analysis of the MVC System}

\subsection{Stability of the MVC System}\label{subsec:MVC_stability}

\begin{recall}
    For stability let's recall a result of \emph{negative feedback system}:
    \begin{figure}[H]
        \centering
        \begin{tikzpicture}[node distance=2cm,auto,>=latex']
            \node[block] at (0,1.5) (f1) {$F_1(t)$};
            \node[block] at (0,0.5) (f2) {$F_2(t)$};
            \node[sum] at (-1.5,1.5) (sum) {};

            \draw[<-] (sum) -- ++(-1,0) node[pos=0.2, above] {$+$};
            \draw[->] (f2) -| (sum) node[pos=0.9] {$-$};
            \draw[->] (sum) -- (f1);
            \draw[->] (f1) -- ++(2,0);
            \draw[->] (1.5,1.5) |- (f2);
        \end{tikzpicture}
    \end{figure}

    To check the closed-loop stability:
    \begin{itemize}
        \item compute the \emph{loop-function} $L(z) = F_1(z) F_2(z)$ (\textbf{remember}: do not simplify!)
        \item build the \emph{characteristic polynomial} $\chi(z) = L_N(z) + L_D(z)$ (sum of the numerator and the denominator of $L(z)$)
        \item find the roots of $\chi(z)$: closed loop system is asymptotically stable if and only if all the roots of $\chi(z)$ are strictly inside the unit circle
    \end{itemize}
\end{recall}

Check the stability of the MVC system represented in figure \ref{fig:MVC_sys}

\[
    L(z) = \frac{1}{B(z)E(z)}\cdot \frac{z^{-k}B(z)}{A(z)}\cdot\tilde{R}(z)
\]
\begin{align*}
    \chi(z) &= z^{-k}B(z)\tilde{R}(z) + B(z)E(z)A(z) \\
    &= B(z) \underbrace{\left( z^{-k}\tilde{R}(z)+E(z)A(z) \right)}_{C(z)}  \\
    &= B(z)C(z)
\end{align*}

where we used the following result of the long division between $C(z)$ and $A(z)$:

\[ \frac{C(z)}{A(z)} = E(z) + \frac{z^{-k} \tilde{R}(z)}{A(z)} \]

The MVC system is always guaranteed asymptotically stable, since the roots of $\chi(z)$ are the roots of $B(z)$ and $C(z)$ and:
\begin{itemize}
    \item all roots of $B(z)$ are strictly inside the unit circle (thanks to minimum phase assumption on $\frac{B(z)}{A(z)}$)
    \item all roots of $C(z)$ are inside the unit circle (thanks to the assumption of canonical representation of $\frac{C(z)}{A(z)}$)
\end{itemize}

\subsection{Performance of the MVC System}\label{subsec:MVC_performance}

The system can be rewritten as follows considering the two inputs: $y^0(t)$ and $e(t)$ (non-measurable input)
\[ 
	y(t) = F_{y^0y}(z) y^0(t) + F_{ey}(z) e(t)
\]

\begin{recall}
    The transfer function from the input to the output of a \emph{negative feedback system} can be computed with:
    \begin{figure}[H]
        \centering
        \begin{tikzpicture}[node distance=2cm,auto,>=latex']
            \node[block] at (0,1.5) (f1) {$F_1(t)$};
            \node[block] at (0,0.5) (f2) {$F_2(t)$};
            \node[sum] at (-1.5,1.5) (sum) {};

            \draw[<-] (sum) -- ++(-1,0) node[left] {$u(t)$} node[pos=0.2, above] {$+$} ;
            \draw[->] (f2) -| (sum) node[pos=0.9] {$-$};
            \draw[->] (sum) -- (f1);
            \draw[->] (f1) -- ++(2,0) node[right] {$y(t)$};
            \draw[->] (1.5,1.5) |- (f2);
        \end{tikzpicture}
    \end{figure}

    \[ 
    	F(z) = \frac{F_1(z)}{1 + F_1(z) F_2(z)} \qquad y(t) = F(z) u(t)
    \]
    where we recall that $F_1(z) F_2(z)$ is the loop-function $L(z)$
 	and $F_1(z)$ is the direct line from the input to the output.
\end{recall}

Therefore, looking at the figure \ref{fig:MVC_sys}, the \gls{tf} from $y^0(t)$ to $y(t)$ is
\begin{align*}
	F_{y^0y}(z) =& \frac{C(z) \cdot \frac{1}{B(z) E(z)} \cdot \frac{z^{-k} B(z)}{A(z)}}{1 + \underbrace{\frac{1}{B(z)E(z)}\cdot \frac{z^{-k}B(z)}{A(z)}\cdot\tilde{R}(z)}_{L(z)}} \\
	=& \dots \\
	=& z^{-k}
\end{align*}
where we remind that $L(z)$ has been already computed for the stability check (\ref{subsec:MVC_stability}). 

Similarly, the \gls{tf} from $e(t)$ to $y(t)$ is 
\begin{align*}
	F_{ey}(z) =& \frac{\frac{C(z)}{A(z)}}{1 + \underbrace{\frac{1}{B(z)E(z)}\cdot \frac{z^{-k}B(z)}{A(z)}\cdot\tilde{R}(z)}_{L(z)}} \\
	=& \dots \\
	=& E(z)
\end{align*}

Thus we can say that

\begin{align*}
	y(t) =& F_{y^0y}(z) y^0(t) + F_{ey}(z) e(t) \\
		 =& z^{-k} y^0(t) + E(z) e(t) \\
         =& y^0(t-k) + E(z) e(t)
\end{align*}
 
which is the very simple closed-loop relationship between input and output in a MVC system.

% \begin{remark}
%     There are 2 sub-classes of control problems:
%     \begin{itemize}
%         \item When $y^0(t)$ is constant or step-wise (regulation problem)
%         \item When $y^0(t)$ is varying (tracking problem)
%     \end{itemize}

%     \begin{figure}[H]
%         \centering
%         \begin{minipage}[t]{0.48\textwidth}
%             \centering
%             \begin{tikzpicture}[
%                 node distance=2cm,auto,>=latex',
%                 declare function={
%                     f(\x) =  (\x < 0.5) * 1 +
%                              (\x >= 0.5) * (\x < 2) * 2 +
%                              (\x >= 2) * (\x < 3) * 3 +
%                              (\x >= 3) * (\x < 4) * 1.5 +
%                              (\x >= 4) * 1;
%                     f2(\x) = (f(\x-0.5) + (f(\x) - f(\x-0.5)) / 720 +
%                              f(\x-0.4) + (f(\x) - f(\x-0.4)) / 120 +
%                              f(\x-0.3) + (f(\x) - f(\x-0.3)) / 24 +
%                              f(\x-0.2) + (f(\x) - f(\x-0.2)) / 6 +
%                              f(\x-0.1) + (f(\x) - f(\x-0.1)) / 2) / 5 +
%                              rand/8;
%                 }
%             ]
%                 \draw[->] (0,0) -- (0,3) node[above] {$y(t)$};
%                 \draw[->] (0,0) -- (5,0) node[below] {$t$};
%                 \draw[domain=0:5,variable=\x,blue,samples=100] plot ({\x},{f(\x)});
%                 \draw[domain=0:5,variable=\x,red,smooth,samples=100] plot ({\x},{f2(\x)});
%             \end{tikzpicture}
%             \caption*{Regulation problem}
%         \end{minipage}
%         \begin{minipage}[t]{0.48\textwidth}
%             \centering
%             \begin{tikzpicture}[
%                 node distance=2cm,auto,>=latex',
%                 declare function={
%                     f(\x) =  (sin(\x*180)/2+sin(\x*270)/2)+1.5;
%                     f2(\x) = (f(\x-0.3) + (f(\x) - f(\x-0.3)) / 24 +
%                              f(\x-0.2) + (f(\x) - f(\x-0.2)) / 6 +
%                              f(\x-0.1) + (f(\x) - f(\x-0.1)) / 2) / 3 +
%                              rand/8;
%                 }
%             ]
%                 \draw[->] (0,0) -- (0,3) node[above] {$y(t)$};
%                 \draw[->] (0,0) -- (5,0) node[below] {$t$};
%                 \draw[domain=0:5,variable=\x,blue,samples=70] plot ({\x},{f(\x)});
%                 \draw[domain=0:5,variable=\x,red,smooth,samples=50] plot ({\x},{f2(\x)});
%             \end{tikzpicture}
%             \caption*{Tracking problem}
%         \end{minipage}
%     \end{figure}
% \end{remark}

% Bottom-up way of presenting M.V.C.

% \paragraph{Simplified problem \#1}
% \[
%     S: y(t) = ay(t-1) + b_0u(t-1) + b_1u(t-2) \qquad y(t) = \frac{b_0+b_1z^{-1}}{1-az^{-1}}u(t-1)
% \]

% We assume that $y^0(t)=\overline{y}^0$ (regulation problem) and the system is noise-free.
% \begin{itemize}
%     \item $b_0\ne 0$
%     \item Root of numerator must be inside the unit circle
% \end{itemize}

% To design the minimum variance controller we must minimize the performance index:
% \[
%     J = E\left[ (y(t) - y^0(t))^2 \right]
% \]
% There is no noise so we can remove the expected value
% \begin{align*}
%     J &= \left( y(t) - y^0(t) \right)^2 = \left( y(t) - \overline{y}^0 \right)^2 = \left( ay(t-1)+b_0u(t-1)+b_1u(t-2) - \overline{y}^0 \right)^2 = \\
%     &= \left( ay(t) + b_0u(t) + b_1u(t-1)-\overline{y}^0 \right)^2 \\
%     \frac{\partial J}{\partial u(t)} &= 2\left( ay(t)+b_0u(t)+b_1u(t-1)-\overline{y}^0 \right)\left(b_0\right)
% \end{align*}

% Why the derivative is just $b_0$? We are at present time $t$ and at time $t$ the control algorithm must take a decision on the value of $u(t)$.
% At time $t$, $y(t)$, $y(t-1)$, \dots, $u(t-1)$, $u(t-2)$, \dots{} are no longer variables but numbers.

% \[
%     \frac{\partial J}{\partial u(t)} = 0 \qquad ay(t)+b_0u(t)+b_1u(t-1)-\overline{y}^0 = 0 \\
% \]
% \[
%     u(t) = \left( \overline{y}^0  - ay(t)\right)\frac{1}{b_0+b_1z^{-1}}
% \]
% \begin{figure}[H]
%     \centering
%     \begin{tikzpicture}[node distance=2cm,auto,>=latex']
%         \node[sum] at (0,0) (sum) {};
%         \node[block] at (1.5,0) (b1) {$\frac{1}{b_0+b_1z^{-1}}$};
%         \node[block] at (5,0) (b2) {$z^{-1}\frac{b_0+b_1z^{-1}}{1-az^{-1}}$};
%         \node[block] at (3,-1.5) (b3) {$a$};

%         \draw[->] (sum) -- (b1);
%         \draw[->] (b1) -- (b2) node[pos=0.5] {$u(t)$};
%         \draw[->] (b3) -| (sum) node[pos=0.9] {-};
%         \draw[<-] (sum) -- ++(-1.5,0) node[left] {$\overline{y}^0$} node[pos=0.2] {+};
%         \draw[->] (b2) -- ++(2.5,0) node[right] {$y(t)$};
%         \draw[->] (6.5,0) |- (b3);
%     \end{tikzpicture}
% \end{figure}

% \paragraph{Simplified problem \#2}
% \[
%     S: y(t) = ay(t-1) + b_0u(t-1) + b_1u(t-2) + e(t) \qquad e(t) \sim WN(0, \lambda^2)
% \]

% The reference variable is $y^0(t)$ (tracking problem).

% The performance index is
% \[
%     J = E\left[ (y(t) - y^0(t))^2 \right]
% \]

% The fundamental trick to solve this problem is to re-write $y(t)$ as:
% \[
%     y(t) = \hat{y}(t|t-1) + \epsilon(t)
% \]

% Since $k=1$ we know that $\epsilon(t) = e(t)$, so $y(t) = \hat{y}(t|t-1)+e(t)$.
% \begin{align*}
%     J &= E\left[ \left(\hat{y}(t|t-1)+e(t) - y^0(t)\right)^2 \right] \\
%     &= E\left[   \left((\hat{y}(t|t-1)-y^0(t)) +e(t)\right)^2 \right] \\
%     &= E\left[ \left(\hat{y}(t|t-1)-y^0(t)\right)^2 \right] + E\left[e(t)^2\right] + \cancel{2E\left[e(t)\left( \hat{y}(t|t-1)-y^0(t) \right)\right]} \\
% \end{align*}

% Notice that
% \[
%     \argmin_{u(t)} \left\{ E\left[ \left(\hat{y}(t|t-1)-y^0(t)\right)^2 \right] + \lambda^2 \right\} = \argmin_{u(t)} \left\{ E\left[ \left(\hat{y}(t|t-1)-y^0(t)\right)^2 \right] \right\}
% \]
% The best result is when $\hat{y}(t|t-1)=y^0(t)$, we can force this relationship.

% Now we must compute the 1-step predictor of the system:
% \[
%     S: y(t) = \frac{b_0+b_1z^{-1}}{1-az^{-1}}u(t-1) + \frac{1}{1-az^{-1}}e(t)
% \]
% Note that this is an $ARMAX(1,0,1+1)=ARX(1,2)$.
% \[
%     k=1 \qquad B(z) = b_0+b_1z^{-1} \qquad A(z)=1-az^{-1} \qquad C(z) = 1
% \]

% General solution for 1-step prediction of ARMAX:
% \[
%     \hat{y}(t|t-1) = \frac{B(z)}{C(z)}u(t-1) + \frac{C(z)-A(z)}{C(z)}y(t)
% \]
% If we apply this formula we obtain:
% \[
%     \hat{y}(t|t-1) = \frac{b_0+b_1z^{-1}}{1}u(t-1) + \frac{1-1+az^{-1}}{1}y(t) = (b_0+b_1z^{-1})u(t-1)+ay(t-1)
% \]

% Now we can impose that $\hat{y}(t|t-1)=y^0(t)$
% \[
%     b_0u(t) + b_1u(t-1) + ay(t) = y^0(t+1) \qquad u(t) = \left( y^0(t+1) - ay(t) \right)\frac{1}{b_0+b_1z^{-1}}
% \]
% But we don't have $y^0(t+1)$, so we use $y^0(t)$.
% \[
%     u(t) = \left( y^0(t) - ay(t) \right)\frac{1}{b_0+b_1z^{-1}}
% \]
% \begin{figure}[H]
%     \centering
%     \begin{tikzpicture}[node distance=2cm,auto,>=latex']
%         \node[sum] at (0,0) (sum) {};
%         \node[block] at (1.5,0) (b1) {$\frac{1}{b_0+b_1z^{-1}}$};
%         \node[block] at (5,0) (b2) {$z^{-1}\frac{b_0+b_1z^{-1}}{1-az^{-1}}$};
%         \node[block] at (3,-1.5) (b3) {$a$};
%         \node[block] at (7,1.5) (b4) {$\frac{1}{1-az^{-1}}$};
%         \node[sum] at (7,0) (sum2) {};

%         \draw[->] (sum) -- (b1);
%         \draw[->] (b1) -- (b2) node[pos=0.5] {$u(t)$};
%         \draw[->] (b3) -| (sum) node[pos=0.9] {-};
%         \draw[<-] (sum) -- ++(-1.5,0) node[left] {$y^0(t)$} node[pos=0.2] {+};
%         \draw[->] (sum2) -- ++(1.5,0) node[right] {$y(t)$};
%         \draw[->] (b2) -- (sum2);
%         \draw[->] (b4) -- (sum2);
%         \draw[<-] (b4) -- ++(0,1) node[above] {$e(t)$};
%         \draw[->] (8,0) |- (b3);
%     \end{tikzpicture}
% \end{figure}

