%!TEX root = ../main.tex
\setcounter{chapter}{-1}
\chapter{Introduction}

\section{Prerequisites}
The \emph{Model Identification and Data Analysis - Part 2}\footnote{For more information about the MIDA course see \href{https://www4.ceda.polimi.it/manifesti/manifesti/controller/ManifestoPublic.do?EVN_DETTAGLIO_RIGA_MANIFESTO=evento&k_corso_la=481&k_indir=T2A&idItemOfferta=156912&idGruppo=4332&idRiga=271034&codDescr=051587&semestre=2&aa=2021&lang=EN&jaf_currentWFID=main}{here}.} course is a graduate level course of the MSc in Computer Science and Engineering held at Politecnico di Milano; hence, familiarity with basic concepts of computer science (algorithms and complexity), dynamical systems theory and a
mathematical maturity in linear algebra and probability theory are prerequisites. 

Furthermore, in order to understand the concepts of this course it is recommended to firstly follow the \emph{Model Identification and Data Analysis - Part 1}\footnote{MIDA1 lecture notes available \href{https://github.com/teobucci/mida/releases}{here}.} course.

\section{General topics of MIDA course}

\begin{itemize}
    \item Collect digitally data from real systems
    \item Build \acrfull{bb} (or \gls{gb}) models from data, with emphasis on
    \begin{itemize}
        \item Dynamic systems
        \item Control/automation-oriented applications
    \end{itemize}
    \item Purpose of modelling (area of machine leasing focusing on ``control'')
    \begin{itemize}
        \item Prediction
        \item Software-sensing
        \item Modelling for control design
    \end{itemize}
\end{itemize}

\subsection{Super summary of MIDA1}
The focus is on \emph{Time Series} (output-only systems) and \emph{input/output} (I/O) systems.

Models used in MIDA1:
\begin{itemize}
    \item \gls{arma} models for time series
    \item \gls{armax} models for I/O systems
\end{itemize}

\begin{figure}[H]
    \begin{minipage}[t]{0.4\textwidth}
	\centering
	\begin{tikzpicture}

		\node (input) at (0,0) {};
		\draw[-] (input.east) -- (1,0)
		    node[midway,above] {$u(t)$}
		    node[at end] (inputR) {};
		
		% blocks
		\node[block, right=0cm of inputR] (w1) {$\frac{C(z)}{A(z)}$};
		
		% connect block with input
		\draw[-stealth] (inputR.center)|-(w1.west);
		
		\draw[-stealth] (w1.east) -- ++(1,0)
		    node[midway, above] {$y(t)$}
		;
	\end{tikzpicture}
        \caption*{ARMA model}
    \end{minipage}
    \begin{minipage}[t]{0.4\textwidth}
        \centering
	\begin{tikzpicture}
		% place nodes
		\node [sum] (sum) at (0,0){};
		\node [block, left=1.5cm of sum] (wu) {$z^{-k}\frac{B(z)}{A(z)}$};
		\node [block,above left=0.7cm and 0.5cm of sum] (we) {$\frac{C(z)}{A(z)}$};

		% connect nodes
		\draw[stealth-] (wu.west) -- ++(-1,0) node[midway, above]{$u(t)$};
		\draw[stealth-] (we.west) -- ++(-1,0) node[midway, above]{$e(t)$};

		\draw[-stealth] (wu.east) -- (sum.west)
			node[midway, above] {$y_{1}(t)$}
			node[very near end, below] {$+$};
		\draw[-stealth] (we.east) -| (sum.north)
			node[midway, above] {$y_{2}(t)$}
			node[very near end, right] {$+$};

		\draw[-stealth] (sum.east) -- ++(1,0) node[midway, above] {$y(t)$};
	\end{tikzpicture}
        \caption*{\gls{armax} model}
    \end{minipage}
\end{figure}
We've used a \emph{\acrlong{bb} model identification} of the model, indicated as $\Mc(\theta)$ where $\theta$ is the parameter vector, which contains the coefficients of $A(z)$, $B(z)$, $C(z)$.

In particular we've seen the \textbf{\gls{pem}} method, a \emph{parametric approach} based on the minimization of the \emph{performance index} defined as:
\begin{definition}[\gls{pem} cost function]
    $J_{N}(\theta) = \frac{1}{N} \sum_{t=1}^N \left(y(t) - \hat{y}(t|t-1, \theta)\right)^2$
\end{definition}

which is the variance of the \emph{prediction error} made by the model. The optimal $\theta$ is $\hat{\theta}_N = \argmin_\theta J_{N}(\theta)$

\subsection{MIDA 2}

The focus is on \emph{Dynamical System} with a special eye on control (or feedback) application (more close to real applications than time series). We can divide the course in three main blocks: \emph{Advanced Model Identification} methods, \emph{Software-Sensing (SW)} and \emph{Minimum Variance Control}.\\
In particular we will see: 

\begin{itemize}
    \item Non-parametric (direct/constructive) \gls{bb} identification of I/O systems using \acrlong{ss} models
    \item Parametric identification for \gls{bb} I/O systems, with a frequency-domain approach
    \item Kalman-filter for SW-sensing using feedback on \gls{wb} models
    \item \gls{bb} methods for SW-sensing without feedback
    \item \gls{gb} system identification using Kalman-filter and using \emph{simulation-error methods} (SEM)
    \item  Minimum-Variance Control (MVC), design of optimal feedback controllers using the theory background of the MIDA course
    \item Recursive (online) implementation of algorithms for system identification
\end{itemize}

\section{Motivation example for the course: ABS (Anti-Lock Braking System)} \label{abs_ex}
ABS is an example of a control system since it follows this general scheme:
\begin{figure}[H]
    \centering
    	\begin{tikzpicture}
		% place nodes
		\node [block] (ctrl) at (0,0) {$\text{Control Algorithm}$};
		\node [block, right=2cm of ctrl] (system) {$\text{System}$};

		% connect nodes
		\draw [stealth-] (ctrl.west) -- ++(-2,0) node[midway,above] {$\bar{y}(t)$};
		\draw [-stealth] (ctrl.east) -- (system.west) node[midway,above] {$u(t)$};
		\draw [-stealth] (system.east) -- ++(2,0)  
			node[midway,above] {$y(t)$}
			node[midway] (yt) {}; ;
		\draw [-] (yt.center) --+(0,-1)
			node[at end] (fb) {};
		\draw[-stealth] (fb.center)-|(ctrl.south);
	\end{tikzpicture}
	\caption*{Control System Scheme}
\end{figure}
    
where $\bar{y}(t)$ is the reference value of $y(t)$.\\

We define the \emph{slip} of the wheel as $\lambda(t) = \frac{v(t)-\omega(t) r}{v(t)}$, where $v(t)$ is the horizontal velocity of the car, $\omega(t)$ is the angular velocity of the wheel and $r$ is the radius of the wheel.\\

During a brake $0 \le \lambda (t) \le 1$ (from free rolling wheel (i.e. $v(t) = \omega (t) r$) to locked wheel (i.e. $\omega(t) = 0)$).\\
The curve of $F_{x}$, the braking force, is as showed below:
\vspace{-3cm}
\begin{figure}[h!]
    \centering
    \resizebox{8cm}{!}{%
    \begin{tikzpicture}[node distance=2.5cm,auto,>=latex',scale=1.5]
        \draw[->] (-0.5,0) -- (2.5,0) node[right] {$\lambda$};
        \draw[->] (0,-0.5) -- (0,2.3) node[left] {$F_x$};
        \begin{scope}[scale=2]
            \draw[dotted] (0.23, 1) -- (0.23, 0) node[below] {$\bar{\lambda}$};
            \draw[dotted] (1, 0.75) -- (1, 0) node[below] {$1$};
            \node[red,rotate=-23] at (0.65,1) {unstable};
            \node[green!60!black,rotate=77] at (0.13,0.5) {stable};
            \begin{scope}
                \clip (0,0) rectangle (0.23, 2);
                \draw [green!60!black,line width=0.4mm] plot [smooth] coordinates {(0,0) (0.2, 1) (1, 0.75)};
            \end{scope}
            \begin{scope}
                \clip (0.23, 0) rectangle (1, 2);
                \draw [red,line width=0.4mm] plot [smooth] coordinates {(0,0) (0.2, 1) (1, 0.75)};
            \end{scope}
        \end{scope}
    \end{tikzpicture}
    }
    \caption*{Relation between $\lambda$ and the braking force.}
\end{figure}


In the case of ABS, $u(t) = x(t)$ which is the voltage of the electric braking motor (control variable), $y(t) = \lambda (t)$ (controlled variable) and $\bar{y}(t) = \bar{\lambda} (t)$, which is the maximum braking point that we want to reach during an emergency brake.

The problem can be divided into subproblems:
\begin{itemize}
    \item Model of the system
    \item SW-estimation of $\lambda$ since $v$ is \textbf{not} directly measurable, so $\lambda$ cannot be computed
    \item Design of the ABS control algorithm
\end{itemize}

Because of that measurement problem we have to build a \acrlong{bb} model from data.\\

Why \acrlong{bb} modelling?
The control variable $x$ (the voltage to the actuator) controls a complex systems from the actuator to $\lambda$.
The system can be seen as a chain of components:
\begin{itemize}
    \item Current dynamics and electric motor
    \item Position dynamics of the actuator
    \item Dynamics of the hydraulic circuit of the braking system
    \item Tire dynamics
    \item Wheel rotational dynamics
    \item Vehicle full dynamics
\end{itemize}

It's simply deducible that such system is really difficult to model.

\begin{recall}[\acrfull{wb} vs. \acrfull{bb}]
    \hfill \break
    \begin{itemize}
	   \item \gls{wb} modelling: write the physical equations from \emph{first principles}.
	   \item \gls{bb} modelling: experiment $\rightarrow$ collect data $\rightarrow$ build model.
        Using only I/O measured data we can \emph{learn} a mathematical model of the I/O behavior of the system.
    \end{itemize}
\end{recall}

In order to estimate the variable $v(t)$ (and so also $\lambda(t)$) we use an SW-sensing algorithm which takes as inputs other measurable variables (e.g. angular velocities of all the wheels). \\
Thus, the scheme of a general control system become something like this: 

\begin{figure}[H]
    \centering
    	\begin{tikzpicture}
		% place nodes
		\node [block] (ctrl) at (0,0) {$\text{Control Algorithm}$};
		\node [block, right=2cm of ctrl] (system) {$\text{System}$};
		\node [block, below=1.5cm of ctrl] (sw) {$\text{SW-sensing Algorithm}$};

		% connect nodes
		\draw [stealth-] (ctrl.west) -- ++(-2,0) node[midway,above] {$\bar{y}(t)$};
		\draw [-stealth] (ctrl.east) -- (system.west) node[midway,above] {$u(t)$};
		\draw [-stealth] (system.east) -- ++(2,0)  node[midway,above] {$y(t)$};
		\draw [-stealth] (system.south) |- (sw.east)  node[midway, right] {$\Phi(t)$};
		\draw[-stealth] (sw.north)-|(ctrl.south) node[near end, right] {$\hat{y}(t)$};
	\end{tikzpicture}
    \vspace{5pt}
	\caption*{Control System Scheme with SW-sensing}
\end{figure}

where $\Phi(t)$ are the available (measurable) variables of the system and $\hat{y}(t)$ is the estimation of the SW-sensing algorithm of $y(t)$

\chapter{\acrlong{bb} non-parametric identification of I/O systems using \acrlong{ss} models in time domain}

\vspace{-12pt}
\begin{figure}[H]
    \centering
    	\begin{tikzpicture}
    		% place nodes
		\node [block] (sys) at (0,0) {System};
	
		% connect nodes
		\draw [stealth-] (sys.west) -- ++(-1,0) node[midway,below] {$u(t)$};
		\draw [-stealth] (sys.east) -- ++(1,0) node[midway,below] {$y(t)$};
         	\draw[dashed, stealth-] (sys.north) -- ++(0,1) node[midway,right] {$d(t)$ (not measured disturbance)};
    \end{tikzpicture}
\end{figure}


\begin{recall}[General path of a Parametric Identification Methods]
\hfill \break
    \begin{enumerate}
        \item Collect data: inputs: $\left\{u(1), u(2), \ldots, u(N)\right\}$, outputs: $\left\{y(1), y(2), \ldots, y(N)\right\}$
        \item Select \textbf{a-priori} a class/family of \textbf{parametric models}: $\Mc(\theta)$
        \item Select \textbf{a-priori} a performance index $J(\theta)$ (which gives an order to the quality of the models)
        \item Optimization step (minimize $J(\theta)$ w.r.t $\theta$): $\hat{\theta}_N = \argmin_\theta J(\theta)$ $\rightarrow$ optimal model $\Mc(\hat{\theta}_N)$ characterized by the \textbf{optimal parameters} $\hat{\theta}_N$
    \end{enumerate}
    
    Note:    $J(\theta): \RR^{n_\theta} \rightarrow \RR^+$ (where $n_\theta$ is the order of the model).
    
    Note: $\Mc(\theta_1)$ can be sorted, that is $\Mc(\theta)$ is better than $\Mc(\theta_2)$ if $J(\theta_1) < J(\theta_2)$.

\end{recall}

In this chapter we are presenting a totally different system identification approach, the \textbf{non-parametric} one, which means:
\begin{itemize}
    \item No a-priori model-class selection
    \item No performance index definition
    \item No optimization task
\end{itemize}

Before entering in the system identification algorithm we need to recall the three main mathematical representation of \textbf{Discrete-time Dynamic Linear Systems} which are characterized by their internal variables called \emph{states} and indicated with $x(t)$.


\section{Representations}

\subsection{Representation \#1: \acrfull{ss}}

\[
\Sys: 
\begin{cases}
    x(t+1) = F x(t) + G u(t) & \qquad \text{state equations} \\
    y(t+1) = H x(t) + D u(t) & \qquad \text{output equations}
\end{cases}
\]

where $F$, $G$, $H$ and $D$ are matrices defined a follows:
\begin{align*}
    F = \begin{bmatrix}
        \\
        n \times n \\
        \text{state matrix} \\ \\
    \end{bmatrix}
    &
    \qquad
    G = \begin{bmatrix}
        \\
        \\
        n \times 1 \\
        \text{input} \\
        \text{matrix} \\ \\
    \end{bmatrix}
    \\ \\
    H = \begin{bmatrix}
        1 \times n \;\;\; \text{output matrix}
    \end{bmatrix}
    &
    \qquad
    D = \begin{bmatrix}
        1 \times 1 \;\;\; \text{i/o matrix}
    \end{bmatrix}
\end{align*}

In this case we have 1 input and 1 output (i.e. \emph{SISO} system), but those \emph{difference equations} can be extended for multiple inputs and outputs systems. Usually $D=0$ since we can say that the majority of real systems have this property. 

\begin{definition} [Strictly-proper system]
A system is called \emph{strictly-proper} when the output of the system doesn't directly depend on the input (i.e. $D=0$). 
\end{definition}

\begin{obs}
$n$ is the \emph{order} of the system.
\end{obs}

\begin{example}[SISO system of order $n=2$]
    \[
    \Sys: 
        \begin{cases}
            x_1(t+1) = \frac{1}{2} x_1(t) + 2u(t) \\
            x_2(t+1) = x_1(t) + 2x_2(t) + u(t) \\
            y(t) = \frac{1}{4}x_1(t) + \frac{1}{2}x_2(t)
        \end{cases}
    \]
    In this case $n=2$, $x(t) = \begin{bmatrix}
        x_1(t) \\
        x_2(t)
    \end{bmatrix}$, one input $u(t)$ and one output $y(t)$.

    \begin{align*}
        F = \begin{bmatrix}
            \frac{1}{2} & 0 \\
            1 & 2
        \end{bmatrix}
        & \qquad
        G = \begin{bmatrix}
            2 \\ 1
        \end{bmatrix}
        \\
        H = \begin{bmatrix}
            \frac{1}{4} & \frac{1}{2}
        \end{bmatrix}
        & \qquad
        D = 0
    \end{align*}
\end{example}


\begin{remark}[\acrlong{ss} representation is not unique]
    $F_1 = TFT^{-1}$, $G_1 = TG$, $H_1 = HT^{-1}$, $D_1 = D$ for any invertible $(n\times n)$ matrix $T$. The system $\{F, G, H, D\}$ is equivalent to $\{F_1, G_1, H_1, D_1\}$.
\end{remark}