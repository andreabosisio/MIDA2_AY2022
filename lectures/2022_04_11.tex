%!TEX root = ../main.tex
\subsection{Summary of transformations}
Notice that the \gls{ir} representation is very easy to obtain experimentally, since we only need to measure the system response to the impulse signal.
However, given the \gls{ir} representation, it is difficult to get to the other representations, since the transformation from \gls{ir} to \gls{tf} is only theoretical.
Moving from the \gls{ir} to the \gls{ss} representation is the key task of the \emph{Subspace-based \acrlong{ss} System Identification}, also known as \emph{4SID method}.

\begin{figure}[H]
    \centering
    \begin{tikzpicture}[node distance=2.5cm,auto,>=latex']
        \node (n1) [draw, circle, align=center]{\#1\\\acrshort{ss}};
        \node (n2) [draw, circle, align=center, below of=n1, xshift=-1.8cm] {\#2\\\gls{tf}};
        \node (n3) [draw, circle, align=center, below of=n1, xshift= 1.8cm] {\#3\\\gls{ir}};
        
        %legend
        \draw[-stealth, line width=0.4mm] (3,0) -- (5, 0)
            node[midway, above] {useful and feasible};
        \draw[-stealth] (3,-0.8) -- (5, -0.8)
            node[midway, above] {feasible};
        \draw[-stealth, dashed] (3,-1.6) -- (5, -1.6)
        node[midway, above] {not feasible};

        \draw[->, line width=0.4mm] (n1) edge[bend right] (n2);
        \draw[->, line width=0.4mm] (n2) edge[bend right] (n3);
        \draw[->, line width=0.4mm] (n1) edge[bend left] (n3);
        \draw[->] (n2) edge[bend right=20] (n1);
        \draw[->, dashed] (n3) edge[bend right=20] (n2);
        \draw[->, line width=0.4mm, red] (n3) edge[bend left=20] node {?} (n1);
    \end{tikzpicture}
    \caption*{Transformations between representations in practice}
\end{figure}


Before moving to the topic of \emph{4SID method}, a recall of the \emph{observability} and \emph{controllability} properties for a linear dynamic system is needed.

\section{Fundamental concepts of Observability and Controllability}

\[
    \Sc: 
    \begin{cases}
        x(t+1) = Fx(t) + Gu(t) \\
        y(t) = Hx(t)
    \end{cases}
\]

\begin{defn}[Fully Observable]
The system is fully observable (from the output) if and only if the \textbf{observability matrix} $O$ is full rank:
\[
    O = \begin{bmatrix}
        H \\
        HF \\
        \vdots \\
        HF^{n-1}
    \end{bmatrix}
    \qquad
    \rank (O) = n
\]
where $n$ is the order of the system.
\end{defn}

\begin{rem}[Observability]
Observability is a property of the system: by observing the output $y(t)$ it's possible to observe the state $x(t)$.\\
\textbf{Note}: It refers only to state and output (i.e. $F$ and $H$).
\end{rem}

\begin{defn}[Fully Controllable]
The system is fully controllable (from the input) if and only if the \textbf{controllability matrix} $R$ is full rank:
\[
    R = \begin{bmatrix}
        G & FG & \cdots & F^{n-1}G
    \end{bmatrix}
    \qquad
    \rank (R) = n
\]
where, again, $n$ is the order of the system.\\
$R$ is also called \emph{reachability} matrix.
\end{defn}

\begin{rem}[Controllability]
Controllability is a property of the system: by driving (or move) the input $u(t)$ it's possible to control the state $x(t)$.\\
\textbf{Note}: It refers only to input and state (i.e. $F$ and $G$).\\
\textbf{Note}: Controllability and Reachability are synonyms.
\end{rem}

\begin{exa}[SISO system of order $n=2$]
    \begin{align*}
    \Sc: 
        \begin{cases}
            x_1(t+1) = \frac{1}{2} x_1(t) + u(t) \\
            x_2(t+1) = \frac{1}{3}x_2(t) \\
            y(t) = \frac{1}{4}x_1(t)
        \end{cases}
        \qquad
        F = \begin{bmatrix}
            \frac{1}{2} & 0 \\
            0 & \frac{1}{3}
        \end{bmatrix}
        \qquad
        H = \begin{bmatrix}
            \frac{1}{4} & 0
        \end{bmatrix}
        \qquad
        G = \begin{bmatrix}
            1 \\
            0
        \end{bmatrix}
    \end{align*}

    \[
        O = \begin{bmatrix}
            H \\
            HF
        \end{bmatrix} = \begin{bmatrix}
            \frac{1}{4} & 0 \\
            \frac{1}{8} & 0
        \end{bmatrix}
        \qquad
        \rank (O) = 1 < n = 2
        \quad\implies\quad \text{not fully observable}
    \]

    \[
        R = \begin{bmatrix}
            G & FG
        \end{bmatrix} = \begin{bmatrix}
            1 & \frac{1}{2} \\
            0 & 0
        \end{bmatrix}
        \qquad
        \rank (R) = 1 < n = 2
        \quad\implies\quad \text{not fully controllable}
    \]
    If $\Sc$ becomes 
    \begin{align*}
    \mathcal{S'}: 
        \begin{cases}
            x_1(t+1) = \frac{1}{2} x_1(t) + u(t) + \color{blue}\frac{1}{6} x_2(t)\\
            x_2(t+1) = \frac{1}{3}x_2(t) \\
            y(t) = \frac{1}{4}x_1(t)
        \end{cases}
        \qquad
        \text{only $F$ changes and becomes}\quad
        F = \begin{bmatrix}
            \frac{1}{2} & \color{blue}\frac{1}{6} \\
            0 & \frac{1}{3}
        \end{bmatrix}
    \end{align*} 
    And     
    \[
        O = \begin{bmatrix}
            H \\
            HF
        \end{bmatrix} = \begin{bmatrix}
            \frac{1}{4} & 0 \\
            \frac{1}{8} & \color{blue}\frac{1}{24}
        \end{bmatrix}
        \qquad
        \rank (O) = 2 = n         \quad\implies\quad \text{fully observable}
    \]
    
    Observability and Controllability can also be checked graphically by building the block scheme 
    
      \begin{figure}[H]
        \centering
        \begin{tikzpicture}[node distance=1.5cm,auto,>=latex']
            \node[block, align=center] (z1) {$z^{-1}$};
            \node[block, align=center, below of=z1] (12) {$\frac{1}{2}$};
            \node[block, align=center, right=2cm of z1] (14) {$\frac{1}{4}$};
            \node[sum, align=center, left=2cm of z1] (sum1) {};
            
            \node[coordinate] at (0,-2.5cm) (block2) {};
            \node[block, align=center, below of=block2] (z1_2) {$z^{-1}$};
            \node[block, align=center, below of=z1_2] (13) {$\frac{1}{3}$};
            \node[sum, align=center, left=2cm of z1_2] (sum2) {};
            
            \node[block, align=center, below right = 2cm and 0.5cm of sum1, blue] (16) {$\frac{1}{6}$};
            
            \draw[->] (sum1) -- (z1)
                node[midway, above] {$x_1(t+1)$};;
            \draw[->] (z1) -- (14)
                node[midway,above] (x1t) {$x_1(t)$};
            \draw[->] (x1t.south) |- (12.east);
            \draw[->] (12) -| (sum1)
                node[very near end, right] {$+$};
            \draw[<-] (sum1) --++ (-2,0)
                node[near start,above] {$+$}
                node[very near end, above] (ut) {$u(t)$};
            \draw[->] (14) --++ (2,0)
                node[near end,above] (yt) {$y(t)$};
                
            \draw[->] (sum2) -- (z1_2)
                node[midway, above] {$x_2(t+1)$};
            \draw[->] (13) -| (sum2)
                node[very near end, right] {$+$};
            \draw[->] (z1_2) --++ (2,0)
                node[midway,above] (x2t) {$x_2(t)$};
            \draw[->] (x2t.south) |- (13.east);    

            \draw[->, blue] (x2t.west)+(0,-0.25cm) |- (16.east);
            \draw[->, blue] (16.west) -| (sum1.west)
                node[very near end, left] {$+$};
                
            \draw[black, dashed] ([yshift=3mm]14.north)-|(ut.east)|-([yshift=-3mm]13.south)-|(yt.west)|-([yshift=3mm]14.north);
        \end{tikzpicture}
    \end{figure}
    From this block scheme it can be seen that $x_1$ is \emph{directly} observable and reachable; $x_2$ becomes only \emph{indirectly} observable through $x_1(t)$ with the introduction of the blue block that correspond to the term $\color{blue}\frac{1}{6} x_2(t)$ in $\mathcal{S'}$.
\end{exa}

\begin{exa}[SISO system of order $n=2$]
    \begin{align*}
    \Sc: 
        \begin{cases}
            x_1(t+1) = \frac{1}{2} x_1(t)\\
            x_2(t+1) = \frac{1}{3}x_2(t) + u(t)\\
            y(t) = \frac{1}{4}x_1(t)
        \end{cases}
        \qquad
        F = \begin{bmatrix}
            \frac{1}{2} & 0\\
            0 & \frac{1}{3}
        \end{bmatrix}
        \qquad
        H = \begin{bmatrix}
            \frac{1}{4} & 0
        \end{bmatrix}
        \qquad
        G = \begin{bmatrix}
            0 \\
            1
        \end{bmatrix}
    \end{align*}

    \[
        O = \begin{bmatrix}
            H \\
            HF
        \end{bmatrix} = \begin{bmatrix}
            \frac{1}{4} & 0 \\
            \frac{1}{8} & 0
        \end{bmatrix}
        \qquad
        \rank (O) = 1 < n = 2
        \quad\implies\quad \text{not fully observable}
    \]

    \[
        R = \begin{bmatrix}
            G & FG
        \end{bmatrix} = \begin{bmatrix}
            0 & 0 \\
            1 & \frac{1}{3}
        \end{bmatrix}
        \qquad
        \rank (R) = 1 < n = 2
        \quad\implies\quad \text{not fully controllable}
    \]
    
    If $\Sc$ becomes 
    \begin{align*}
    \mathcal{S'}: 
        \begin{cases}
            x_1(t+1) = \frac{1}{2} x_1(t) + \color{blue} \frac{1}{6} x_2(t)\\
            x_2(t+1) = \frac{1}{3}x_2(t) + u(t)\\
            y(t) = \frac{1}{4}x_1(t)
        \end{cases}
        \qquad
        \text{only $F$ change and becomes}\quad
        F = \begin{bmatrix}
            \frac{1}{2} & \color{blue} \frac{1}{6} \\
            0 & \frac{1}{3}
        \end{bmatrix}
    \end{align*} 
    And  
    
    \[
        O = \begin{bmatrix}
            H \\
            HF
        \end{bmatrix} = \begin{bmatrix}
            \frac{1}{4} & 0 \\
            \frac{1}{8} & \color{blue} \frac{1}{24}
        \end{bmatrix}
        \qquad
        \rank (O) = 2 = n = 2
        \quad\implies\quad \text{fully observable}
    \]
    
    \[
        R = \begin{bmatrix}
            G & FG
        \end{bmatrix} = \begin{bmatrix}
            0 & \color{blue} \frac{1}{6} \\
            1 & \frac{1}{3}
        \end{bmatrix}
        \qquad
        \rank (R) = 2 = n = 2
        \quad\implies\quad \text{fully controllable}
    \]
    
    Again, this properties can be also check using the block scheme
    
    \begin{figure}[H]
        \centering
        \begin{tikzpicture}[node distance=1.5cm,auto,>=latex']
            \node[block, align=center] (z1) {$z^{-1}$};
            \node[block, align=center, below of=z1] (12) {$\frac{1}{2}$};
            \node[block, align=center, right=2cm of z1] (14) {$\frac{1}{4}$};
            \node[sum, align=center, left=2cm of z1] (sum1) {};
            
            \node[coordinate] at (0,-2.5cm) (block2) {};
            \node[block, align=center, below of=block2] (z1_2) {$z^{-1}$};
            \node[block, align=center, below of=z1_2] (13) {$\frac{1}{3}$};
            \node[sum, align=center, left=2cm of z1_2] (sum2) {};
            
            \node[block, align=center, below right = 2cm and 0.5cm of sum1, blue] (16) {$\frac{1}{6}$};
            
            \draw[->] (sum1) -- (z1)
                node[midway, above] {$x_1(t+1)$};;
            \draw[->] (z1) -- (14)
                node[midway,above] (x1t) {$x_1(t)$};
            \draw[->] (x1t.south) |- (12.east);
            \draw[->] (12) -| (sum1)
                node[very near end, right] {$+$};
            \draw[<-] (sum2) --++ (-2,0)
                node[near start,above] {$+$}
                node[very near end, above] (ut) {$u(t)$};
            \draw[->] (14) --++ (2,0)
                node[near end,above] (yt) {$y(t)$};
                
            \draw[->] (sum2) -- (z1_2)
                node[midway, above] {$x_2(t+1)$};
            \draw[->] (13) -| (sum2)
                node[very near end, right] {$+$};
            \draw[->] (z1_2) --++ (2,0)
                node[midway,above] (x2t) {$x_2(t)$};
            \draw[->] (x2t.south) |- (13.east);    

            \draw[->, blue] (x2t.west)+(0,-0.25cm) |- (16.east);
            \draw[->, blue] (16.west) -| (sum1.west)
                node[very near end, left] {$+$};
                
            \draw[black, dashed] ([yshift=3mm]14.north)-|(ut.east)|-([yshift=-3mm]13.south)-|(yt.west)|-([yshift=3mm]14.north);
        \end{tikzpicture}
    \end{figure}
    
    It can be noticed that $x_1(t)$ is \emph{directly} observable but not reachable and $x_2(t)$ is \emph{directly} controllable. In $\mathcal{S}'$, with the introduction of the same blue block, the system becomes fully observable and controllable; in particular, $x_2(t)$ becomes also \emph{indirectly} observable and $x_1(t)$ becomes also \emph{indirectly} controllable.
    
\end{exa}

\begin{rem}[4 sub-systems]
\hfill \break
    Any \acrlong{ss} system can be \textbf{internally} seen as 4 sub-systems as follows:

    \begin{figure}[H]
        \centering
        \begin{tikzpicture}[node distance=1.5cm,auto,>=latex']
            \node[block, align=center] (noc) {NO\\C};
            \node[block, align=center, below of=noc] (onc) {O\\NC};
            \node[block, align=center, below of=onc] (oc) {O\\C};
            \node[block, align=center, below of=oc] (nonc) {NO\\NC};
            
            %legend
            \node[coordinate] at (5cm, 0cm) (lgnd) {};
            \node[align=center, right of=lgnd]{O: observable\\C: controllable\\NX: not X};

            \node at (-5cm,-1.5cm) (u) {$u$};
            \node[coordinate] at (-3cm,-1.5cm) (input) {};
            \node at (5cm,-2.5cm) (y) {$y$};
            \node[coordinate] at (3cm,-2.5cm) (output) {};
            \node[right of=noc] (noc_out) {};
            \node[right of=nonc] (nonc_out) {};
            \node[left of=onc] (onc_in) {};

            \draw[-Rays] (noc) -- (noc_out);
            \draw[-Rays] (nonc) -- (nonc_out);
            \draw[Rays-] (onc_in) -- (onc);
            \draw (input) edge[->, bend left=10] (noc);
            \draw[line width=0.5mm] (input) edge[->, bend right=10] (oc);
            \draw[line width=0.5mm] (u) edge (input);
            \draw (onc) edge[bend left=10] (output);
            \draw[line width=0.5mm] (oc) edge[bend right=10] (output);
            \draw[line width=0.5mm] (output) edge[->] (y);

            \draw[draw=black] (-4cm,-5.5cm) rectangle ++(8cm,6.5cm);
        \end{tikzpicture}
    \end{figure}

    Which \textbf{externally} is equivalent to a systems like this:
    \vspace{-5pt}
    \begin{figure}[H]
        \centering
        \begin{tikzpicture}[node distance=1.5cm,auto,>=latex']
            \node[block, align=center, blue] (oc) {O\\C};
            \node[left of=oc] (in) {$u$};
            \node[right of=oc] (out) {$y$};
            
            % \node[align=center, below=0.1cm of oc, blue] {$\uparrow$\\$W(z)$};

            \node[below=0.7cm of oc.north, blue] {$\underbrace{\phantom{OoC}}_{W(z)}$};

            \draw[->] (in) -- (oc);
            \draw[->] (oc) -- (out);
        \end{tikzpicture}
    \end{figure}
    
    \vspace{-15pt}
    
    Hence, this is a graphic proof of the fact that the I/O representation of a system through the \acrlong{tf} can only represent the \emph{observable} and \emph{controllable} part of the system; all the other sub-systems remains \textbf{hidden} with this representation. For this reason, the \gls{ss} representation is "more complete" than the \gls{tf} one when the system is not fully observable or not fully controllable.
\end{rem}

Another final definition is needed before presenting the 4SID algorithm.

\begin{defn}[Hankel matrix of order n]
    Starting from the \gls{ir} $\omega(t) = \{\omega(1), \omega(2), \ldots, \omega(N)\}$ we can build the Hankel matrix of order $n$ as follows:

\[
    H_n = \begin{bmatrix}
        \omega(1) & \omega(2) & \omega(3) & \cdots & \omega(n) \\
        \omega(2) & \omega(3) & \omega(4) & \cdots & \omega(n+1) \\
        \omega(3) & \omega(4) & \omega(5) & \cdots & \omega(n+2) \\
        \vdots    & \vdots    & \vdots    & \ddots & \vdots \\
        \omega(n) & \omega(n+1) & \omega(n+2) & \ldots & \omega(2n-1)
    \end{bmatrix}
\]

    \textbf{Note}: it is a square matrix of size $n\times n$.

    \textbf{Note}: we need the IR up to time $2n-1$.

    \textbf{Note}: it starts from $\omega(1)$ and not from $\omega(0)$ (since, for strictly-proper systems, $\omega(0)=0$).

    \textbf{Note}: the anti-diagonals all have the same element repeated.
    
    We know that $\omega(t) = \begin{cases}
    0 &\quad \text{if } t = 0 \\
    HF^{t-1}G &\quad \text{if } t > 0
    \end{cases}$ ,\qquad therefore $H_n$ can be rewritten as

    \[
        H_n = \begin{bmatrix}
            HG     & HFG    & HF^2G  & \cdots & HF^{n-1}G \\
            \vdots & \ddots &        &        & \vdots \\
            \vdots &        & \ddots &        & \vdots \\
            \vdots &        &        & \ddots & \vdots \\
            HF^{n-1}G & \cdots & \cdots & \cdots & HF^{2n-2}G
        \end{bmatrix} = \begin{bmatrix}
            H \\
            HF \\
            \vdots \\
            HF^{n-1}
        \end{bmatrix} \cdot \begin{bmatrix}
            G & FG & \cdots & F^{n-1}G
        \end{bmatrix} = O \cdot R
    \]
    
    % \[
    %     \Rightarrow H_n = O \cdot R
    % \]
    
    Therefore, $H_n = O \cdot R$, where $O$ is the observability matrix and $R$ is the reachability matrix.
\end{defn}




\section{Subspace-based \acrlong{ss} System Identification (4SID)}
The original 4SID method starts from the measurement of the system output in a very simple experiment, that is the \emph{impulse experiment}.
\vspace{-7pt}
\begin{figure}[H]
    \begin{minipage}[t]{0.5\textwidth}
        \centering
        \begin{tikzpicture}[node distance=2.5cm,auto,>=latex']
            \draw[->] (-0.5,0) -- (3,0) node[right] {$t$};
            \draw[->] (0,-0.5) -- (0,2) node[left] {$u(t)$};
            \draw[domain=-0.3:0,smooth,variable=\x,red] plot ({\x},{0});
            \draw[domain=0:0.2,smooth,variable=\x,red] plot ({\x},{1.5});
            \draw[domain=0.2:2.5,smooth,variable=\x,red] plot ({\x},{0});
            \draw[mark=*, mark options={fill=blue},blue,samples=1,domain=0:0.00001,only marks,variable=\x] plot ({\x},{1.5});
            \draw[mark=*, mark options={fill=blue},blue,samples=10,domain=0.25:2.5,only marks,variable=\x] plot ({\x},{0});
        \end{tikzpicture}
        \caption*{Impulse in input}
    \end{minipage}
    \begin{minipage}[t]{0.5\textwidth}
        \centering
        \begin{tikzpicture}[node distance=2.5cm,auto,>=latex']
            \draw[->] (-0.5,0) -- (3,0) node[right] {$t$};
            \draw[->] (0,-0.5) -- (0,2) node[left] {$y(t)$};
            \draw[domain=0:2.5,smooth,variable=\x,red] plot ({\x},{1.5*sin(\x*180/3.14*5)*e^(-\x)});
            \draw[mark=*, mark options={fill=blue},blue,samples=15,domain=0:2.5,only marks,variable=\x] plot ({\x},{1.5*sin(\x*180/3.14*5)*e^(-\x)});
        \end{tikzpicture}
        \caption*{\gls{ir} in output}
    \end{minipage}
    \vspace{-15pt}
    %\caption*{Impulse experiment}
\end{figure}

The fundamental idea of this experiment is that it is very simple to measure the output of system given an impulse signal as the input.
Since the goal of the method is to identify an \gls{ss} model $\left\{ \hat{F}, \hat{G}, \hat{H} \right\}$ starting from the \gls{ir} $\omega(t) = \left\{ \omega(0), \omega(1), \omega(2), \cdots \right\}$, it can be said that it is a \acrlong{bb} system identification method (we need only data and knowledge of the system isn't required).

We will see the solution of the problem with two different assumptions:
\begin{description}
    \item [Basic problem] \gls{ir} measurement is assumed to be \textbf{noise-free}. Easier and not realistic problem. Described in \ref{sec:4SID-NF}.
    \item [Real problem] \gls{ir} is measured with noise, that is:
    \[ \widetilde{\omega}(t) = \omega(t) + \eta(t) \quad t = 0, 1,\dots, N \] where
        $\widetilde{\omega}(t)$ is the measured noisy \gls{ir},
        $\omega(t)$ is the ``true'' noise-free \gls{ir} and
        $\eta(t)$ is the measurement noise (e.g. \gls{wn}). Described in \ref{sec:4SID-N}.
\end{description}

\begin{rem}
    We will see in detail only the original version of 4SID, that is that the experiment is an impulse-experiment.
    However 4SID can be extended to any generic input signal $\left\{ u(1), u(2), \cdots, u(N) \right\}$ that is sufficiently exciting.
\end{rem}

\begin{rem}[Unstable System]
    In case of an unstable system the measurements must be collected in a closed-loop experiment.
    Indeed, if the experiment was open-loop, the experiment would be unfeasible.

    \begin{figure}[H]
        \centering
        \begin{tikzpicture}[node distance=2.5cm,auto,>=latex']
            \node [block,align=center] (stab) {stability\\controller};
            \node [sum] (sum) [right of=stab, node distance=2cm] {};
            \node [block] (sys) [right of=sum, node distance=2cm]{system};
            \node [coordinate] (split) [right of=sys, node distance=1cm]{};
            \node [coordinate] (end) [right of=split, node distance=1cm]{};
            \node (in) [above of=sum, node distance=1.5cm] {};
            \node [coordinate] (mid) [below of=sum, node distance=1cm] {};

            \draw[->] (in) edge node[pos=0.2, align=left] {experimental\\excitation input} (sum);
            \draw[->] (stab) edge node {} (sum);
            \draw[->] (sum) edge node {} (sys);
            \draw (sys) edge (split);
            \draw[->] (split) -- (end);
            \draw (split) |- (mid);
            \draw[->] (mid) -| (stab);
        \end{tikzpicture}
        %\vspace{3pt}
        %\caption*{Closed Loop System}
    \end{figure}
\end{rem}

\section{4SID procedure (noise-free)} \label{sec:4SID-NF}

\paragraph{Step 1} \label{4SID-NF:step1} Build the Hankel matrix in increasing order and each time compute the rank of the matrix.

\[
    H_1 = \begin{bmatrix}
        \omega(1)
    \end{bmatrix}
    \qquad
    H_2 = \begin{bmatrix}
        \omega(1) & \omega(2) \\
        \omega(2) & \omega(3)
    \end{bmatrix}
    \qquad
    H_3 = \ldots
    \qquad
    \cdots
    \qquad
    H_n = \ldots
\]

Suppose that $\rank (H_i) = i \quad \forall i \in \{1, \dots, n\}$ \quad and $\rank (H_{n+1}) = n$.\\
If this happens, it means that we found the first Hankel matrix which in not full rank and it also means that we have estimated (found) the order of the system.

\paragraph{Step 2} Take $H_{n+1}$ (\textbf{recall}: $\rank (H_{n+1}) = n$) and factorize it in two rectangular matrices of size $(n+1) \times n$ and $n \times (n+1)$.

\[
    H_{n+1} = \begin{bmatrix}
        \text{extended} \\
        \text{observability} \\
        \text{matrix:} \\
        O_{n+1}
    \end{bmatrix} \cdot \begin{bmatrix}
        \text{extended} \\
        \text{controllability} \\
        \text{matrix:} \\
        R_{n+1}
    \end{bmatrix}
\]

where $O_{n+1} = \begin{bmatrix}
    \color{blue}H \\ HF \\ \vdots \\ HF^n
\end{bmatrix}$ of size $(n+1)\times n$ and $R_{n+1} = \begin{bmatrix}
    \color{red}G & FG & \cdots & F^nG
\end{bmatrix}$ of size $n\times (n+1)$.

\paragraph{Step 3} \label{4SID-NF:step3} $H$, $F$, $G$ estimation.

Using $O_{n+1}$ and $R_{n+1}$ we can easily find:
\begin{align*}
    \hat{G} =  \color{red} R_{n+1}(\texttt{:;1}) & \quad\text{(first column of $R_{n+1}$)} \\
    \hat{H} =  \color{blue} O_{n+1}(\texttt{1;:}) & \quad\text{(first row of $O_{n+1}$)} \\
\end{align*}

To estimate $\hat{F}$ consider $O_{n+1}$ (or, similarly, $R_{n+1}$):\\
define $O'$ as $O_{n+1}$ without the last row, and $O''$ as $O_{n+1}$ without the first row, that is
\[O' = O_{n+1}\texttt{(1:n;:)} = \begin{bmatrix}
        H \\
        HF \\
        \vdots \\
        HF^{n-1}
    \end{bmatrix}\]

\[ O'' = O_{n+1}\texttt{(2:n+1;:)} = \begin{bmatrix}
        HF \\
        HF^2 \\
        \vdots \\
        HF^{n}
    \end{bmatrix}\]

\textbf{Note} $O'$ and $O''$ are $n\times n$ matrices.

\textbf{Note} It is easy to verify that $O'$ and $O''$ follow the \emph{shift-invariance} property: $O''= O' F$


Thanks to the latter property, if we assume that $O'$ is invertible, we can find $\hat{F} = (O')^{-1}O''$.

\paragraph{Conclusions}
In conclusion, in a simple and \emph{constructive} (i.e. NON parametric) way we have estimated a \acrlong{ss} model of the system starting from measured \emph{noise-free} \gls{ir}, using only $2n+1$ samples of \gls{ir}. \\
In practice $n \ll N$, where, again, $n$ is the order of the system and $N$ the number of sample in the dataset.

\[ \left\{\hat{H}, \hat{G}, \hat{F}\right\} = \left\{O_{n+1}(\texttt{1;:}), R_{n+1}(\texttt{:;1}), (O')^{-1}O'' \right\}\]

\begin{rem}
    If the measurement is noisy all this process is \textbf{useless}. Also, if $n$ is unknown, \nameref{4SID-NF:step1} could never stop.
\end{rem}

\section{4SID procedure (with Noise)} \label{sec:4SID-N}

In reality, the measurements of \gls{ir} are noisy: $\tilde{\omega}(t) = \omega(t) + \eta(t)$. Therefore, the \textbf{real problem} is the estimation of the system taking into account the noise; indeed, the available dataset will be $\tilde{\omega}(t) = \{\tilde{\omega}(0), \tilde{\omega}(1), \cdots, \tilde{\omega}(N)\}$.  

\begin{figure}[H]
    \begin{minipage}[t]{0.5\textwidth}
        \centering
        \begin{tikzpicture}[node distance=2.5cm,auto,>=latex']
            \draw[->] (-0.5,0) -- (3,0) node[right] {$t$};
            \draw[->] (0,-0.5) -- (0,1.7) node[left] {$u(t)$};
            \draw[domain=-0.3:0,smooth,variable=\x,red] plot ({\x},{0});
            \draw[domain=0:0.2,smooth,variable=\x,red] plot ({\x},{1.5});
            \draw[domain=0.2:2.5,smooth,variable=\x,red] plot ({\x},{0});
        \end{tikzpicture}
        \caption*{Impulse in input}
    \end{minipage}
    \begin{minipage}[t]{0.5\textwidth}
        \centering
        \begin{tikzpicture}[node distance=2.5cm,auto,>=latex']
            \draw[->] (-0.5,0) -- (3,0) node[right] {$t$};
            \draw[->] (0,-0.5) -- (0,1.7) node[left] {$y(t)$};
            \draw[domain=0:2.5,smooth,variable=\x,red] plot ({\x},{cos(\x*180/3.14*5)*e^(-\x)});
            \draw[samples=100, domain=0:2.5,smooth,variable=\x,blue] plot ({\x},{cos(\x*180/3.14*5)*e^(-\x)+rand/4});
        \end{tikzpicture}
        \caption*{\gls{ir}}
    \end{minipage}
\end{figure}


