%%%%%%%%%%%%%%%%%%%%%%%%%%%%%%%%%%%%%%%%%
% Template Dispense
% Autore: Teo Bucci
%%%%%%%%%%%%%%%%%%%%%%%%%%%%%%%%%%%%%%%%%

%----------------------------------------------------------------------------------------
%	FONTS AND LANGUAGE
%----------------------------------------------------------------------------------------

\usepackage[T1]{fontenc} % Use 8-bit encoding that has 256 glyphs
\usepackage[utf8]{inputenc} % Required for including letters with accents
\usepackage[english]{babel} % Italian language/hyphenation

%----------------------------------------------------------------------------------------
%	PACKAGES
%----------------------------------------------------------------------------------------

\usepackage{amsmath,amssymb,amsthm} % amssymb carica anche amsfonts

\usepackage{dsfont} % per funzione indicatrice
\newcommand{\Ind}{\mathds{1}}
%\usepackage{tikz-among-us}

\usepackage{booktabs}
\usepackage{pgfplots}
\usepackage{tikz}
\usetikzlibrary{arrows}
\usetikzlibrary{positioning}
\usetikzlibrary{shapes.misc}
\usetikzlibrary{intersections}
\usetikzlibrary{shapes.symbols}
\usetikzlibrary{through,backgrounds,fit}
\usetikzlibrary{arrows.meta,decorations.pathreplacing,decorations.pathmorphing,matrix,positioning,calc,shapes,patterns}
\usepackage{mathdots}
\usepackage{cancel}
\usepackage{color}
\usepackage{siunitx}
\usepackage{array}
\usepackage{multirow}
\usepackage{makecell}
\usepackage{tabularx}
%\usepackage{booktabs}
\usepackage{caption}\captionsetup{belowskip=12pt,aboveskip=4pt}
\usepackage{subcaption}
\usepackage{placeins} % The placeins package gives the command \FloatBarrier, which will make sure any floats will be put in before this point
\usepackage{flafter}  % The flafter package ensures that floats don't appear until after they appear in the code.

\usepackage{import}
\usepackage{pdfpages}
\usepackage{transparent}
\usepackage{xcolor}
\usepackage{graphicx}
\graphicspath{ {./img/} }
\usepackage{float}
\usepackage[italian]{varioref}

\newcommand{\fg}[3][\relax]{%
  \begin{figure}[H]%[htp]%
    \centering
    \captionsetup{width=0.7\textwidth}
      \includegraphics[width = #2\textwidth]{#3}%
      \ifx\relax#1\else\caption{#1}\fi
      \label{#3}
  \end{figure}%
  \FloatBarrier%
}

%----------------------------------------------------------------------------------------
%	PARAGRAFI, INTERLINEA E MARGINE
%----------------------------------------------------------------------------------------

\usepackage[none]{hyphenat} % Per non far andare a capo le parole con il trattino

\emergencystretch 3em % Per evitare che il testo vada oltre i margini

% \parindent 0ex % Toglie intendamento paragrafi, e' incluso nel pacchetto \parskip
% \setlength{\parindent}{4em} % Variante di quello sopra

% \setlength{\parskip}{\baselineskip} % Cambia spazio tra paragrafi (posso mettere anche 1em) inclusa la table of contents, pertanto uso il comando che segue

\usepackage[skip=0.2\baselineskip+2pt]{parskip}

% \renewcommand{\baselinestretch}{1.5} % Cambia interlinea

%----------------------------------------------------------------------------------------
%	HEADERS AND FOOTERS
%----------------------------------------------------------------------------------------

\usepackage{fancyhdr}

\pagestyle{empty} % Il fancy serve a partire al primo capitolo
\fancyhead{} % Pulisci header
\fancyfoot{} % Pulisci footer

\fancyhead[RE]{\nouppercase{\leftmark}}
%\fancyhead[LO]{}
\fancyhead[LO]{\nouppercase{\rightmark}}
\fancyhead[LE,RO]{\thepage}

% Removes the header from odd empty pages at the end of chapters
\makeatletter
\renewcommand{\cleardoublepage}{
\clearpage\ifodd\c@page\else
\hbox{}
\vspace*{\fill}
\thispagestyle{empty}
\newpage
\fi}

%----------------------------------------------------------------------------------------
%	COMANDI PERSONALIZZATI
%----------------------------------------------------------------------------------------

\newcommand{\Tau}{\mathcal{T}}
\newcommand{\myNewEmptyPage}{\newpage $ $}
\DeclareMathOperator{\sgn}{sgn}
\newcommand{\degree}{^\circ\text{C}} % SIMBOLO GRADI
\newcommand{\notimplies}{\mathrel{{\ooalign{\hidewidth$\not\phantom{=}$\hidewidth\cr$\implies$}}}}
\renewcommand{\qed}{\tag*{$\blacksquare$}}
\newcommand{\latex}{\LaTeX\xspace}
\newcommand{\tex}{\TeX\xspace}
\newcommand{\questeq}{\overset{?}{=}} % è vero che?
\newcommand{\complementary}{^{\mathrm{C}}}
\newcommand{\transpose}{^{\mathrm{T}}}
\renewcommand{\emptyset}{\varnothing} % simbolo insieme vuoto
\newcommand{\Bot}{\perp \!\!\! \perp} % indipendenza
\renewcommand{\tilde}{\widetilde}
\renewcommand{\hat}{\widehat}
\renewcommand{\theta}{\vartheta}
\newcommand{\boxedText}[1]{\noindent\fbox{\parbox{\textwidth-0.5cm}{#1}}}

\usepackage{mathtools} % Serve per i comandi dopo
\DeclarePairedDelimiter{\abs}{\left\lvert}{\right\rvert}
\DeclarePairedDelimiter{\norm}{\left\lVert}{\right\rVert}
\DeclarePairedDelimiter{\sca}{\left\langle}{\right\rangle}

\newcommand{\Mc}{\mathcal{M}} %model class
\newcommand{\Sys}{\mathcal{S}} %system
\newcommand{\E}{\mathbb{E}}
\newcommand{\R}{\mathbb{R}}
\newcommand{\N}{\mathbb{N}}
\newcommand{\C}{\mathbb{C}}
\newcommand{\Z}{\mathbb{Z}}
%\DeclarePairedDelimiter{\fou}{\mathcal{F}}{}
%\DeclarePairedDelimiter{\ifou}{\mathcal{F}^{-1}\left\{}{\right\}}


%----------------------------------------------------------------------------------------
%	APPENDICE
%----------------------------------------------------------------------------------------

\usepackage[toc,page]{appendix}

%----------------------------------------------------------------------------------------
%	SIMBOLI CARTE DA GIOCO
%
% Sono i seguenti:
%   \varheartsuit
%   \vardiamondsuit
%   \clubsuit
%   \spadesuit
%----------------------------------------------------------------------------------------

\DeclareSymbolFont{extraup}{U}{zavm}{m}{n}
\DeclareMathSymbol{\varheartsuit}{\mathalpha}{extraup}{86}
\DeclareMathSymbol{\vardiamondsuit}{\mathalpha}{extraup}{87}

%----------------------------------------------------------------------------------------

\definecolor{grey245}{RGB}{245,245,245}

\newtheoremstyle{blacknumbox} % Theorem style name
{0pt}% Space above
{0pt}% Space below
{\normalfont}% Body font
{}% Indent amount
{\bf\scshape}% Theorem head font --- {\small\bf}
{.\;}% Punctuation after theorem head
{0.25em}% Space after theorem head
{\small\thmname{#1}\nobreakspace\thmnumber{\@ifnotempty{#1}{}\@upn{#2}}% Theorem text (e.g. Theorem 2.1)
%{\small\thmname{#1}% Theorem text (e.g. Theorem)
\thmnote{\nobreakspace\the\thm@notefont\normalfont\bfseries---\nobreakspace#3}}% Optional theorem note

% Per gli unnumbered tolgo il \nobreakspace subito dopo {\small\thmname{#1} perché altrimenti c'è uno spazio tra Teorema e il ".", lo spazio lo voglio solo se sono numerati per distanziare Teorema e "(2.1)"
\newtheoremstyle{unnumbered} % Theorem style name
{0pt}% Space above
{0pt}% Space below
{\normalfont}% Body font
{}% Indent amount
{\bf\scshape}% Theorem head font --- {\small\bf}
{.\;}% Punctuation after theorem head
{0.25em}% Space after theorem head
{\small\thmname{#1}\thmnumber{\@ifnotempty{#1}{}\@upn{#2}}% Theorem text (e.g. Theorem 2.1)
%{\small\thmname{#1}% Theorem text (e.g. Theorem)
\thmnote{\nobreakspace\the\thm@notefont\normalfont\bfseries---\nobreakspace#3}}% Optional theorem note

\newcounter{dummy} 
\numberwithin{dummy}{chapter}

\theoremstyle{blacknumbox}
\newtheorem{definitionT}[dummy]{Definition}
\newtheorem{theoremT}[dummy]{Theorem}

\theoremstyle{unnumbered}
\newtheorem*{ossT}{Observation}
\newtheorem*{remarkT}{Remark}
\newtheorem*{recallT}{Recall}
\newtheorem*{exT}{Example}
\newtheorem*{dimT}{Proof}

\RequirePackage[framemethod=default]{mdframed} % Required for creating the theorem, definition, exercise and corollary boxes

% orange box
\newmdenv[skipabove=7pt,
skipbelow=7pt,
rightline=false,
leftline=true,
topline=false,
bottomline=false,
linecolor=orange,
backgroundcolor=orange!5,
innerleftmargin=5pt,
innerrightmargin=5pt,
innertopmargin=5pt,
leftmargin=0cm,
rightmargin=0cm,
linewidth=2pt,
innerbottommargin=5pt]{oBox}

% green box
\newmdenv[skipabove=7pt,
skipbelow=7pt,
rightline=false,
leftline=true,
topline=false,
bottomline=false,
linecolor=green,
backgroundcolor=green!0,
innerleftmargin=5pt,
innerrightmargin=5pt,
innertopmargin=5pt,
leftmargin=0cm,
rightmargin=0cm,
linewidth=2pt,
innerbottommargin=5pt]{gBox}

% blue box
\newmdenv[skipabove=7pt,
skipbelow=7pt,
rightline=false,
leftline=true,
topline=false,
bottomline=false,
linecolor=blue,
backgroundcolor=blue!0,
innerleftmargin=5pt,
innerrightmargin=5pt,
innertopmargin=5pt,
leftmargin=0cm,
rightmargin=0cm,
linewidth=2pt,
innerbottommargin=5pt]{bBox}

% dim box
\newmdenv[skipabove=7pt,
skipbelow=7pt,
rightline=false,
leftline=true,
topline=false,
bottomline=false,
linecolor=black,
backgroundcolor=grey245!0,
innerleftmargin=5pt,
innerrightmargin=5pt,
innertopmargin=5pt,
leftmargin=0cm,
rightmargin=0cm,
linewidth=2pt,
innerbottommargin=5pt]{dimBox}

% recall box
\newmdenv[skipabove=7pt,
skipbelow=7pt,
rightline=false,
leftline=true,
topline=false,
bottomline=false,
linecolor=gray,
backgroundcolor=gray!5,
innerleftmargin=5pt,
innerrightmargin=5pt,
innertopmargin=5pt,
leftmargin=0cm,
rightmargin=0cm,
linewidth=2pt,
innerbottommargin=5pt]{recallBox}

\newenvironment{theorem}{\begin{gBox}\begin{theoremT}}{\end{theoremT}\end{gBox}}	
\newenvironment{definition}{\begin{bBox}\begin{definitionT}}{\end{definitionT}\end{bBox}}	
\newenvironment{obs}{\begin{dimBox}\begin{ossT}}{\end{ossT}\end{dimBox}}
\newenvironment{remark}{\begin{dimBox}\begin{remarkT}}{\end{remarkT}\end{dimBox}}
\newenvironment{recall}{\begin{recallBox}\begin{recallT}}{\end{recallT}\end{recallBox}}
\newenvironment{example}{\begin{oBox}\begin{exT}}{\end{exT}\end{oBox}}
\newenvironment{Proof}{\begin{dimBox}\begin{dimT}}{\[\qed\]\end{dimT}\end{dimBox}}

%----------------------------------------------------------------------------------------
%	INDICE INIZIALE
%----------------------------------------------------------------------------------------

\setcounter{secnumdepth}{3} % DI DEFAULT LE SUBSECTION NON SONO NUMERATE, COSÌ SÌ
\setcounter{tocdepth}{2} % FISSA LA PROFONDITÀ DELLE COSE MOSTRATE NELL'INDICE

\usepackage[hidelinks]{hyperref} % Rende l'indice interattivo e hidelinks nasconde il bordo rosso dai riferimenti

\usepackage{glossaries} % certain packages that must be loaded before glossaries, if they are required: hyperref, babel, polyglossia, inputenc and fontenc
\setacronymstyle{long-short}

% Questo comando e quello dopo servono per avere il comando \tocless da mettere prima di una sezione che non voglio far apparire nell'indice
\newcommand{\nocontentsline}[3]{}
\newcommand{\tocless}[2]{\bgroup\let\addcontentsline=\nocontentsline#1{#2}\egroup}

%\usepackage{showframe}
\usepackage[textsize=tiny, textwidth=1.5cm]{todonotes} %add disable to options to not show in pdf